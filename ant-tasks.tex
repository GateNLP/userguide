%%%%%%%%%%%%%%%%%%%%%%%%%%%%%%%%%%%%%%%%%%%%%%%%%%%%%%%%%%%%%%%%%%%%%%%%%%%%
\chapt[chap:ant]{Ant Tasks for GATE}
\markboth{Ant Tasks for GATE}{Ant Tasks for GATE}
%%%%%%%%%%%%%%%%%%%%%%%%%%%%%%%%%%%%%%%%%%%%%%%%%%%%%%%%%%%%%%%%%%%%%%%%%%%%%
\nnormalsize
%%%% qqqqqqqqqqqqqqqqqqqqqqqqq %%%%

This \chapthing\ describes the Ant tasks provided by GATE that you can use in
your own build files.  The tasks require Ant 1.7 or later.

%%%%%%%%%%%%%%%%%%%%%%%%%%%%%%%%%%%%%%%%%%%%%%%%%%%%%%%%%%%%%%%%%%%%%%%%%%%%%
\sect[sec:ant:declare]{Declaring the Tasks}
%%%%%%%%%%%%%%%%%%%%%%%%%%%%%%%%%%%%%%%%%%%%%%%%%%%%%%%%%%%%%%%%%%%%%%%%%%%%%

To use the GATE Ant tasks in your build file you must include the following
\verb|<typedef>| (where \verb|${gate.home}| is the location of your GATE
installation):

\begin{small}
\begin{verbatim}
<typedef resource="gate/util/ant/antlib.xml">
  <classpath>
    <pathelement location="${gate.home}/bin/gate.jar" />
    <fileset dir="${gate.home}/lib" includes="*.jar" />
  </classpath>
</typedef>
\end{verbatim}
\end{small}

If you have problems with library conflicts you should be able to reduce the
JAR files included from the lib directory to just jdom, xstream and jaxen.

%%%%%%%%%%%%%%%%%%%%%%%%%%%%%%%%%%%%%%%%%%%%%%%%%%%%%%%%%%%%%%%%%%%%%%%%%%%%%
\sect[sec:ant:packagegapp]{The \texttt{packagegapp} task - bundling an
application with its dependencies}
%%%%%%%%%%%%%%%%%%%%%%%%%%%%%%%%%%%%%%%%%%%%%%%%%%%%%%%%%%%%%%%%%%%%%%%%%%%%%

\subsect{Introduction}

GATE saved application states (GAPP files) are an XML representation of the
state of a GATE application.  One of the features of a GAPP file is that it
holds references to the external resource files used by the application as
paths relative to the location of the GAPP file itself (or relative to the
location of the GATE home directory where appropriate).  This is useful in many
cases but if you want to package up a copy of an application to send to a third
party or to use in a web application, etc., then you need to be very careful to
save the file in a directory above \emph{all} its resources, and package the
resources up with the GAPP file at the same relative paths.  If the application
refers to resources outside its own file tree (i.e. with relative paths that
include \texttt{..}) then you must either maintain this structure or manually
edit the XML to move the resource references around and copy the files to the
right places to match.  This can be quite tedious and error-prone...

The \texttt{packagegapp} Ant task aims to automate this process.  It extracts
all the relative paths from a GAPP file, writes a modified version of the file
with these paths rewritten to point to locations below the new GAPP file
location (i.e. with no \texttt{..} path segments) and copies the referenced
files to their rewritten locations.  The result is a directory structure that
can be easily packaged into a zip file or similar and moved around as a
self-contained unit.

This Ant task is the underlying driver for the `Export for GATE Cloud' option
described in Section~\ref{sec:developer:export}.  Export for GATE Cloud does the
equivalent of:
\begin{small}
\begin{verbatim}
<packagegapp src="sourceFile.gapp"
             destfile="{tempdir}/application.xgapp"
             copyPlugins="yes"
             copyResourceDirs="yes"
             onUnresolved="recover" />
\end{verbatim}
\end{small}
followed by packaging the temporary directory into a zip file.  These options
are explained in detail below.

The \texttt{packagegapp} task requires Ant 1.7 or later.

\subsect[sec:ant:packagegapp:basic]{Basic Usage}

In many cases, the following simple invocation will do what you want:

\begin{small}
\begin{verbatim}
<packagegapp src="original.xgapp"
             gatehome="/path/to/GATE"
             destfile="package/target.xgapp" />
\end{verbatim}
\end{small}

Note that the parent directory of the \texttt{destfile} (in this case
\texttt{package}) must already exist.  It will not be created automatically.
The value for the \verb|gatehome| attribute should be the path to your GATE
installation (the directory containing \verb|build.xml|, the \verb|bin|,
\verb|lib| and \verb|plugins| directories, etc.).  If you know that the gapp
file you want to package does not reference any resources relative to the GATE
home directory\footnote{You can check this by searching for the string
``\$gatehome\$'' in the XML} then this attribute may be omitted.

This will perform the following steps:
\begin{enumerate}
\item Read in the \texttt{original.xgapp} file and extract all the relative
  paths it contains.
\item For each plugin referred to by a relative path,
  \texttt{foo/bar/MyPlugin}, rewrite the plugin location to be
  \texttt{plugins/MyPlugin} (relative to the location of the
  \texttt{destfile}).  If the application refers to two plugins in different
  original locations with the same name, one of them will be renamed to avoid a
  name clash.  If one plugin is a subdirectory of another plugin, this nesting
  will be maintained in the relocated directory structure.
\item For each resource file referred to by the gapp, see if it lives under the
  original location of one of the plugins moved in the previous step.  If so,
  rewrite its location relative to the new location of the plugin.
\item If there are any relative resource paths that are not accounted for by
  the above rule (i.e. they do not live inside a referenced plugin), the build
  fails (see Section~\ref{sec:ant:packagegapp:unresolved} for how to change
  this behaviour).
\item Write out the modified GAPP to the \texttt{destfile}.
\item Recursively copy the whole content of each of the plugins from step 2 to
  their new locations\footnote{This is done with an Ant copy task and so is
  subject to the normal defaultexcludes}.
\end{enumerate}

This means that the all the relative paths in the new GAPP file
(\texttt{package/target.xgapp}) will point to \texttt{plugins/Something}.  You
can now bundle up the whole \texttt{package} directory and take it elsewhere.

\subsect{Handling Non-Plugin Resources}

By default, the task only handles relative resource paths that point within one
of the plugins that the GAPP refers to.  However, many applications refer to
resources that live outside the plugin directories, for example custom JAPE
grammars, gazetteer lists, etc.  The task provides two approaches to support
this: it can handle the unresolved references automatically, or you can provide
your own `hints' to augment the default plugin-based ones.

\subsubsect[sec:ant:packagegapp:unresolved]{Resolving Unresolved Resources}

By default, the build will fail if there are any relative paths that cannot be
accounted for by the plugins (or the explicit hints, see
Section~\ref{sec:ant:packagegapp:hints}).
However, this is configurable using the \texttt{onUnresolved} attribute, which can
take the following values:

\begin{description}
\item[fail] (default) the build fails if an unresolved relative path is found.
\item[absolute] unresolved relative paths are left pointing to the same location
  as in the original file, but as an \emph{absolute} rather than a relative URL.
  The same file will be used even if you move the GAPP file to a different
  directory.  This option is useful if the resource in question is visible at
  the same absolute location on the machine where you will be putting the
  packaged file (for example a very large dictionary or ontology held on a
  network share).
\item[recover] attempt to recover gracefully (see below).
\end{description}

With \verb|onUnresolved="recover"|, unresolved resources are relocated to a
directory named \texttt{application-resources} under the target GAPP file
location.  Resources in the same original directory are copied to the same
subdirectory of \texttt{application-resources}, files from different original
directories are copied to different subdirectories.  Typically, for a resource
whose original location was \texttt{.../myresources/grammar/clever.jape} the
target location would be \texttt{application-resources/grammar/clever.jape} but
if the application also referred to (say)
\texttt{.../otherresources/grammar/clean.jape} then this would be
mapped into \texttt{application-resources/grammar-2} to avoid a name clash.

As with plugins, if one unresolved resource is contained in a subdirectory of a
directory containing another unresolved resource, the relative path will be
preserved, i.e. if the application refers to \texttt{.../dictionaries/main.txt}
and also \texttt{.../dictionaries/specialist/medical.txt} then the latter will
be relocated to \texttt{application-resources/dictionaries/specialist} rather
than simply creating another top-level
\texttt{application-resources/specialist} directory.  This is particularly
relevant when using the \texttt{copyResourceDirs} option described below.

Example:
\begin{small}
\begin{verbatim}
<packagegapp src="original.xgapp" destfile="package/target.xgapp"
             onUnresolved="recover" />
\end{verbatim}
\end{small}

\subsubsect[sec:ant:packagegapp:hints]{Providing Mapping Hints}

By default, the task knows how to handle resources that live inside plugins.
You can think of this as a `hint' \texttt{/foo/bar/MyPlugin ->
plugins/MyPlugin}, saying that whenever the mapper finds a resource path of
the form \texttt{/foo/bar/MyPlugin/\emph{X}}, it will relocate it to
\texttt{plugins/MyPlugin/\emph{X}} relative to the output GAPP file.  You can
specify your own hints which will be used the same way.

\begin{small}
\begin{verbatim}
<packagegapp src="original.xgapp" destfile="package/target.xgapp">
  <hint from="${user.home}/my-app-v1" to="resources/my-app" />
  <hint from="/share/data/bigfiles" absolute="yes" />
</packagegapp>
\end{verbatim}
\end{small}

In this example, \verb|~/my-app-v1/grammar/main.jape| would be mapped to the
location \verb|resources/my-app/grammar/main.jape| (as always, relative to the
output GAPP file).  You can also hint that certain resources should be
converted to absolute paths rather than being packaged with the application,
using \verb|absolute="yes"|.  The \texttt{from} and \texttt{to} values refer to
directories - you cannot hint a single file, nor put two files from the same
original directory into different directories in the packaged GAPP.

Explicit hints override the default plugin-based hints.  For example given the
hint \verb|from="${gate.home}/plugins/ANNIE/resources" to="resources/ANNIE"|,
resources in the ANNIE plugin would be mapped into
\texttt{resources/ANNIE}, but the plugin \texttt{creole.xml} itself would still
be mapped into \texttt{plugins/ANNIE}.

As well as providing the hints inline in the build file you can also read them
from a file in the normal Java Properties format\footnote{the hint tag supports
all the attributes of the standard Ant property tag so can load the hints from
a file on disk or from a resource in a JAR file}, using

\begin{verbatim}
<hint file="hints.properties" />
\end{verbatim}

The keys in the property file are the \texttt{from} paths (in this case,
relative paths are resolved against the project base directory, as with the
\texttt{location} attribute of a \texttt{property} task) and the values are the
\texttt{to} paths relative to the output file location.

The order of the \verb|<hint>| elements is significant -- if more than one hint
could apply to the same resource file, the one defined first is used.  For
example, given the hints

\begin{small}
\begin{verbatim}
<hint from="${gate.home}/plugins/ANNIE/resources/tokeniser" to="tokeniser" />
<hint from="${gate.home}/plugins/ANNIE/resources" to="annie" />
\end{verbatim}
\end{small}

the resource \texttt{plugins/ANNIE/resources/tokeniser/DefaultTokeniser.rules}
would be mapped into the \texttt{tokeniser} directory, but if the hints were
reversed it would instead be mapped into \texttt{annie/tokeniser}.  Note,
however, that this does not necessarily extend to hints loaded from property
files, as the order in which hints from a single property file are applied is
not specified.  Given

\begin{small}
\begin{verbatim}
<hint file="file1.proeprties" />
<hint file="file2.properties" />
\end{verbatim}
\end{small}

the relative precedence of two hints from file1 is not fixed, but it is the
case that all hints in file1 will be applied before those in file2.

\subsect[sec:ant:packagegapp:streamline]{Streamlining your Plugins}

By default, the task will recursively copy the whole content of every plugin
into the target directory.  In most cases this is OK but it may be the case
that your plugins contain many extraneous resources that are not used by your
application.  In this case you can specify \verb|copyPlugins="no"|:
\begin{small}
\begin{verbatim}
<packagegapp src="original.xgapp" destfile="package/target.xgapp"
             copyPlugins="no" />
\end{verbatim}
\end{small}

In this mode, the packager task will copy only the following files from each
plugin:

\begin{itemize}
\item \texttt{creole.xml}
\item any JAR files referenced from \verb|<JAR>| elements in
\texttt{creole.xml}\footnote{When loading a plugin, the classloader inspects
  the Class-Path attribute in each JAR file's manifest and also loads the JARs
  that this references.  However the packager task does {\bf not} do this, so
  if you use the manifest mechanism with your plugins you will need to
  explicitly reference the additional JAR files using an
  \texttt{extraresourcespath}.}
\end{itemize}

In addition it will of course copy any files \emph{directly} referenced by the
GAPP, but not files referenced indirectly (the classic examples being
\texttt{.lst} files used by a gazetteer \texttt{.def}, or the individual phases
of a multiphase JAPE grammar) or files that are referenced by the
\texttt{creole.xml} itself as \texttt{AUTOINSTANCE} parameters (e.g. the
annotation schemas in ANNIE).  You will need to name these extra files
explicitly as extra resources (see the next section).

\subsect[sec:ant:packagegapp:extraresources]{Bundling Extra Resources}

Apart from plugins (when you don't use \verb|copyPlugins="no"|), the only files
copied into the target directory are those that are referenced directly from
the GAPP file.  This is often but not always sufficient, for example if your
application contains a multiphase JAPE transducer then \texttt{packagegapp} will
include the main JAPE file but not the individual phase files.  The task provides two ways to include extra files in the package:

\begin{itemize}
\item If you set the attribute \verb|copyResourceDirs="yes"| on the
  \texttt{packagegapp} task then whenever the task packages a referenced
  resource file it will also recursively include the whole contents of the
  directory containing that file in the output package.  You probably don't
  want to use this option if you have resource files in a directory shared with
  other files (e.g. your home directory...).
\item To include specific extra resources you can use an
  \verb|<extraresourcespath>| (see below).
\end{itemize}

The \verb|<extraresourcespath>| allows you to specify specific extra files that should be included in the package:

\begin{small}
\begin{verbatim}
<packagegapp src="original.xgapp" destfile="package/target.xgapp">
  <extraresourcespath>
    <pathelement location="${user.home}/common-files/README" />
    <fileset dir="${user.home}/my-app-v1" includes="grammar/*.jape" />
  </extraresourcespath>
</packagegapp>
\end{verbatim}
\end{small}

As the name suggests, this is a
\htlink{http://ant.apache.org/manual/using.html\#path}{path-like structure} and
supports all the usual elements and attributes of an Ant \verb|<path>|,
including multiple nested \texttt{fileset}, \texttt{filelist},
\texttt{pathelement} and other \texttt{path} elements.  For specific types of
indirect references, there are helper elements that can be included under
\texttt{extraresourcespath}.  Currently the only one of these is
\texttt{gazetteerlists}, which takes the path to a gazetteer definition file
and returns the set of \texttt{.lst} files the definition uses:
\begin{verbatim}
<gazetteerlists definition="my/resources/lists.def" encoding="UTF-8" />
\end{verbatim}
Other helpers (e.g. for multiphase JAPE) may be implemented in future.

You can also refer to a path defined elsewhere in the usual way:

\begin{small}
\begin{verbatim}
<path id="extra.files">
  ...
</path>

<packagegapp ...>
  <extraresourcespath refid="extra.files" />
</packagegapp>
\end{verbatim}
\end{small}

Resources declared in the \texttt{extraresourcespath} and directories included
using\linebreak \verb|copyResourceDirs| are treated exactly the same as resources that
are referenced by the GAPP file - their target locations in the package are
determined by the mapping hints, default plugin-based hints, and the
\texttt{onUnresolved} setting as above.  If you want to put extra resource
files at specific locations in the package tree, independent of the mapping
hints mechanism, you should do this with a separate \verb|<copy>| task after
the \verb|<packagegapp>| task has done its work.

%%%%%%%%%%%%%%%%%%%%%%%%%%%%%%%%%%%%%%%%%%%%%%%%%%%%%%%%%%%%%%%%%%%%%%%%%%%%%
\sect[sec:ant:expandcreoles]{The \texttt{expandcreoles} Task - Merging
Annotation-Driven Config into creole.xml}
%%%%%%%%%%%%%%%%%%%%%%%%%%%%%%%%%%%%%%%%%%%%%%%%%%%%%%%%%%%%%%%%%%%%%%%%%%%%%

The \texttt{expandcreoles} task processes a number of creole.xml files from
plugins, processes any \verb|@CreoleResource| and \verb|@CreoleParameter|
annotations on the declared resource classes, and merges this configuration
with the original XML configuration into a new copy of the creole.xml.  It is
not necessary to do this in the normal use of GATE, and this task is documented
here simply for completeness.  It is intended simply for use with non-GATE
tools that can process the creole.xml file format to extract information about
plugins (the prime use case for this is to generate the
\htlink{http://gate.ac.uk/gate/doc/plugins.html}{GATE plugins information page}
automatically from the plugin definitions).

The typical usage of this task (taken from the GATE build.xml) is:

\begin{small}
\begin{verbatim}
<expandcreoles todir="build/plugins" gatehome="${basedir}">
  <fileset dir="plugins" includes="*/creole.xml" />
</expandcreoles>
\end{verbatim}
\end{small}

This will initialise GATE with the given GATE\_HOME directory, then read each
file from the nested fileset, parse it as a creole.xml, expand it from any
annotation configuration, and write it out to a file under
\texttt{build/plugins}.  Each output file will be generated at the same
location relative to the \texttt{todir} as the original file was relative to
the \texttt{dir} of its \texttt{fileset}.

