%%%%%%%%%%%%%%%%%%%%%%%%%%%%%%%%%%%%%%%%%%%%%%%%%%%%%%%%%%%%%%%%%%%%%%%%%%%%% %
% uima.tex
%
% Ian Roberts, November 2005
%
% $Id: uima.tex,v 1.3 2006/10/21 11:44:47 ian Exp $
%
%%%%%%%%%%%%%%%%%%%%%%%%%%%%%%%%%%%%%%%%%%%%%%%%%%%%%%%%%%%%%%%%%%%%%%%%%%%%%


%%%%%%%%%%%%%%%%%%%%%%%%%%%%%%%%%%%%%%%%%%%%%%%%%%%%%%%%%%%%%%%%%%%%%%%%%%%%%
\chapt[chap:uima]{Combining GATE and UIMA}
\markboth{Combining GATE and UIMA}{Combining GATE and UIMA}
%%%%%%%%%%%%%%%%%%%%%%%%%%%%%%%%%%%%%%%%%%%%%%%%%%%%%%%%%%%%%%%%%%%%%%%%%%%%%
\nnormalsize

\htlink{http://incubator.apache.org/uima/}{UIMA} (Unstructured Information
Management Architecture) is a platform for natural language processing,
originally developed by IBM but now maintained by the Apache Software
Foundation.  It has many similarities to the GATE architecture -- it represents
documents as text plus annotations, and allows users to define pipelines of
\emph{analysis engines} that manipulate the document (or \emph{Common Analysis
Structure} in UIMA terminology) in much the same way as processing resources do
in GATE.  The Apache UIMA SDK provides support for building analysis components
in Java and C++ and running them either locally on one machine, or deploying
them as services that can be accessed remotely.  The SDK is available for
download from \htlinkplain{http://incubator.apache.org/uima/}.

Clearly, it would be useful to be able to include UIMA components in GATE
applications and vice-versa, letting GATE users take advantage of UIMA's
flexible deployment options and UIMA users access JAPE and the many useful
plugins already available in GATE.  This \chapthing\ describes the
interoperability layer provided as part of GATE to support this.
The UIMA-GATE interoperability layer is based on Apache UIMA 2.2.2.  GATE 5.0
and earlier included an implementation based on version 1.2.3 of the pre-Apache
IBM UIMA SDK.

The rest of this \chapthing\ assumes that you have at least a basic
understanding of core UIMA concepts, such as \emph{type systems},
\emph{primitive} and \emph{aggregate analysis engines} (AEs),
\emph{feature structures}, the format of AE XML descriptors, etc.  It will
probably be helpful to refer to the relevant sections of the UIMA SDK User's
Guide and Reference (supplied with the SDK) alongside this document.

There are two main parts to the interoperability layer:
%
\begin{enumerate}
\item A wrapper to allow a UIMA Analysis Engine (AE), whether primitive
or aggregate, to be used within GATE as a Processing Resource (PR).
\item A wrapper to allow a GATE processing pipeline (specifically a
\texttt{CorpusController}) to be used within UIMA as an AE.
\end{enumerate}
%
The two components operate in very similar ways.  Given a document in the
source form (either a GATE \texttt{Document} or a UIMA \texttt{CAS}), a
document in the target form is created with a copy of the source document's
text.  Some of the annotations from the source are transferred to the target,
according to a mapping defined by the user, and the target component is then
run.  Finally, some of the annotations on the updated target document are then
transferred back to the source, according to the user-defined mapping.

The rest of this document describes this process in more detail.  Section
\ref{sec:uima:UIMAInGATE} describes the GATE AE wrapper, and Section
\ref{sec:uima:GATEInUIMA} describes the UIMA CorpusController wrapper.

%%%%%%%%%%%%%%%%%%%%%%%%%%%%%%%%%%%%%%%%%%%%%%%%%%%%%%%%%%%%%%%%%%%%%%%%%%%%%
\sect[sec:uima:UIMAInGATE]{Embedding a UIMA AE in GATE}
%%%%%%%%%%%%%%%%%%%%%%%%%%%%%%%%%%%%%%%%%%%%%%%%%%%%%%%%%%%%%%%%%%%%%%%%%%%%%

Embedding a UIMA analysis engine in a GATE application is a two step
process.  First, you must construct a \emph{mapping descriptor} XML file to
define how to map annotations between the UIMA CAS and the GATE Document.  This
mapping file, along with the analysis engine descriptor, is used to instantiate
an \emph{AnalysisEnginePR} which calls the analysis engine on an appropriately
initialized CAS.  Examples of all the XML files discussed in this section are
available in \texttt{examples/conf} under the \texttt{UIMA} plugin directory.

%%%%%%%%%%%%%%%%%%%%%%%%%%%%%%%%%%%%%%%%%%%%%%%%%%%%%%%%%%%%%%%%%%%%%%%%%%%%%
\subsect[sec:uima:UIMAInGATE:mapping]{Mapping File Format}
%%%%%%%%%%%%%%%%%%%%%%%%%%%%%%%%%%%%%%%%%%%%%%%%%%%%%%%%%%%%%%%%%%%%%%%%%%%%%

Figure \ref{fig:mappingFile} shows the structure of a mapping descriptor.  The
\texttt{inputs} section defines how annotations on the GATE document are
transferred to the UIMA CAS.  The \texttt{outputs} section defines how
annotations which have been added, updated and removed by the AE are
transferred back to the GATE document.
\begin{figure}
\begin{small}
\begin{verbatim}
<uimaGateMapping>
  <inputs>
    <uimaAnnotation type="..." gateType="..." indexed="true|false">
      <feature name="..." kind="string|int|float|fs">
        <!-- element defining the feature value goes here -->
      </feature>
      ...
    </uimaAnnotation>
  </inputs>

  <outputs>
    <added>
      <gateAnnotation type="..." uimaType="...">
        <feature name="...">
          <!-- element defining the feature value goes here -->
        </feature>
        ...
      </gateAnnotation>
    </added>
    
    <updated>
      ...
    </updated>
    
    <removed>
      ...
    </removed>
  </outputs>
</uimaGateMapping>
\end{verbatim}
\end{small}
\caption{Structure of a mapping descriptor for an AE in GATE}
\label{fig:mappingFile}
\end{figure}

%%%%%%%%%%%%%%%%%%%%%%%%%%%%%%%%%%%%%%%%%%%%%%%%%%%%%%%%%%%%%%%%%%%%%%%%%%%%%
\subsubsect[sec:uima:UIMAInGATE:mapping:inputs]{Input Definitions}
%%%%%%%%%%%%%%%%%%%%%%%%%%%%%%%%%%%%%%%%%%%%%%%%%%%%%%%%%%%%%%%%%%%%%%%%%%%%%

Each input definition takes the following form:
\begin{small}
\begin{verbatim}
  <uimaAnnotation type="uima.Type" gateType="GATEType" indexed="true|false">
    <feature name="..." kind="string|int|float|fs">
      <!-- element defining the feature value goes here -->
    </feature>
    ...
  </uimaAnnotation>
\end{verbatim}
\end{small}
%
When a document is processed, this will create one UIMA annotation of type
\texttt{uima.Type} in the CAS for each GATE annotation of type
\texttt{GATEType} in the input annotation set, covering the same offsets in the
text.  If \texttt{indexed} is \texttt{true}, GATE will keep a record of which
GATE annotation gave rise to which UIMA annotation.  If you wish to be able to
track updates to this annotation's features and transfer the updated values
back into GATE, you must specify \texttt{indexed="true"}.  The \texttt{indexed}
attribute defaults to \texttt{false} if omitted.

Each contained \texttt{feature} element will cause the corresponding feature to
be set on the generated annotation.  UIMA features can be string, integer or
float valued, or can be a reference to another feature structure, and this must
be specified in the \texttt{kind} attribute.  The feature's value is specified
using a nested element, but exactly how this value is handled is determined by
the \texttt{kind}.

There are various options for setting feature values:
\begin{itemize}
\item \verb|<string value="fixed string" />| The simplest case - a fixed Java
String.
\item \verb|<docFeatureValue name="featureName" />| The value of the given
named feature of the current GATE document.
\item \verb|<gateAnnotFeatureValue name="featureName" />| The value of a given
feature on the current GATE annotation (i.e. the one on which the offsets of
the UIMA annotation are based).
\item \verb|<featureStructure type="uima.fs.Type">...</featureStructure>| A
feature structure of the given type.  The \texttt{featureStructure} element can
itself contain \texttt{feature} elements recursively.
\end{itemize}

The value is assigned to the feature according to the feature's \texttt{kind}:
\begin{description}
\item[string] The value object's \texttt{toString()} method is called,
and the resulting String is set as the string value of the feature.
\item[int] If the value object is a subclass of
\texttt{java.lang.Number}, its \texttt{intValue()} method is called, and the
result is set as the integer value of the feature.  If the value object is not
a \texttt{Number}, it is \texttt{toString()}ed, and the resulting String is
parsed using \texttt{Integer.parseInt()}.  If this succeeds, the integer result
is used, if it fails the feature is set to zero.
\item[float] As for \texttt{int}, except that \texttt{Number}s are
converted by calling \texttt{floatValue()}, and non-\texttt{Number}s are parsed
using \texttt{Float.parseFloat()}.
\item[fs] The value object is assumed to be a \texttt{FeatureStructure}, and is
used as-is.  A \texttt{ClassCastException} will result if the value object is
not a \texttt{FeatureStructure}.
\end{description}
%
In particular, \verb|<featureStructure>| value elements should only be used
with features of kind \texttt{fs}.  While nothing will stop you using them with
\texttt{string} features, the result will probably not be what you expected.

%%%%%%%%%%%%%%%%%%%%%%%%%%%%%%%%%%%%%%%%%%%%%%%%%%%%%%%%%%%%%%%%%%%%%%%%%%%%%
\subsubsect[sec:uima:UIMAInGATE:mapping:outputs]{Output Definitions}
%%%%%%%%%%%%%%%%%%%%%%%%%%%%%%%%%%%%%%%%%%%%%%%%%%%%%%%%%%%%%%%%%%%%%%%%%%%%%

The output definitions take a similar form.  There are three groups:
\begin{description}
\item[added] Annotations which have been added by the AE, and for which
corresponding new annotations are to be created in the GATE document.
\item[updated] Annotations that were created by an input definition (with
\texttt{indexed="true"}) whose feature values have been modified by the AE,
and these values are to be transferred back to the original GATE annotations.
\item[removed] Annotations that were created by an input definition (with
\texttt{indexed="true"}) which have been removed from the CAS\footnote{Strictly
speaking, removed from the annotation index, as feature structures cannot be
removed from the CAS entirely.} and whose source annotations are to be removed
from the GATE document.
\end{description}

The definition elements for these three types all take the same form:
\begin{small}
\begin{verbatim}
  <gateAnnotation type="GATEType" uimaType="uima.Type">
    <feature name="featureName">
      <!-- element defining the feature value goes here -->
    </feature>
    ...
  </gateAnnotation>
\end{verbatim}
\end{small}
%
For \texttt{added} annotations, this has the mirror-image effect to the input
definition -- for each UIMA annotation of the given type, create a GATE
annotation at the same offsets and set its feature values as specified by
\texttt{feature} elements.  For a \texttt{gateAnnotation} the \texttt{feature}
elements do not have a \texttt{kind}, as features in GATE can have arbitrary
Objects as values.  The possible feature value elements for a
\texttt{gateAnnotation} are:
\begin{itemize}
\item \verb|<string value="fixed string" />| A fixed string, as before.
\item \verb'<uimaFSFeatureValue name="uima.Type:FeatureName" kind="string|int|float" />'
The value of the given feature of the current UIMA annotation.  The feature
name must be specified in fully-qualified form, including the type on which it
is defined.  The \texttt{kind} is used in a similar way as in input definitions:
  \begin{description}
  \item[string] The Java \texttt{String} object returned as the string value of
  the feature is used.
  \item[int] An \texttt{Integer} object is created from the integer value of
  the feature.
  \item[float] A \texttt{Float} object is created from the float value of the
  feature.
  \item[fs] The UIMA \texttt{FeatureStructure} object is returned.  Since
  \texttt{FeatureStructure} objects are not guaranteed to be valid once the CAS
  has been cleared, a downstream GATE component must extract the relevant
  information from the feature structure before the next document is processed.
  You have been warned.
  \end{description}
\end{itemize}
%
Feature names in \texttt{uimaFSFeatureValue} must be qualified with their type
name, as the feature may have been defined on a supertype of the feature's own
type, rather than the type itself.  For example, consider the following:
\begin{small}
\begin{verbatim}
<gateAnnotation type="Entity" uimaType="com.example.Entity">
  <feature name="type">
    <uimaFSFeatureValue name="com.example.Entity:Type" kind="string" />
  </feature>
  <feature name="startOffset">
    <uimaFSFeatureValue name="uima.tcas.Annotation:begin" kind="int" />
  </feature>
</gateAnnotation>
\end{verbatim}
\end{small}

For \texttt{updated} annotations, there must have been an input definition with
\texttt{indexed="true"} with the same GATE and UIMA types.  In this case, for
each GATE annotation of the appropriate type, the UIMA annotation that was
created from it is found in the CAS.  The feature definitions are then used as
in the \texttt{added} case, but here, the feature values are set on the
\emph{original} GATE annotation, rather than on a newly created annotation.

For \texttt{removed} annotations, the feature definitions are ignored, and the
annotation is removed from GATE if the UIMA annotation which it gave rise to
has been removed from the UIMA annotation index.

%%%%%%%%%%%%%%%%%%%%%%%%%%%%%%%%%%%%%%%%%%%%%%%%%%%%%%%%%%%%%%%%%%%%%%%%%%%%%
\subsubsect[sec:uima:UIMAInGATE:mapping:example]{A Complete Example}
%%%%%%%%%%%%%%%%%%%%%%%%%%%%%%%%%%%%%%%%%%%%%%%%%%%%%%%%%%%%%%%%%%%%%%%%%%%%%

Figure \ref{fig:UIMAInGATE:example} shows a complete example mapping descriptor
for a simple UIMA AE that takes tokens as input and adds a feature to each
token giving the number of lower case letters in the token's
string.\footnote{The Java code implementing this AE is in the \texttt{examples}
directory of the \texttt{UIMA} plugin.  The AE descriptor and mapping file are
in \texttt{examples/conf}.}  In this case the UIMA feature that holds the
number of lower case letters is called \texttt{LowerCaseLetters}, but the GATE
feature is called \texttt{numLower}.  This demonstrates that the feature names
do not need to agree, so long as a mapping between them can be defined.
%
\begin{figure}
\begin{small}
\begin{verbatim}
<uimaGateMapping>
  <inputs>
    <uimaAnnotation type="gate.uima.cas.Token" gateType="Token" indexed="true">
      <feature name="String" kind="string">
        <gateAnnotFeatureValue name="string" />
      </feature>
    </uimaAnnotation>
  </inputs>
  <outputs>
    <updated>
      <gateAnnotation type="Token" uimaType="gate.uima.cas.Token">
        <feature name="numLower">
          <uimaFSFeatureValue name="gate.uima.cas.Token:LowerCaseLetters"
                              kind="int" />
        </feature>
      </gateAnnotation>
    </updated>
  </outputs>
</uimaGateMapping>
\end{verbatim}
\end{small}
\caption{An example mapping descriptor}
\label{fig:UIMAInGATE:example}
\end{figure}

%%%%%%%%%%%%%%%%%%%%%%%%%%%%%%%%%%%%%%%%%%%%%%%%%%%%%%%%%%%%%%%%%%%%%%%%%%%%%
\subsect[sec:uima:UIMAInGATE:descriptor]{The UIMA Component Descriptor}
%%%%%%%%%%%%%%%%%%%%%%%%%%%%%%%%%%%%%%%%%%%%%%%%%%%%%%%%%%%%%%%%%%%%%%%%%%%%%

As well as the mapping file, you must provide the UIMA component descriptor
that defines how to access the AE that is to be called.  This could be a
primitive or aggregate analysis engine descriptor, or a URI specifier giving
the location of a remote Vinci or SOAP service.  It is up to the developer to
ensure that the types and features used in the mapping descriptor are
compatible with the type system and capabilities of the AE, or a runtime error
is likely to occur.

%%%%%%%%%%%%%%%%%%%%%%%%%%%%%%%%%%%%%%%%%%%%%%%%%%%%%%%%%%%%%%%%%%%%%%%%%%%%%
\subsect[sec:uima:UIMAInGATE:using]{Using the \texttt{AnalysisEnginePR}}
%%%%%%%%%%%%%%%%%%%%%%%%%%%%%%%%%%%%%%%%%%%%%%%%%%%%%%%%%%%%%%%%%%%%%%%%%%%%%

To use a UIMA AE in GATE Developer, load the \texttt{UIMA} plugin and
create a `UIMA Analysis Engine' processing resource.  If using the
GATE Embedded, rather than GATE Developer, the class name
is \texttt{gate.uima.AnalysisEnginePR}.  The processing resource
expects two parameters:
\begin{description}
\item[analysisEngineDescriptor] The URL of the UIMA analysis engine descriptor
(or URI specifier, for a remote AE service).  This must be a \texttt{file:}
URL, as UIMA needs a file path against which to resolve imports.
\item[mappingDescriptor] The URL of the mapping descriptor file.  This may be
any kind of URL (\texttt{file:}, \texttt{http:}, \texttt{Class.getResource()},
\texttt{ServletContext.getResource()}, etc.)
\end{description}
%
Any errors processing either of the descriptor files will cause an exception to
be thrown.  Once instantiated, you can add the PR to a pipeline in the usual
way.  \texttt{AnalysisEnginePR} implements \texttt{LanguageAnalyser}, so can be
used in any of the standard GATE pipeline types.

The PR takes the following runtime parameter (in addition to the
\texttt{document} parameter which is set automatically by a
\texttt{CorpusController}):
\begin{description}
\item[annotationSetName] The annotation set to process.  Any input mappings
take annotations from this set, and any output mappings place their new
annotations in this set (\texttt{added} outputs) or update the input
annotations in this set (\texttt{updated} or \texttt{removed}).  If not
specified, the default (unnamed) annotation set is used.
\end{description}

The Annotator implementation must be available for GATE to load.  For an
annotator written in Java, this means that the JAR file containing the
annotator class (and any other classes it depends on) must be present in the
GATE classloader.  The easiest way to achieve this is to put the JAR file or
files in a new directory, and create a \texttt{creole.xml} file in the same
directory to reference the JARs:
\begin{small}
\begin{verbatim}
<CREOLE-DIRECTORY>
  <JAR>my-annotator.jar</JAR>
  <JAR>classes-it-uses.jar</JAR>
</CREOLE-DIRECTORY>
\end{verbatim}
\end{small}
%
This directory should then be loaded in GATE as a CREOLE plugin.  Note that,
due to the complex mechanics of classloaders in Java, putting your JARs in
GATE's \texttt{lib} directory will \emph{not} work.

For annotators written in C++ you need to ensure that the C++ enabler libraries
(available separately from \htlinkplain{http://incubator.apache.org/uima/})
and the shared library containing your annotator are in a directory which is on
the \texttt{PATH} (Windows) or \texttt{LD\_LIBRARY\_PATH} (Linux) when GATE is
run.

%%%%%%%%%%%%%%%%%%%%%%%%%%%%%%%%%%%%%%%%%%%%%%%%%%%%%%%%%%%%%%%%%%%%%%%%%%%%%
\sect[sec:uima:GATEInUIMA]{Embedding a GATE \texttt{CorpusController} in UIMA}
%%%%%%%%%%%%%%%%%%%%%%%%%%%%%%%%%%%%%%%%%%%%%%%%%%%%%%%%%%%%%%%%%%%%%%%%%%%%%

The process of embedding a GATE controller in a UIMA application is more or
less the mirror image of the process detailed in the previous section.  Again,
the developer must supply a mapping descriptor defining how to map between UIMA
and GATE annotations, and pass this, plus the GATE controller definition, to an
AE which performs the translation and calls the GATE controller.

%%%%%%%%%%%%%%%%%%%%%%%%%%%%%%%%%%%%%%%%%%%%%%%%%%%%%%%%%%%%%%%%%%%%%%%%%%%%%
\subsect[sec:uima:GATEInUIMA:mapping]{Mapping File Format}
%%%%%%%%%%%%%%%%%%%%%%%%%%%%%%%%%%%%%%%%%%%%%%%%%%%%%%%%%%%%%%%%%%%%%%%%%%%%%

The mapping descriptor format is virtually identical to that described in
Section \ref{sec:uima:UIMAInGATE:mapping}, except that the input definitions are
\verb|<gateAnnotation>| elements and the output definitions are
\verb|<uimaAnnotation>| elements.  The input and output definition elements
support an extra attribute, \texttt{annotationSetName}, which allows inputs to
be taken from, and outputs to be placed in, different annotation sets.  For
example, the following hypothetical example maps \texttt{com.example.Person}
annotations into the default set and \texttt{com.example.html.Anchor}
annotations to `\texttt{a}' tags in the `Original markups' set.
%
\begin{small}
\begin{verbatim}
<inputs>
  <gateAnnotation type="Person" uimaType="com.example.Person">
    <feature name="kind">
      <uimaFSFeatureValue name="com.example.Person:Kind" kind="string"/>
    </feature>
  </gateAnnotation>

  <gateAnnotation type="a" annotationSetName="Original markups"
                  uimaType="com.example.html.Anchor">
    <feature name="href">
      <uimaFSFeatureValue name="com.example.html.Anchor:hRef" kind="string" />
    </feature>
  </gateAnnotation>
</inputs>
\end{verbatim}
\end{small}

Figure \ref{fig:GATEInUIMA:example} shows a
mapping descriptor for an application that takes tokens and sentences produced
by some UIMA component and runs the GATE part of speech tagger to tag them with
Penn TreeBank POS tags.\footnote{The \texttt{.gapp} file implementing this
example is in the \texttt{test/conf} directory under the \texttt{UIMA} plugin,
along with the mapping file and the AE descriptor that will run it.}  In the
example, no features are copied from the UIMA tokens, but they are still
\texttt{indexed="true"} as the POS feature must be copied back from GATE.
%
\begin{figure}
\begin{small}
\begin{verbatim}
<uimaGateMapping>
  <inputs>
    <gateAnnotation type="Token"
                    uimaType="com.ibm.uima.examples.tokenizer.Token"
                    indexed="true" />
    <gateAnnotation type="Sentence"
                    uimaType="com.ibm.uima.examples.tokenizer.Sentence" />
  </inputs>
  <outputs>
    <updated>
      <uimaAnnotation type="com.ibm.uima.examples.tokenizer.Token"
                      gateType="Token">
        <feature name="POS" kind="string">
          <gateAnnotFeatureValue name="category" />
        </feature>
      </uimaAnnotation>
    </updated>
  </outputs>
</uimaGateMapping>
\end{verbatim}
\end{small}
\caption{An example mapping descriptor for the GATE POS tagger}
\label{fig:GATEInUIMA:example}
\end{figure}

%%%%%%%%%%%%%%%%%%%%%%%%%%%%%%%%%%%%%%%%%%%%%%%%%%%%%%%%%%%%%%%%%%%%%%%%%%%%%
\subsect[sec:uima:GATEInUIMA:gapp]{The GATE Application Definition}
%%%%%%%%%%%%%%%%%%%%%%%%%%%%%%%%%%%%%%%%%%%%%%%%%%%%%%%%%%%%%%%%%%%%%%%%%%%%%

The GATE application to embed is given as a standard `\texttt{.gapp} file',
as produced by saving the state of an application in the GATE GUI.  The
\texttt{.gapp} file encodes the information necessary to load the correct
plugins and create the various CREOLE components that make up the application.
The \texttt{.gapp} file must be fully specified and able to be executed with no
user intervention other than pressing the Go button.  In particular, all
runtime parameters must be set to their correct values before saving the
application state.  Also, since paths to things like CREOLE plugin directories,
resource files, etc. are stored relative to the \texttt{.gapp} file's location,
you must not move the \texttt{.gapp} file to a different directory unless you
can keep all the CREOLE plugins it depends on at the same relative locations.
The `Export for GATE Cloud' option (section~\ref{sec:developer:export}) may
help you here.

%%%%%%%%%%%%%%%%%%%%%%%%%%%%%%%%%%%%%%%%%%%%%%%%%%%%%%%%%%%%%%%%%%%%%%%%%%%%%
\subsect{Configuring the \texttt{GATEApplicationAnnotator}}
%%%%%%%%%%%%%%%%%%%%%%%%%%%%%%%%%%%%%%%%%%%%%%%%%%%%%%%%%%%%%%%%%%%%%%%%%%%%%

\texttt{GATEApplicationAnnotator} is the UIMA annotator that handles mapping
the CAS into a GATE document and back again and calling the GATE controller.
There is a template AE descriptor XML file for the annotator provided in the
\texttt{conf} directory.  Most of the template file can be used unchanged, but
you will need to modify the type system definition and input/output
capabilities to match the types and features used in your mapping descriptor.
If the mapping descriptor references a type or feature that is not defined in
the type system, a runtime error will occur.

The annotator requires two external resources:
\begin{description}
\item[GateApplication] The \texttt{.gapp} file containing the saved application
state.
\item[MappingDescriptor] The mapping descriptor XML file.
\end{description}
%
These must be bound to suitable URLs, either by editing the
\texttt{resourceManagerConfiguration} section of the primitive descriptor, or by
supplying the binding in an aggregate descriptor that includes the
\texttt{GATEApplicationAnnotator} as one of its delegates.

In addition, you may need to set the following Java system properties:
\begin{description}
\item[uima.gate.configdir] The path to the GATE config directory.  This
defaults to \texttt{gate-config} in the same directory as
\texttt{uima-gate.jar}.
\item[uima.gate.siteconfig] The location of the sitewide \texttt{gate.xml}
configuration file.  This defaults to
\texttt{\textit{gate.uima.configdir}/site-gate.xml}.
\item[uima.gate.userconfig] The location of the user-specific \texttt{gate.xml}
configuration file.  This defaults to
\texttt{\textit{gate.uima.configdir}/user-gate.xml}.
\end{description}

The default config files are deliberately simplified from the standard versions
supplied with GATE, in particular they do not load any plugins automatically
(not even ANNIE).  All the plugins used by your application are specified in
the \texttt{.gapp} file, and will be loaded when the application is loaded, so
it is best to avoid loading any others from \texttt{gate.xml}, to avoid
problems such as two different versions of the same plugin being loaded from
different locations.

\subsubsect{Classpath Notes}
In addition to the usual UIMA library JAR files,
\texttt{GATEApplicationAnnotator} requires a number of JAR files from the GATE
distribution in order to function.  In the first instance, you should include
\texttt{gate.jar} from GATE's \texttt{bin} directory, and also all the JAR
files from GATE's \texttt{lib} directory on the classpath.  If you use the
supplied Ant build file, \texttt{ant documentanalyser} will run the document
analyser with this classpath.  Depending on exactly which GATE plugins your
application uses, you may be able to exclude some of the \texttt{lib} JAR
files (for example, you will not need Weka if you do not use the machine
learning plugin), but it is safest to start with them all.  GATE will load
plugin JAR files through its own classloader, so these do not need to be on the
classpath.
