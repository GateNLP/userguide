%%%%%%%%%%%%%%%%%%%%%%%%%%%%%%%%%%%%%%%%%%%%%%%%%%%%%%%%%%%%%%%%%%%%%%%%%%%%%
%
% changes.tex
%
% diana, Oct 2002
%
% $Id: changes.tex,v 1.78 2006/10/23 12:56:36 ian Exp $
%
%%%%%%%%%%%%%%%%%%%%%%%%%%%%%%%%%%%%%%%%%%%%%%%%%%%%%%%%%%%%%%%%%%%%%%%%%%%%%

%%%%%%%%%%%%%%%%%%%%%%%%%%%%%%%%%%%%%%%%%%%%%%%%%%%%%%%%%%%%%%%%%%%%%%%%%%%%%
\chapt[chap:changes]{Change Log}
\markboth{Change Log}{Change Log}
%%%%%%%%%%%%%%%%%%%%%%%%%%%%%%%%%%%%%%%%%%%%%%%%%%%%%%%%%%%%%%%%%%%%%%%%%%%%%

This \chapthing\ lists major changes to GATE (currently Developer and Embedded
only) in roughly chronological order by
release. Changes in the documentation are also referenced here.

%NEW CHANGES SHOULD BE ADDED IN `recent-changes.tex' NOT HERE!!!
%'nested' switch is used to make section headings format correctly whether
%they are in the appendix or the introduction.
%
% Within this file you should use \rcSect[label]{Heading} instead of
% \sect[sec:changes:label]{Heading}, and \rcSubsect{Heading} instead of
% \subsect{Heading}.  These commands are redefined appropriately depending on
% whether we are processing the intro subsection or the full changelog
% appendix.
%
\ifnested
  \def\rcSectNoLabel#1{\subsect{#1}}
  \def\rcSect[#1]#2{\subsect[subsec:changes:#1]{#2}}
  \def\rcSubsect#1{\subsubsect{#1}}
  \def\rcSubsubsect#1{\subsubsubsect{#1}}
\else
  \def\rcSectNoLabel#1{\sect{#1}}
  \def\rcSect[#1]#2{\sect[sec:changes:#1]{#2}}
  \def\rcSubsect#1{\subsect{#1}}
  \def\rcSubsubsect#1{\subsubsect{#1}}
\fi

It was brought to our attention that in versions 9.0.1 and below there was a very small
chance that the GUI action ``Export for GATE Cloud'' could be compromised. This would have
required malicious code to be running locally on the machine; either by another user
on a multi-user machine or because the computer had already been compromised. This issue
only occurred within the GUI action and did not affect API use of the \verb!gate-core!
Maven artifact. Note that no known exploits exist for this issue, and we do not know for
certain that the code could be exploited. If, however, you are at all concerned then we
suggest you regenerate any packaged applications using a recent version of GATE Developer;
at minimum 9.2-SNAPSHOT built on or after the 10th of August 2022.

\rcSect[9.0.1]{Version 9.0.1 (March 2021)}
GATE Developer 9.0.1 is a bugfix release -- the only change is to the way URL redirects
are handled when loading a document. Support for following redirects from http to https
was added in 9.0 which, while correct, broke the way URLs were used within GCP. This
release fixes that bug and adds some additional security checking to the redirect handling.

\rcSect[9.0]{Version 9.0 (February 2021)}

Whilst the majority of changes in GATE Developer 9.0 are small a number of them
change default behaviour (in the UI or API) hence the change in version number.
These changes include:

\begin{itemize}
\item We now recommend users install a 64 bit version of Java whenever possible. This
      seems to be especially important on Windows.
\item We now default to assuming documents are UTF-8 encoded unless you specify otherwise.
      In previous versions if no encoding was specified GATE would use the default platform
      encoding, but this seemed to cause more problems than it solved (especially for Windows
      users). If you want the old behaviour then ensure the encoding parameter is set to the
      empty string when creating a document.
\item GATE uses a library called XStream for saving and loading GATE XML documents and
      applications. This allows us to store features of any Java type, but that can be
      abused by maliciously crafted files. In general use this is unlikely to be a problem,
      but in situations where GATE may be used as part of a service with no way of vetting
      input files it could present a serious security threat. XStream now offers a security
      framework to restrict the types of objects that can be loaded/saved. This can work either
      by allowing only specific types or by preventing specific types from being used. As we
      often do not know in advance what features might be used we have opted to use a minimal
      blacklist as the default security setting. This blocks the Java classes known to be
      exploitable. This can be further configured via calls to \verb!Gate.setXStreamSecurity()!
      and we strongly encourage developers who depend on gate-core within larger applications
      to configure this based on their specific use cases.
\item Developers wishing to build GATE from source need to use Maven v3.6.0 or above.
\item Previous versions of GATE used Log4J for some of the logging. This was problematic
      when using gate-core as a dependency in larger projects and was awkward to configure
      properly. In this release we've switched to using SLF4J allowing the actual logging
      back-end to be configured independently. Plugins and code compiled against previous
      versions of GATE should work with the new release without change (we include the
      log4j-over-slf4j bridge as a dependency), although Log4J specific methods within
      gate-core have been deprecated and may be removed in a future release.
\end{itemize}

Many bugs have been fixed and documentation improved, in particular:

\begin{itemize}
\item the \verb!Twitter! plugin has been improved to make better use of the information
      provided by Twitter within a JSON Tweet object. The Hashtag tokenizer has been
      updated to provide a \verb!tokenized! feature to make grouping semantically similar
      hashtags easier. Lots of other minor improvements and efficiency changes have been
      made throughout the rest of the TwitIE pipelines.
\item the \verb!ANNIE! gazetteers have been updated to better support different ways of referring
      to countries and a blacklist option to prevent things being wrongly annotated.
\item A new addition to the JAPE syntax allows you to copy all features from a matched annotation
      to the new annotation being created
\item the \verb!Format_CSV! plugin now allows the document cell to be interpreted as being
      a URL pointing to the document to load rather than the contents of the document. See
      Section~\ref{sec:creole:csv} for more details.
\end{itemize}

\rcSect[8.6.1]{Version 8.6.1 (January 2020)}

GATE Developer 8.6.1 is a bugfix release -- the only change is to adjust for
the fact that the Central Maven repository has been switched from \verb!http!
to \verb!https!.

\rcSect[8.6]{Version 8.6 (June 2019)}

GATE Developer 8.6 is mainly a maintenance and stability release, but there are
some important new features, in particular around the processing of
Twitter data:

\begin{itemize}
\item The \verb!Format_Twitter! plugin can now correctly handle extended 280
  character tweets and the latest Twitter JSON format. See
  Section~\ref{sec:social:twitter:format} for full details.
\item The new \verb!Format_JSON! plugin provides import/export support for GATE
  JSON.  This is essentially the old style Twitter format, but it no longer
  needs to track changes to the Twitter JSON format so should be more suitable
  for long term storage of GATE documents as JSON files. See
  Section~\ref{sec:creole:gatejson} for more details.  This plugin makes use of
  a new mechanism whereby document format parsers can take parameters via the
  document MIME type, which may be useful to third party formats too.
\end{itemize}

Many bugs have been fixed and documentation improved, in particular:

\begin{itemize}
\item The plugin loading mechanism now properly respects the user's Maven
  \verb!settings.xml!:
  \begin{itemize}
  \item HTTP proxy and ``mirror'' repository settings now work properly,
  including authentication.  Also plugin resolution will now use the system
  proxy (if there is one) by default if there is no proxy specified in the
  Maven settings.
  \item The ``offline'' setting is respected, and will prevent GATE from trying
  to fetch plugins from remote repositories altogether -- for this to work, all
  the plugins you want to use must already be cached locally, or you can use
  ``Export for GATE Cloud'' to make a self-contained copy of an application
  including all its plugins.
  \end{itemize}
\item Upgraded many dependencies including Tika and Jackson to avoid known
  security bugs in the previous versions.
\item Documentation improvements for the Kea plugin, the Corpus QA and
  annotation diff tools, and the default GATE XML and inline XML formats
  (section~\ref{sec:developer:dump})
\item For plugin developers, the standard plugin testing framework generates a
  report detailing all the plugin-to-plugin dependencies, including those that
  are only expressed in the plugin's example saved applications
  (section~\ref{sec:api:bootstrap:dependencies}).
\end{itemize}

Some obsolete plugins have been removed (Websphinx web crawler, which depends
on an unmaintained library, and the RASP parser, whose external binary is no
longer available for modern operating systems), and there are many smaller bug
fixes and improvements.

Note: following changes to Oracle's JDK licensing scheme, we now recommend
running GATE using the freely-available OpenJDK.  The
\htlink{https://adoptopenjdk.net}{AdoptOpenJDK} project offers simple
installers for all major platforms, and major Linux distributions such as
Ubuntu and CentOS offer OpenJDK packages as standard.  See
section~\ref{sec:gettingstarted:install} for full installation instructions.

\rcSect[8.5.1]{Version 8.5.1 (June 2018)}

Version 8.5.1 is a minor release to fix a few critical bugs in 8.5:

\begin{itemize}
\item Fixed an exception that prevented the ANNIC search GUI from opening.
\item Fixed a problem with ``Export for GATE Cloud'' that meant some resources
  were not getting included in the output ZIP file.
\item Fixed the XML schema in the \verb!gate-spring! library.
\end{itemize}

\rcSect[8.5]{Version 8.5 (May 2018)}

GATE Developer and Embedded 8.5 introduces a number of significant internal
changes to the way plugins are managed, but with the exception of the plugin
manager most users will not see significant changes in the way they use GATE.

\begin{itemize}
\item The GATE plugins are no longer bundled with the GATE Developer
  distribution, instead each plugin is downloaded from a repository at runtime,
  the first time it is used.  This means the distribution is much smaller than
  previous versions.
\item Most plugins are now distributed as a single JAR file through the
  Java-standard ``Central Repository'', and resource files such as gazetteers
  and JAPE grammars are bundled inside the plugin JAR rather than being
  separate files on disk.  If you want to modify the resources of a plugin
  then GATE provides a tool to extract an editable copy of the files from
  a plugin onto your disk -- it is no longer possible to edit plugin grammars
  in place.
\item This makes dependencies between plugins much easier to manage -- a plugin
  can specify its dependencies declaratively by name and version number rather
  than by fragile relative paths between plugin directories.
\end{itemize}

GATE 8.5 remains backwards compatible with existing third-party plugins, though
we encourage you to convert your plugins to the new style where possible.

Further details on these changes can be found in sections
\ref{sec:developer:plugins} (the plugin manager in GATE Developer),
\ref{sec:api:plugins} (loading plugins via the GATE Embedded API),
\ref{sec:api:bootstrap} (creating a new plugin from scratch), and
\ref{sec:api:updateplugins} (converting an existing plugin to the new style).

If you have an existing saved application from GATE version 8.4.1 or earlier it
will be necessary to ``upgrade'' it to use the new core plugins.  An upgrade
tool is provided on the ``Tools'' menu of GATE Developer, and is described in
section Section~\ref{sec:developer:convertxgapp}.

\rcSubsect{For developers}

As part of this release, GATE development has moved from SourceForge to GitHub
-- bug reports, patches and feature requests should now use the GitHub issue
tracker as described in section~\ref{sec:development:bugs}.

% vim:ft=tex


\rcSect[8.4.1]{Version 8.4.1 (June 2017)}

This is a minor release that fixes one rarely encountered but serious bug with
the handling of CDATA sections within the text content of GATE XML format
documents.  CDATA has always been handled correctly in annotation and document
feature values, this bug would only affect a small number of documents where
the text contains many less-than signs (\verb!<<<!) \emph{and} few annotations.
In particular, \emph{annotated} documents that have been processed using the
GATE tokeniser are extremely unlikely to be affected as each less-than sign is
treated as a separate \texttt{Token} annotation.

This release also includes one small improvement to the Twitter hashtag
tokeniser so it recognises the names of some political parties when they occur
within hashtags such as \verb!#VoteLabour!.

\rcSect[8.4]{Version 8.4 (February 2017)}

GATE Developer and Embedded 8.4 is mainly a bug fix release, with a small
number of critical fixes compared to version 8.3.  This will be the final
major release of GATE before major re-structuring of the codebase and the plugin
system for version 8.5.

\begin{itemize}
\item Fixed an issue which had prevented the use of Java 8 lambda expressions
  in the RHS of JAPE rules, even when running on Java 8.
\item Removed OpenCalais and Zemanta plugins as the web services they depend on
  have changed and the plugins no longer work.
\item Fixed a bug that could cause the searchable datastore GUI to freeze.
\item Fixes to the TermRaider and Hindi sample applications
\end{itemize}

\rcSubsect{Java compatibility}

For GATE 8.4 we recommend the use of the latest Java 8 from Oracle.  If you are
still restricted to Java 7, most components will still work with the exception
of the Stanford CoreNLP tools and the TwitIE application (which uses the
Stanford POS tagger).  Future versions of GATE will require Java 8 as a
minimum.

\rcSect[8.3]{Version 8.3 (January 2017)}

GATE Developer and Embedded 8.3 is mainly a bug fix release, with several
critical fixes and functionality improvements.

\begin{itemize}
\item JAPE grammars can now match and create annotation types and features
  with spaces or punctuation in their names, by using double quotes around
  the type or feature name (e.g. \verb!{"w:p"}!).
\item Fixed a regression in 8.2 that meant saved application states and
  ``export for GATE Cloud'' packages created on Windows would not load on other
  platforms.
\item Fixed a bug in the Stanford CoreNLP plugin which would sometimes fail
  when GATE is installed in a directory whose path contains spaces.
\item Various improvements to the Twitter normaliser and emoticon finder.
\item Improvements to the \verb!Lang_French! and \verb!Lang_German! components.
  Further improvements will follow in the next release.
\item Improvement to the \verb!Crowd_Sourcing! plugin to allow a default
  option to be specified for clasification jobs.
\item Fixed Java version detection in the Windows EXE launcher.
\item Detection of document format using clues from the content is now much
  more efficient.
\item Fixed some GUI deadlocks in the searchable data store GUI and the plugin
  manager.
\item Fixed a long-standing bug in the regex sentence splitter for documents
  with long sequences of blank lines.
\item Removed Minipar parser plugin as the data files on which it depends are
  no longer available for download, and the \verb!Tagger_NormaGene! plugin as
  the service it relies on is no longer online.
\end{itemize}

Plus the usual suite of miscellaneous smaller bug fixes.

\rcSubsect{Java compatibility}

For GATE 8.3 we recommend the use of the latest Java 8 from Oracle.  If you are
still restricted to Java 7, most components will still work with the exception
of the Stanford CoreNLP tools and the TwitIE application (which uses the
Stanford POS tagger).  Future versions of GATE will require Java 8 as a
minimum.

\rcSect[8.2]{Version 8.2 (May 2016)}

GATE Developer and Embedded 8.2 is mainly a bug fix release -- there are a few
new plugins but the emphasis is on bug fixing and library updates.

\begin{itemize}
\item New tools for temporal expression and event detection, including a wrapper
  for the HeidelTime tagger (section~\ref{sec:misc-creole:gate-time}).
\item New language plugins for Danish and Welsh named entitiy recognition.
\item Performance improvements in the ANNIE NER system, in particular to deal
  better with hyphenated names and titles.
\item Improvements to TermRaider to support GATE documents that contain many
  independent sections (e.g. web forums, lists of tweets).
\item Bug fixes in the handling of Twitter JSON data -- GATE now has full
  round-trip support for Twitter JSON, tweets can be loaded, annotated, and
  saved back to the same format accurately.  The JSON format parser has been
  separated from the rest of the Twitter plugin, making it easier to add JSON
  support to non-TwitIE applications.
\item Updated dependencies -- the \verb!Stanford_CoreNLP! plugin now uses
  version 3.6.0 of Stanford CoreNLP, and the Groovy plugin uses Groovy version
  2.4.4
\item GCP input and output handlers added to the \verb!Format_CSV! plugin.
\end{itemize}

Plus the usual suite of miscellaneous smaller bug fixes.

\rcSubsect{Java compatibility}

For GATE 8.2 we recommend the use of the latest Java 8 from Oracle.  If you are
still restricted to Java 7, most components will still work with the exception
of the Stanford CoreNLP tools.  Future versions of GATE will require Java 8 as
a minimum.


\rcSect[8.1]{Version 8.1 (June 2015)}

\rcSubsect{New plugins and significant new features}

\begin{itemize}
\item Integration of the Stanford NER tools -- all the Stanford tools in GATE
  have been brought together under a single \texttt{Stanford\_CoreNLP} plugin.
\item Improved crowdsourcing tools (chapter~\ref{chap:crowd}), including tools
  to perform automatic adjudication of multiply-annotated documents.
\item Parsers for new document formats, including the \emph{DataSift} format
  (section~\ref{sec:creole:datasift}) for social media data, and an improved
  Twitter JSON parser (section~\ref{sec:social:twitter:format}) which can import
  the standoff annotations Twitter themselves provide (hashtags, etc.).
\item Improved support for data \emph{export}, making it easy for plugins to
  add their own export formats accessible through the GUI and the API. New
  exporters are provided for the Twitter JSON format
  (section~\ref{sec:social:twitter:export}) and a more configurable inline
  XML format.
\item A new plugin for simplifying sentences using linguistic rules and other
  information (section~\ref{sec:misc-creole:linguistic-simplifier}), contributed
  by the \htlink{https://www.forgetit-project.eu}{ForgetIT project}.
\end{itemize}

\rcSubsect{Library updates and bugfixes}

\begin{itemize}
\item Apache Tika (for parsing PDF, MS Word, etc.) updated to version 1.7.
\item ASM (for processing CREOLE metadata) updated to version 5.0.3.  This
  allows the use of Java 8 language features such as lambdas in third-party
  plugins, though GATE itself remains compatible with Java 7.
\item Stanford CoreNLP tools updated to version 3.4 (the latest version
  that is compatible with Java 7).
\item MetaMap libraries (for UMLS) updated to version 2014.
\item Better support in the GATE Unicode Tokeniser for scripts that use
  supplementary characters beyond the basic 16 bit range.
\item Bugfixes in the segment processing PR.
\end{itemize}

\rcSubsect{Tools for developers}

Three new tools have been added to the \verb!Developer_Tools! plugin:

\begin{itemize}
\item Menu options to produce Java heap dumps to aid debugging.
\item Menu option to dynamically increase the Log4J logging verbosity
  at runtime.
\item A tool that attempts to unload all plugins that are loaded but not
  currently in use.
\end{itemize}

Other changes that benefit developers include:
\begin{itemize}
\item New helper methods in the \verb!gate.Utils! class.
\item The \verb!gate.util.ProcessManager! API has been extended to allow an
  external process to be kept running.  This is especially useful for running
  some external tools which have very long starup times yet can be reused
  across documents.
\item Some changes in the management of classloaders that should reduce the
  potential for deadlocks.
\end{itemize}

\ldots and as always, a range of smaller improvements and bug fixes.

\rcSect[8.0]{Version 8.0 (May 2014)}

GATE 8.0 is a major release which brings some major new features, many new and
updated plugins, and significant under-the-bonnet changes to GATE Embedded.

\rcSubsect{Major changes}

\textbf{Java 7 required}

GATE 8.0 requires \textbf{Java 7} or later to run.

\textbf{Tools for Twitter}

A new ``Twitter'' plugin provides tools dedicated to Twitter data:
\begin{itemize}
\item format parsers to handle Tweets in the JSON formats produced by the
  Twitter APIs
\item Twitter-specific components such as a tokeniser and POS tagger
\item the \emph{TwitIE} named entity annotation pipeline.
\end{itemize}

See section~\ref{sec:social:twitter} for full details.

\textbf{ANNIE Refreshed}

The ANNIE named entity annotation pipeline which has been the mainstay of many
GATE applications for many years has been brought up to date, with new
gazetteers and improved JAPE grammars giving improved precision and recall on
common test corpora.

\textbf{Tools for Crowd Sourcing}

A new \verb!Crowd_Sourcing! plugin provides facilities to support generation of
manually annotated corpora via the CrowdFlower crowdsourcing
platform\footnote{\url{http://www.crowdflower.com}}.  The plugin provides
support for two different kinds of tasks, general entity annotation (e.g.
determining which words in a given sentence are person names) and entity
linking (e.g. for ontology-based annotation, where the spans of the entities
are known but not which particular ontology instance each annotation
corresponds to).  Using crowdsourcing you can generate multiply-annotated gold
standard corpora rapidly and at relatively low cost.  For full details see
chapter~\ref{chap:crowd}.

\rcSubsect{Other new and improved plugins}

\begin{itemize}
\item New language plugins to support \emph{Russian} and \emph{Bulgarian}
\item Integration of the Stanford POS Tagger
  (section~\ref{sec:misc:creole:stanford}), which is used by TwitIE
\item A document normalizer plugin, predominantly to normalize punctuation such
  as Microsoft Word ``ssec:creole:datasiftmart quotes'' (see
  section~\ref{sec:misc-creole:doc-normalizer})
\item Wrappers for the \emph{AlchemyAPI} keyword and entity extraction services
  (in the \verb!AlchemyAPI! plugin)
\item Wrapper for the \emph{TextRazor} annotation service (see
  section~\ref{sec:misc-creole:textrazor}).
\item New document format parser to populate a GATE corpus from one or more CSV
  files (see section~\ref{sec:creole:csv}).
\item Support for loading and saving GATE XML files in the binary
  \emph{FastInfoset} format (see section~\ref{sec:creole:fastinfoset}).
\item Various improvements to the \verb!Learning! plugin, in particular to
  support numeric and boolean features (see
  section~\ref{sec:ml:batch-learning-pr})
\item Improvements to the \emph{TermRaider} term extraction plugin (see
  section~\ref{sec:creole:termraider})
\item The OntoRoot Gazetteer (in the \verb!Gazetteer_Ontology_Based! plugin)
  now supports tokenisers and POS taggers other than the default ANNIE PR
  types, making it possible to use other preprocessing tools for non-English
  data
\item Further improvements to the classloading model to better isolate plugins
  from one another.
\item A new \verb!enableDebugging! runtime parameter for JAPE grammars will add
  additional features to every generated annotation detailing which rule was
  responsible for creating the annotation.
\end{itemize}

\rcSubsect{Bug fixes and other improvements}

\begin{itemize}
\item The annotation schema LR type is now available by default without the
  need to load any plugins.  The schemas that were previously loaded by default
  by the ANNIE plugin must now be loaded explicitly if you require them
  (section~\ref{sec:developer:schemaannotationeditor}).  Annotation schemas now
  support the \verb!include! element, so multiple schemas can be loaded by
  loading a single master file.
\item The segment processing PR
  (section~\ref{sec:alignment:segment-processing}) now preserves annotation
  IDs, allowing ID-sensitive tools such as coreference to work properly.
\end{itemize}

\rcSubsect{For developers}

Changes of note for users of the GATE Embedded APIs include:
\begin{itemize}
\item A new data model to represent relations between annotations, see
  section~\ref{sec:api:relations} for details.  The \verb!Coref_Tools! plugin
  has been retrofitted to use this new model to represent coreference
  chains.  Relation information is preserved when saving documents as GATE XML,
  but note that such documents will not be compatible with older versions of
  GATE.
\item A new ``resource helper'' mechanism allows plugins to contribute
  additional actions to existing resource types, both in the Developer GUI
  (section~\ref{sec:creole-model:tools:resourcehelpers}) and in the Embedded API
  (section~\ref{sec:api:resourcehelpers})
\item A new class \verb!gate.corpora.DocumentJsonUtils! provides methods to
  export a GATE document in a JSON format compatible with that used by Twitter.
  See the JavaDoc documentation for details.
\item Many deprecated classes, fields and methods have been removed.  If you
  were previously calling any of these deprecated APIs you will need to update
  your code accordingly.  Also some classes in the GATE core that were only
  used by one plugin have been moved into the respective plugin's source tree.
  In particular, Java RHS actions in JAPE rules no longer provide the
  long-deprecated \verb!annotations! variable -- use \verb!inputAS! or
  \verb!outputAS! as appropriate.
\item Many library dependencies have been updated to more recent versions.
\item The GATE APIs make much wider use of generics than previously -- many
  places in the code that previously used raw types are now properly generic
\item A new \verb!Developer_Tools! plugin
  (section~\ref{sec:misc-creole:dev-tools}) provides utilities to assist in
  debugging applications in GATE Developer.
\end{itemize}

If you are working on the core GATE source code, note that:
\begin{itemize}
\item the source tree has been split into ``main'' and ``test'', isolating the
  test classes from the rest of the source
\item each plugin is now a separate Eclipse ``project'', and the main project
  is just the core sources, which makes it easier to control dependencies among
  the different parts
\item dependencies are no longer checked in to subversion, instead they are
  fetched at build time from the Maven central repository by Apache Ivy.
\end{itemize}

\rcSect[7.1]{Version 7.1 (November 2012)}

\rcSubsect{New plugins}

The \emph{TermRaider} plugin (see Section~\ref{sec:creole:termraider})
provides a toolkit and sample application for term extraction.

Two new plugins, \emph{Tagger\_Zemanta} (since removed) and
\emph{Tagger\_Lupedia} provide PRs that wrap online annotation services provided
by Zemanta and Ontotext.

A new plugin named \emph{Coref\_Tools} includes a framework for fast
co-reference processing, and one PR that performs orthographical co-reference in
the style of the ANNIE Orthomatcher. See Section~\ref{sec:creole:coref-tools}
for full details.

A new \emph{Configurable Exporter} PR in the Tools plugin, allowing annotations
and features to be exported in formats specified by the user (e.g. for use with
external machine learning tools).  See Section~\ref{sec:misc-creole:confexport}
for details.

Support for reading a number of new document formats has also been added:
\begin{itemize}
\item \emph{PubMed and the Cochrane Library} formats (see
  Section~\ref{sec:creole:pubmed}).
\item \emph{CoNLL ``IOB''} format (see Section~\ref{sec:corpora:conll}).
\item\emph{MediaWiki} markup, both plain text and XML dump files such as those
  from Wikipedia (see Section~\ref{sec:creole:mediawiki}).
\end{itemize}

In addition, ``ready-made applications'' have been added to many existing
plugins (notably the \emph{Lang\_*} non-English language plugins) to make it
easier to experiment with their PRs.

\rcSubsect{Library updates}

Updated the Stanford Parser plugin (see Section~\ref{sec:parsers:stanford}) to
version 2.0.4 of the parser itself, and added run-time parameters to the PR to
control the parser's dependency options.

The Measurement and Number taggers have been upgraded to use JAPE+ instead of
JAPE.  This should result in faster processing, and also allows for more memory
efficient duplication of PR instances, i.e. when a pool of applications is
created.

The OpenNLP plugin has been completely revised to use Apache OpenNLP 1.5.2 and
the corresponding set of models.  See Section~\ref{sec:misc-creole:opennlp} for
details.

The native launcher for GATE on Mac OS X now works with Oracle Java 7 as well
as Apple Java 6.

\rcSubsect{GATE Embedded API changes}

Some of the most significant changes in this version are ``under the bonnet''
in GATE Embedded:
\begin{itemize}
\item The class loading architecture underlying the loading of plugins and the
  generation of code from JAPE grammars has been re-worked. The new version
  allows for the complete unloading of plugins and for better memory handling of
  generated classes. Different plugins can now also use different versions of the
  same 3rd party libraries. There have also been a number of changes to the way
  plugins are (un)loaded which should provide for more consistent behaviour.
\item The GATE XML format has been updated to handle more value types
  (essentially every data type supported by XStream
  (\url{http://xstream.codehaus.org/faq.html}) should be usable as feature name
  or value. Files in the new format can be opened without error by older GATE
  versions, but the data for the previously-unsupported types will be
  interpreted as a String, containing an XML fragment.
\item The PRs defined in the ANNIE plugin are now described by annotations on
  the Java classes rather than explicitly inside creole.xml. The main reason for
  this change is to enable the definitions to be inherited to any subclasses of
  these PRs. Creating an empty subclass is a common way of providing a PR with a
  different set of default parameters (this is used extensively in the language
  plugins to provide custom gazetteers and named entity transducers). This has
  the added benefit of ensuring that new features also automatically percolate
  down to these subclasses.  If you have developed your own PR that extends one
  of the ANNIE ones you may find it has acquired new parameters that were not
  there previously, you may need to use the \verb!@HiddenCreoleParameter!
  annotation to suppress them.
\item The corpus parameter of LanguageAnalyser (an interface most, if not all,
  PRs implement) is now annotated as \verb|@Optional| as most implementations
  do not actually require the parameter to be set.
\item When saving an application the plugins are now saved in the same order in
  which they were originally loaded into GATE. This ensures that dependencies
  between plugins are correctly maintained when applications are restored.
\item API support for working with relations between annotations was added. See
  Section~\ref{sec:api:relations} for more details.
\item The method of populating a corpus from a single file has been updated to
  allow any mime type to be used when creating the new documents.
\end{itemize}

And numerous smaller bug fixes and performance improvements\ldots


\rcSect[7.0]{Version 7.0 (February 2012)}

\rcSubsect{Major new features}

The CREOLE Plugin Manager has been completely re-written and now includes
support for installing new plugins from remote update sites. See
section~\ref{sec:developer:installplugins} for more details.  In addition,
plugins can now contribute additional ``ready-made applications'' to the GATE
Developer menus alongside the standard applications (ANNIE, etc.).  Details can
be found in section~\ref{sec:development:readymade}.

A new plugin named \texttt{JAPE\_Plus} has been added. It contains a new JAPE
execution engine that includes various optimisations and should be significantly
faster than the standard engine. \texttt{JAPE\_Plus} has not yet been
comprehensively tested, so it should be considered {\em beta} software, and used
with caution. See Section~\ref{sec:jape:plus} for more details.

A new Java-based launcher has been implemented which now replaces the use of
Apache ANT for starting-up GATE Developer.  The GATE Developer application now
behaves in a more natural way in dock-based desktop environments such as Mac OS
X and Ubuntu Unity.

Improved the support for processing biomedical text by adding new PRs to
incorporate the following tools: AbGene, the NormaGene tagger, the GENIA
sentence splitter, MutationFinder and the Penn BioTagger (contains a tokenizer
and three taggers for gene, malignancy and variation). For full details of
these new resources see section~\ref{sec:domain-creole:biomed}.

The Flexible Gazetteer PR has been rewritten to provide a better and faster
implementation.  The two parameters \texttt{inputAnnotationSetName} and
\texttt{outputAnnotationSetName} have been renamed to \texttt{inputASName} and
\texttt{outputASName}, however old applications with the old parameters
should still work.  Please see Section \ref{sec:gazetteers:flexgazetteer} for
more details.

\rcSubsect{Removal of deprecated functionality}

Various components were removed in this release as they have been unsupported
and deprecated in previous releases:
\begin{itemize}
\item the GATE Unicode Kit (GUK), which has been superseded by improved native
  support for localisation in the various target operating systems.  If you
  still require GUK it is available as a separate software project at
  \url{http://gate.svn.sourceforge.net/viewvc/gate/guk/trunk}.
\item the database-backed datastore implementation.
\item the plugins Jape\_Compiler (superseded by JAPE\_Plus) and
  Ontology\_OWLIM2.
\end{itemize}

In addition the Web\_Search\_Google, Web\_Search\_Yahoo and
Web\_Translate\_Google plugins have been removed as the underlying web services
on which they depend are no longer available.  Documentation for obsolete
plugins can be found in appendix~\ref{chap:obsolete-plugins}, and if you
require any of them for your application please see
\verb|plugins/Obsolete/README.TXT| in the GATE Developer distribution.

\rcSubsect{Other enhancements and bug fixes}

CREOLE plugins can now use \htlink{http://ant.apache.org/ivy/}{Apache Ivy} to
include third-party dependencies. See section~\ref{sec:creole-model:config} for details.

The Default ANNIE Gazetteer now allows a user to specify different annotation
types to be used for annotating entries from different lists. For example, a
user may want to find city names mentioned in a gazetteer list (e.g. city.lst)
and annotate the matching strings as City. Please see
section~\ref{sec:annie:gazetteer} for more details.

The Segment Processing PR has two additional run-time parameters called
\texttt{segmentAnnotationFeatureName} and
\texttt{segmentAnnotationFeatureValue}.  These features allow users to specify
a constraint on feature name and feature value.  If user has provided values
for these parameters, only the annotations with the specified feature name and
feature value are processed with the Segment Processing PR.  Also, the
parameter \texttt{controller} has been renamed to \texttt{analyser} which means
the Segment Processing PR can now also run an individual PR on the specified
segments\footnote{Existing saved applications using the \texttt{controller}
parameter will still work \emph{provided} the controller in question implements
the \texttt{LanguageAnalyser} interface.  The \texttt{CorpusController}
implementations supplied as standard with GATE all implement this interface.}.
See \ref{sec:alignment:segment-processing} for more information on
section-by-section processing.

The Hash Gazetteer (section~\ref{sec:gazetteers:hash}) now properly supports
the \texttt{caseSensitive} parameter (previously the parameter could be set but
had no effect).

The Document Reset PR (Section~\ref{sec:misc-creole:reset}) now defaults to
keeping the Key set as well as Original markups. This makes working with
pre-annotated gold standard document less dangerous (assuming you put the gold
standard annotations in a set called Key).

Updated Stanford Parser plugin (see Section~\ref{sec:parsers:stanford}) to
version~1.6.8.

The TextCat based Language Identification PR now supports generating new
language fingerprints.  See
section~\ref{sec:misc-creole:language-identification} for full details.

Added support for reading XCAS and XMI-format documents created by UIMA. See
section~\ref{sec:corpora:uima} for details.

Various improvements to the GATE Developer GUI:
\begin{itemize}
\item added support in the document editor to switch the principal text
  orientation, to better support documents written in right-to-left languages
  such as Arabic, Hebrew or Urdu (section~\ref{sec:developer:documents}).
\item added new mouse shortcuts to the Annotation Stack view in the document
  editor to speed up the curation process
  (section~\ref{sec:developer:annotationsstackview}).
\item the document editor layout is now saved to the user preferences file,
  gate.xml. It means that you can give this file to a new user so s/he will
  have a preconfigured document editor
  (section~\ref{sec:developer:documents}).
\item the script behind an instance of the Groovy Scripting PR
  (section~\ref{sec:api:groovy:pr}) can now be edited from within GATE
  Developer through a new visual resource which supports syntax highlighting.
\end{itemize}

The rule and phase names are now accessible in a JAPE Java RHS by the 
\verb=ruleName()= and \verb=phaseName()= methods and the name of the 
JAPE processing resource executing the JAPE transducer is accessible
through the action context \verb=getPRName()= method. 
See section~\ref{sec:jape:javarhsoverview}.


%%%%%%%%%%%%%%%%%%%%%%%%%%%%%%%%%%%%%%%%
\rcSect[6.1]{Version 6.1 (April 2011)}

\rcSubsect{New CREOLE Plugins}

\textbf{Tagger\_Numbers} to annotate many kinds of numbers in documents and
determine their numeric values.  The tagger can annotate numbers expressed in
many forms including Arabic and Roman numerals, words (in English, French,
German and Spanish) and scientific notation (4.3e6 = 4300000).  See
section~\ref{sec:misc-creole:numbers} for full details.

\textbf{Tagger\_Measurements} to annotate many different forms of measurement
expressions (``5.5 metres'', ``1 minute 30 seconds'', ``10 to 15 pounds'',
etc.) along with their normalized values in SI units.  See
section~\ref{sec:misc-creole:measurements} for full details.

\textbf{Tagger\_Boilerpipe}, which contains a
boilerpipe\footnote{\htlinkplain{http://code.google.com/p/boilerpipe/}} based
PR for performing content detection. See
section~\ref{sec:misc-creole:boilerpipe} for full details.

\textbf{Tagger\_DateNormalizer} to annotate and normalize dates within a
document. See section~\ref{sec:misc-creole:datenormalizer} for full details.

\textbf{Schema\_Tools} providing a ``Schema Enforcer'' PR that can be used to
create a clean output annotation set based on a set of annotation schemas. See
section~\ref{sec:misc-creole:schemaenforcer} for full details.

\textbf{Teamware\_Tools} providing a new PR called QA Summariser for Teamware.
When documents are annotated using GATE Teamware, this PR can be used for
generating a summary of agreements among annotators.  See
section~\ref{sec:eval:qaForTW} for full details.

\textbf{Tagger\_MetaMap} has been rewritten to make use of the new MetaMap Java
API features.  There are numerous performance enhancements and bug fixes
detailed in section~\ref{sec:misc-creole:metamap}.  Note that this version of
the plugin is \emph{not} compatible with the version provided in GATE 6.0,
though this earlier version is still available in the Obsolete directory if
required.

\rcSubsect{Other new features and improvements}

Added support for handling controller events to JAPE by making it possible
to define \texttt{ControllerStarted}, \texttt{ControllerFinished}, and 
\texttt{ControllerAborted} code blocks in a JAPE file (see 
section~\ref{sec:jape:javarhsoverview}).

JAPE Java right-hand-side code can now access an \texttt{ActionContext} object
through the predefined field \texttt{ctx} which allows access to the corpus LR
and the transducer PR and their features (see
section~\ref{sec:jape:javarhsoverview}).

Three new optional attributes can be specified in {\tt<GATECONFIG>}
element of {\tt gate.xml} or local configuration file:

\begin{itemize}
\item \textbf{addNamespaceFeatures} - set to ``true'' to deserialize namespace prefix and URI information as features.
\item \textbf{namespaceURI} - The feature name to use that will hold the namespace URI of the element, e.g. ``namespace''
\item \textbf{namespacePrefix} - The feature name to use that will hold the namespace prefix of the element, e.g. ``prefix''
\end{itemize}

Setting these attributes will alter GATE's default namespace deserialization
behaviour to remove the namespace prefix and add it as a feature, along with
the namespace URI.  This allows namespace-prefixed elements in the
\textbf{Original markups} annotation set to be matched with JAPE expressions,
and also allows namespace scope to be added to new annotations when serialized
to XML. See  \ref{sec:corpora:input} for details.

Searchable Serial Datastores (Lucene-based) are now portable and can be moved
across different systems. Also, several GUI improvements have been made to ease
the creation of Lucene datastores. See chapter~\ref{chap:annic} for details.

The populate method that allowed populating corpus from a trecweb file has been
made more generic to accept a tag. The method extracts content between the start
and end of this tag to create new documents. In GATE Developer, right-clicking 
on an instance of the Corpus and choosing the option ``Populate from Single
Concatenated File" allows users to populate the corpus using this functionality.
See Section \ref{sec:api:corpora} for more details.

Fixed a regression in the JAPE parser that prevented the use of RHS macros that
refer to a LHS label (named blocks \verb|:label { ... }| and assignments
\verb|:label.Type = {}|

Enhanced the Groovy scriptable controller with some features inspired by the
realtime controller, in particular the ability to ignore exceptions thrown by
PRs and the ability to limit the running time of certain PRs.  See
section~\ref{sec:api:groovy:controller} for details.

The Ontology and Gazetteer\_LKB plugins have been upgraded to use Sesame 3.2.3
and OWLIM 3.5.

The Websphinx Crawler PR (section~\ref{sec:misc-creole:crawler}) has new runtime
parameters for controlling the maximum page size and spoofing the user-agent.

A few bug fixes and improvements to the ``recover'' logic of the
\texttt{packagegapp} Ant task (see section~\ref{sec:ant:packagegapp}).

\ldots and many other smaller bugfixes.

\textbf{Note: As of version 6.1, GATE Developer and Embedded require Java 6 or
later and will no longer run on Java 5}.  If you require Java 5 compatibility
you should use GATE 6.0.

%%%%%%%%%%%%%%%%%%%%%%%%%%%%%%%%%%%%%%%%
\rcSect[6.0]{Version 6.0 (November 2010)}\ifnested\label{subsec:changes:6.0b1}\else\label{sec:changes:6.0b1}\fi

\rcSubsect{Major new features}

Added an annotation tool for the document editor: the Relation Annotation
Tool (RAT). It is designed to annotate a document with ontology instances
and to create relations between annotations with ontology object
properties. It is close and compatible with the Ontology Annotation Tool
(OAT) but focus on relations between annotations. See
section~\ref{sec:ontologies:rat} for details.

Added a new \emph{scriptable controller} to the Groovy plugin, whose execution
strategy is controlled by a simple Groovy DSL.  This supports more powerful
conditional execution than is possible with the standard conditional
controllers (for example, based on the presence or absence of a particular
annotation, or a combination of several document feature values), rich flow
control using Groovy loops, etc.  See
section~\ref{sec:api:groovy:controller} for details.

A new version of Alignment Editor has been added to the GATE distribution. It 
consists of several new features such as the new alignment viewer, ability to
create alignment tasks and store in xml files, three different views to align
the text (links view and matrix view - suitable for character, word and phrase
alignments, parallel view - suitable for sentence or long text alignment), an
alignment exporter and many more.  See chapter \ref{chap:alignment} for more 
information.

MetaMap, from the National Library of Medicine (NLM), maps biomedical text to 
the \textbf{UMLS Metathesaurus} and allows Metathesaurus concepts to be 
discovered in a text corpus. The Tagger\_MetaMap plugin for GATE wraps the 
MetaMap Java API client to allow GATE to communicate with a remote (or local) 
MetaMap PrologBeans \textbf{mmserver} and MetaMap distribution. This allows the 
content of specified annotations (or the entire document content) to be 
processed by MetaMap and the results converted to GATE annotations and features.
See section~\ref{sec:misc-creole:metamap} for details.

A new plugin called Web\_Translate\_Google has been added with a PR called 
Google Translator PR in it.  It allows users to translate text using the
Google translation services.  See section \ref{sec:misc-creole:google-translate}
for more information.

New Gazetteer Editor for ANNIE Gazetteer that can be used instead of
Gaze. It uses tables instead of text area to display the gazetteer
definition and lists, allows sorting on any column, filtering of the lists,
reloading a list, etc. See section~\ref{sec:gazetteers:anniegazeditor}.

\rcSubsect{Breaking changes}

This release contains a few small changes that are not backwards-compatible:
\begin{itemize}
\item Changed the semantics of the ontology-aware matching mode in JAPE to take
account of the default namespace in an ontology.  Now {\tt class} feature
values that are not complete URIs will be treated as naming classes within the
default namespace of the target ontology only, and not (as previously) any
class whose URI ends with the specified name.  This is more consistent with the
way OWL normally works, as well as being much more efficient to execute.  See
section~\ref{sec:ontologies:ontology-aware-jape} for more details.

\item Updated the WordNet plugin to support more recent releases of WordNet
than 1.6.  The format of the configuration file has changed, if you are using
the previous WordNet 1.6 support you will need to update your configuration.
See section~\ref{sec:misc-creole:wn} for details.

\item The deprecated Tagger\_TreeTagger plugin has been removed, applications
that used it will need to be updated to use the Tagger\_Framework plugin
instead.  See section~\ref{sec:parsers:taggerframework} for details of how to
do this.
\end{itemize}

\rcSubsect{Other new features and bugfixes}

The concept of {\it templates} has been introduced to JAPE.  This is a way to
declare named ``variables'' in a JAPE grammar that can contain placeholders
that are filled in when the template is referenced.  See
section~\ref{sec:jape:templates} for full details.

Added a JAPE operator to get the string covered by a left-hand-side label and
assign it to a feature of a new annotation on the right hand side (see
section~\ref{sec:jape:metaproperties}).

Added a new API to the CREOLE registry to permit plugins that live 
entirely on the classpath. {\tt CreoleRegister.registerComponent} instructs
the registry to scan a single java Class for annotations, adding it to the 
set of registered plugins.  See section~\ref{sec:api:plugins} for details.

Maven artifacts for GATE are now published to the central Maven
repository.  See section~\ref{sec:gettingstarted:maven} for details.

Bugfix: {\tt DocumentImpl} no longer changes its {\tt stringContent} parameter
value whenever the document's content changes.  Among other things, this means
that saved application states will no longer contain the full text of the
documents in their corpus, and documents containing XML or HTML tags that were
originally created from string content (rather than a URL) can now safely be
stored in saved application states and the GATE Developer saved session.

A processing resource called Quality Assurance PR has been added in the Tools
plugin. The PR wraps the functionality of the Quality Assurance Tool 
(section \ref{sec:eval:corpusqualityassurance}).

A new section for using the Corpus Quality Assurance from GATE Embedded has
been written. See section~\ref{sec:eval:corpusqualityassurance}.

The Generic Tagger PR (in the Tagger\_Framework plugin) now allows more
flexible specification of the input to the tagger, and is no longer limited to
passing just the ``string'' feature from the input annotations.  See
section~\ref{sec:parsers:taggerframework} for details.

Added new parameters and options to the LingPipe Language Identifier PR.
(section~\ref{sec:misc-creole:lingpipe:langid}), and corrected the
documentation for the LingPipe POS Tagger
(section~\ref{sec:misc-creole:lingpipe:postagger}).

In the document editor, fixed several exceptions to make editing text with
annotations highlighted working. So you should now be able to edit the text
and the annotations should behave correctly that is to say move, expand or
disappear according to the text insertions and deletions.

Options for document editor: read-only and insert append/prepend have been
moved from the options dialogue to the document editor toolbar at the top
right on the triangle icon that display a menu with the options. See
section~\ref{sec:developer:documents}.

Added new parameters and options to the Crawl PR and document features to its
output; see section~\ref{sec:misc-creole:crawler} for details.

Fixed a bug where ontology-aware JAPE rules worked correctly when the target
annotation's class was a subclass of the class specified in the rule, but
failed when the two class names matched exactly.

Improved support for conditional pipelines containing non-LanguageAnalyser
processing resources.

Added the current {\tt Corpus} to the script binding for the Groovy Script PR,
allowing a Groovy script to access and set corpus-level features.  Also added
callbacks that a Groovy script can implement to do additional pre- or
post-processing before the first and after the last document in a corpus.  See
section~\ref{sec:api:groovy} for details.

%%%%%%%%%%%%%%%%%%%%%%%%%%%%%%%%%%%%%%%%
\sect[sec:changes:5.2.1]{Version 5.2.1 (May 2010)}

This is a bugfix release to resolve several bugs that were reported shortly
after the release of version 5.2:

\begin{itemize}
\item Fixed some bugs with the automatic ``create instance'' feature in OAT
  (the ontology annotation tool) when used with the new {\tt Ontology} plugin.

\item Added validation to datatype property values of the \emph{date},
  \emph{time} and \emph{datetime} types.

\item Fixed a bug with Gazetteer\_LKB that prevented it working when the
  {\tt dictionaryPath} contained spaces.

\item Added a utility class to handle common cases of encoding URIs for use in
  ontologies, and fixed the example code to show how to make use of this.  See
  chapter~\ref{chap:ontologies} for details.

\item The annotation set transfer PR now copies the feature map of each
  annotation it transfers, rather than re-using the same FeatureMap (this means
  that when used to copy annotations rather than move them, the copied
  annotation is independent from the original and modifying the features of one
  does not modify the other).  See section~\ref{sec:misc-creole:ast} for
  details.

\item The Log4J log files are now created by default in the {\tt .gate}
  directory under the user's home directory, rather than being created in the
  current directory when GATE starts, to be more friendly when GATE is
  installed in a shared location where the user does not have write permission.
\end{itemize}

This release also fixes some shortcomings in the Groovy support added by 5.2,
in particular:

\begin{itemize}
\item The {\tt corpora} variable in the console now includes persistent corpora
  (loaded from a datastore) as well as transient corpora.
\item The subscript notation for annotation sets works with long values as well
  as ints, so \verb|someAS[annotation.start()..annotation.end()]| works as
  expected.
\end{itemize}

%%%%%%%%%%%%%%%%%%%%%%%%%%%%%%%%%%%%%%%%
\sect[sec:changes:5.2]{Version 5.2 (April 2010)}

\subsect{JAPE and JAPE-related}

Introduced a utility class
\htlink{http://gate.ac.uk/gate/doc/javadoc/gate/Utils.html}{gate.Utils}
containing static utility methods for frequently-used idioms such as getting
the string covered by an annotation, finding the start and end offsets of
annotations and sets, etc.  This class is particularly useful on the right hand
side of JAPE rules (section~\ref{sec:jape:javarhsoverview}).

Added type parameters to the \verb|bindings| map available on the RHS of JAPE
rules, so you can now do \verb|AnnotationSet as = bindings.get("label")|
without a cast (see section~\ref{sec:jape:javarhsoverview}).

Fixed a bug with JAPE's handling of features called ``class'' in
non-ontology-aware mode.  Previously JAPE would always match such features
using an equality test, even if a different operator was used in the grammar,
i.e. \verb|{SomeType.class != "foo"}| was matched as
\verb|{SomeType.class == "foo"}|.  The correct operator is now used.  Note that
this does not affect the ontology-aware behaviour: when an ontology parameter
is specified, ``class'' features are always matched using ontology subsumption.

Custom JAPE operators and annotation accessors can now be loaded from plugins
as well as from the \texttt{lib} directory (see
section~\ref{sec:jape:customoperators}).

\subsect{Other Changes}

Added a mechanism to allow plugins to contribute menu items to the ``Tools''
menu in GATE Developer.  See section~\ref{sec:creole-model:tools} for details.

Enhanced Groovy support in GATE: the Groovy console and Groovy Script PR (in
the Groovy plugin) now import many GATE classes by default, and a number of
utility methods are mixed in to some of the core GATE API classes to make them
more natural to use in Groovy.  See section~\ref{sec:api:groovy} for details.

Modified the batch learning PR (in the \texttt{Learning} plugin) to make it
safe to use several instances in {\it APPLICATION} mode with the same
configuration file and the same learned model at the same time (e.g. in a
multithreaded application).  The other modes (including training and
evaluation) are unchanged, and thus are still \emph{not} safe for use in this
way.  Also fixed a bug that prevented APPLICATION mode from working anywhere
other than as the last PR in a pipeline when running over a corpus in a
datastore.

Introduced a simple way to create duplicate copies of an existing resource
instance, with a way for individual resource types to override the default
duplication algorithm if they know a better way to deal with duplicating
themselves.  See section~\ref{sec:api:duplicate}.

Enhanced the Spring support in GATE to provide easy access to the new
duplication API, and to simplify the configuration of the built-in Spring
pooling mechanisms when writing multi-threaded Spring-based applications.  See
section~\ref{sec:api:spring}.

The GAPP packager Ant task now respects the ordering of mapping hints, with
earlier hints taking precedence over later ones (see
section~\ref{sec:ant:packagegapp:hints}).

Bug fix in the UIMA plugin from Roland Cornelissen - \texttt{AnalysisEnginePR}
now properly shuts down the wrapped \texttt{AnalysisEngine} when the PR is
deleted.

Patch from Matt Nathan to allow several instances of a gazetteer PR in an
embedded application to share a single copy of their internal data structures,
saving considerable memory compared to loading several complete copies of the
same gazetteer lists (see section~\ref{sec:gazetteers:shared}).

In the corpus quality assurance, measures for classification tasks have been
added. You can also now set the beta for the fscore. This tool has been
optimised to work with datastores so that it doesn't need to read all the
documents before comparing them.

%%%%%%%%%%%%%%%%%%%%%%%%%%%%%%%%%%%%%%%%
\sect[5.1]{Version 5.1 (December 2009)}
\label{sec:changes:5.1b1}\label{sec:changes:5.1b2}\htcode{<A name="sec:changes:5.1b1"></A><A name="sec:changes:5.1b2"></A>}

Version 5.1 is a major increment with lots of new features and integration of
a number of important systems from 3rd parties (e.g. LingPipe, OpenNLP,
OpenCalais, a revised UIMA connector). We've stuck with the 5 series (instead
of jumping to 6.0) because the core remains stable and backwards compatible.

Other highlights include:
\begin{itemize}
\item
an entirely new ontology API from Johann Petrak of OFAI (the old one is still
available but as a plugin)
\item
new benchmarking facilities for JAPE from Andrew Borthwick and colleagues at
Intelius
\item
new quality assurance tools from Thomas Heitz and colleagues at Ontotext and
Sheffield 
\item
a generic tagger integration framework from Ren\'e Witte of Concordia University
\item
several new code contributions from Ontotext, including a large
knowledge-based gazetteer and various plugin wrappers from Marin Nozchev,
Georgi Georgiev and colleagues
\item
a revised and reordered user guide, amalgamated with the programmers' guide
and other materials
\item
Groovy support, application composition, section-by-section processing and
lots of other bits and pieces
\end{itemize}

\subsect{New Features}

\subsubsect{LingPipe Support}

LingPipe is a suite of Java libraries for the linguistic analysis of human
language.  We have provided a plugin called `LingPipe' with wrappers for some 
of the resources available in the LingPipe library. For more details, see the 
section \ref{sec:misc-creole:lingpipe}.

\subsubsect{OpenNLP Support}

OpenNLP provides tools for sentence detection, tokenization, pos-tagging,
chunking and parsing, named-entity detection, and coreference. The tools use
Maximum Entropy modelling. We have provided a plugin called `OpenNLP' with
wrappers for some of the resources available in the OpenNLP Tools library. For
more details, see section \ref{sec:misc-creole:opennlp}.

\subsubsect{OpenCalais Support}

We added a new PR called `OpenCalais PR'. This will process a document
through the OpenCalais service, and add OpenCalais entity annotations to the
document. (This plugin was subsequently removed in GATE 8.4)

\subsubsect{Ontology API}

The ontology API (package \verb!gate.creole.ontology! has been changed, the 
existing ontology implementation based on Sesame1 and OWLIM2 
(package \verb!gate.creole.ontology.owlim!) has been moved into 
the plugin \verb!Ontology_OWLIM2!. An upgraded implementation 
based on Sesame2 and OWLIM3 that also provides a number of new 
features has been added as plugin \verb!Ontology!. 
See Section~\ref{sec:ontology:changes5.1} for a detailed description of
all changes.

\subsubsect{Benchmarking Improvements}

A number of improvements to the benchmarking support in GATE. JAPE transducers
now log the time spent in individual phases of a multi-phase grammar and by
individual rules within each phase. Other PRs that use JAPE grammars
internally (the pronominal coreferencer, English tokeniser) log the time taken
by their internal transducers. A reporting tool, called `Profiling Reports' 
under the `Tools' menu makes summary information easily available. For more
details, see chapter~\ref{chap:profiling}.

\subsubsect{GUI improvements}

To deal with quality assurance of annotations, one component has been
updated and two new components have been added. The annotation diff tool has
a new mode to copy annotations to a consensus set, see
section~\ref{sec:eval:adiff}. An annotation stack view has been added in the
document editor and it allows to copy annotations to a consensus set, see
section~\ref{sec:developer:annotationsstackview}. A corpus view has been
added for all corpus to get statistics like precision, recall and F-measure,
see section~\ref{sec:eval:corpusqualityassurance}.

An annotation stack view has been added in the document editor to make
easier to see overlapping annotations, see
section~\ref{sec:developer:annotationsstackview}.

\subsubsect{ABNER Support}

ABNER is A Biomedical Named Entity Recogniser, for finding entities such as
genes in text. We have provided a plugin called `AbnerTagger' with a wrapper
for ABNER. For more details, see section \ref{sec:parsers:abner}.

\subsubsect{Generic Tagger Support}

A new plugin has been added to provide an easy route to integrate taggers with
GATE. The Tagger\_Framework plugin provides examples of incorporating a number
of external taggers which should serve as a starting point for using other
taggers. See Section \ref{sec:parsers:taggerframework} for more details.


\subsubsect{Section-by-Section Processing}

We have added a new PR called `Segment Processing PR'. As the name suggests
this PR allows processing individual segments of a document independently of one
other. For more details, please look at the section 
\ref{sec:alignment:segment-processing}.


\subsubsect{Application Composition}

The {\tt gate.Controller} implementations provided with the main GATE 
distribution now also implement the {\tt gate.ProcessingResource} interface.
This means that an application can now contain another application as one of
its components.

\subsubsect{Groovy Support}

Groovy is a dynamic programming language based on Java. You can now use it as
a scripting language for GATE, via the Groovy Console. For more
details, see Section \ref{sec:api:groovy}.

\subsect{JAPE improvements}

GATE now produces a warning when any Java right-hand-sides in JAPE rules make
use of the deprecated {\tt annotations} parameter.  All bundled JAPE grammars
have been updated to use the replacement {\tt inputAS} and {\tt outputAS}
parameters as appropriate.

The new \verb|Imports:| statement at the beginning of a JAPE grammar file can
now be used to make additional Java import statements available to the Java RHS
code, see~\ref{sec:jape:javarhsoverview}.

The JAPE debugger has been removed. Debugging of JAPE has been made easier
as stack traces now refer to the JAPE source file and line numbers instead
of the generated Java source code.

The Montreal Transducer has been made obsolete.

\subsect{Other improvements and bug fixes}

Plugin names have been rationalised. Mappings exist so that existing
applications will continue to work, but the new names should be used in the
future. Plugin name mappings are given in Appendix~\ref{chap:plugin-names-map}.
Also, the Segmenter\_Chinese plugin (used to be known as chineseSegmenter
plugin) is now part of the Lang\_Chinese plugin.

The User Guide has been amalgamated with the
Programmer's Guide; all material can now be found in the User Guide. The
`How-To' chapter has been converted into separate chapters for
installation, GATE Developer and GATE Embedded. Other material has been
relocated to the appropriate specialist chapter.

Made Mac OS launcher 64-bit compatible.  See
section~\ref{sec:gettingstarted:easy} for details.

The UIMA integration layer (\Chapthing\ \ref{chap:uima}) has been upgraded to
work with Apache UIMA 2.2.2.

Oracle and PostGreSQL are no longer supported.

The MIAKT Natural Language Generation plugin has been removed.

The Minorthird plugin has been removed. Minorthird has changed significantly
since this plugin was written. We will consider writing an up-to-date
Minorthird plugin in the future.

A new gazetteer, Large KB Gazetteer (in the plugin `Gazetteer\_LKB') has been
added, see Section~\ref{sec:gazetteers:lkb-gazetteer} for details.

gate.creole.tokeniser.chinesetokeniser.ChineseTokeniser and related resources 
under the plugins/ANNIE/tokeniser/chinesetokeniser folder have been removed.
Please refer to the Lang\_Chinese plugin for resources related to the Chinese 
language in GATE.

Added an {\tt isInitialised()} method to {\tt gate.Gate()}.

Added a parameter to the chemistry tagger PR
(section~\ref{sec:parsers:chemistrytagger}) to allow it to operate on
annotation sets other than the default one.

Plus many more smaller bugfixes...


%%%%%%%%%%%%%%%%%%%%%%%%%%%%%%%%%%%%%%%%
\sect[sec:changes:5.0]{Version 5.0 (May 2009)}
\label{sec:changes:5.0b1}\htcode{<A name="sec:changes:5.0b1"></A>}

\begin{quote}
\textbf{Note:} {\em existing users -- if you delete your user configuration
file for any reason you will find that GATE Developer no longer loads the ANNIE plugin
by default.  You will need to manually select `load always' in the plugin
manager to get the old behaviour.}
\end{quote}

\subsect{Major New Features}

\subsubsect{JAPE Language Improvements}

Several new extensions to the JAPE language to support more flexible pattern
matching.  Full details are in \Chapthing~\ref{chap:jape} but briefly:

\begin{itemize}
\item Negative constraints, that prevent a rule from matching if certain other
      annotations are present (Section~\ref{sec:jape:negation}).
\item Additional matching operators for feature values, so you can now look for
      \verb|{Token.length < 5}|, \verb|{Lookup.minorType != "ignore"}|, etc.~as
      well as simple equality (Section~\ref{sec:jape:matchingoperators}).
\item `Meta-property' accessors, see Section~\ref{sec:jape:metaproperties} to
permit access to the string covered by an annotation, the length of the annotation, etc., e.g.
      \verb|{Lookup@length > 4}|.
\item Contextual operators, allowing you to search for one annotation contained
      within (or containing) another, e.g.
      \verb|{Sentence contains {Lookup.majorType == "location"}}| (see
      Section~\ref{sec:jape:operators:contextual}).
\item Additional Kleene operator for ranges, e.g. \verb|({Token})[2,5]| matches
      between 2 and 5 consecutive tokens, see Section~\ref{sec:jape:ranges}.
\item Additional operators can be added via runtime configuration (see
Section~\ref{sec:jape:customoperators}).
\end{itemize}

Some of these extensions are similar to, but not the same as, those provided by
the Montreal Transducer plugin.  If you are already familiar with the Montreal
Transducer, you should first look at Section~\ref{sec:jape:montrealdifferences}
which summarises the differences.

\subsubsect{Resource Configuration via Java 5 Annotations}

Introduced an alternative style for supplying resource configuration
information via Java 5 annotations rather than in {\tt creole.xml}.  The
previous approach is still fully supported as well, and the two styles can be
freely mixed.  See Section \ref{sec:creole-model:config} for full details.

\subsubsect{Ontology-Based Gazetteer}

Added a new plugin `Gazetteer\_Ontology\_Based', which contains OntoRoot
Gazetteer -- a dynamically created gazetteer which is, in combination with few
other generic resources, capable of producing ontology-aware annotations
over the given content with regards to the given ontology. For more details see
Section~\ref{sec:gazetteers:ontoRootGaz}.

\subsubsect{Inter-Annotator Agreement and Merging}

New plugins to support tasks involving several annotators working on the same
annotation task on the same documents.  The plugin
`Inter\_Annotator\_Agreement' (Section~\ref{sec:eval:iaaplugin}) computes
inter-annotator agreement scores between the annotators, the
`Copy\_Annots\_Between\_Docs' plugin
(Section~\ref{sec:misc-creole:copyAS2AnoDoc}) copies annotations from several parallel documents into a single master document,
and the `Annotation\_Merging' plugin (Section~\ref{sec:misc-creole:merging})
merges annotations from multiple annotators into a single `consensus'
annotation set.

\subsubsect{Packaging Self-Contained Applications for GATE Teamware}

Added a mechanism to assemble a saved GATE application along with all the
resource files it uses into a single self-contained package to run on another
machine (e.g. as a service in \htlink{http://gate.ac.uk/teamware}{GATE
Teamware}).  This is available as a menu option
(Section~\ref{sec:developer:export}) which will work for most common cases, but for
complex cases you can use the underlying Ant task described in
Section~\ref{sec:ant:packagegapp}.

\subsubsect{GUI Improvements}

\begin{itemize}
\item A new schema-driven tool to streamline manual annotation tasks
(see Section~\ref{sec:developer:schemaannotationeditor}).

\item Context-sensitive help on elements in the resource tree and when
pressing F1 key. Search in mailing list from the Help menu. Help is displayed
in your browser or in a Java browser if you don't have one.

\item Improved search function inside documents with a regular expression
builder. Search and replace annotation function in all annotation editors.

\item Remember for each resource type the last path used when loading/saving
a resource.

\item Remember the last annotations selected in the annotation set
view when you shift click on the annotation set view button.

\item Improved context menu and when possible added drag and drop in:
resource tree, annotation set view, annotation list view, corpus view,
controller view. Context menu key can be now used if you have Java 1.6.

\item New dialog box for error messages with user oriented messages,
optional display of the configuration and proposing some useful
actions. This will progressively replace the old stack trace dump into the
message panel which is still here for the moment but should be hide by
default in the future.

\item Add read-only document mode that can be enable from the Options menu.

\item Add a selection filter in the status bar of the annotations list table
to easily select rows based on the text you enter.

\item Add the last five applications loaded/saved in the context menu of the
language resources in the resources tree.

\item Display more informations on what going's on in the waiting dialog box
when running an application. The goal is to improve it to get a global
progress bar and estimated time.
\end{itemize}

\subsect{Other New Features and Improvements}

\begin{itemize}
\item New parser plugins: A new plugin for the 
  \htlink{http://nlp.stanford.edu/software/lex-parser.shtml}{Stanford Parser}
  (see Section~\ref{sec:parsers:stanford}) and a rewritten plugin for the 
  \htlink{http://www.informatics.sussex.ac.uk/research/groups/nlp/rasp/}{RASP
  NLP tools} (Section~\ref{sec:parsers:rasp}).

\item A new sentence splitter, based on regular expressions, has been added to
  the ANNIE plugin. More details in Section~\ref{sec:annie:regex-splitter}.

\item `Real-time' corpus controller (Section~\ref{sec:creole-model:applications}), which
  terminates processing of a document if it takes longer than a configurable
  timeout..
  
\item Major update to Annie OrthoMatcher coreference engine.  Now correctly matches 
   the sequence `David Jones ... David ... David Smith ... David' as referring to 
   two people.  Also handles nicknames (David = Dave) via a new nickname list.  Added
   optional parameter `highPrecisionOrgs', which if set to true turns off riskier org 
   matching rules.  Many misc. bug fixes.

\item Improved alignment editor (\Chapthing~\ref{chap:alignment}) with several
  advanced features and an API for adding your own actions to the editor.

\item A new plugin for Chinese word segmentation, which is
  based on our work using machine learning algorithms for the Sighan-05 Chinese
  word segmentation task. It can learn a model from manually segmented text, and
  apply a learned model to segment Chinese text. In addition several learned
  models are available with the plugin, which can be used to segment text. For
  details about the plugin and those learned models see Section
  \ref{sec:misc-creole:chineseSeg}.

\item New features in the ML API to produce an n-gram based language model from
  a corpus and a so-called `document-term matrix' (see
  Section~\ref{sec:misc-creole:ir}).  Also introduced features to support active learning,
  a new learning algorithm (PAUM) and various optimisations including the
  ability to use an external executable for SVM training.  Full details in
  \Chapthing~\ref{chap:ml}.

\item A new plugin to compute BDM scores for an ontology. The BDM score
  can be used to evaluate ontology based information extraction and
  classification. For details about the plugin see Section
  \ref{sec:eval:bdmplugin}.

\item Added new `getCovering' method to AnnotationSet.  This method returns
  annotations that completely span the provided range.  An optional annotation
  type parameter can be provided to further limit the returned set.

\item Complete redesign of ANNIC GUI. More details in
  Section~\ref{chap:annic}.
\end{itemize}

\subsect{Specific Bug Fixes}

\begin{itemize}
\item HTML document format parser: several bugs fixed, including a
  null pointer exception if the document contained certain characters illegal
  in HTML
  (\htlink{https://sourceforge.net/support/tracker.php?aid=1754749}{\#1754749}).
  Also, the HTML parser now respects the `Add space on markup unpack'
  configuration option -- previously it would always add space, even if the
  option was set to false.
  
\item Fixed a severe performance bug in the Annie Pronominal Coreferencer
  resulting in a 50X speed improvement.

\item JAPE did not always correctly handle the case when the input and output
  annotation sets for a transducer were different.  This has now been fixed.

\item `Save Preserving Format' was not correctly escaping ampersands
  and less than signs when two HTML entities are close together. Only the first
  one was replaced: A \& B \& C was output as A \&amp; B \& C instead of A
  \&amp; B \&amp; C. This has now been fixed, and the fix is also valid for the
  flexible exporter but only if the standoff annotations parameter is set to
  false.
\end{itemize}

Plus many more minor bug fixes

%%%%%%%%%%%%%%%%%%%%%%%%%%%%%%%%%%%%%%%%
%\sect{April 2009}
%
%\begin{itemize}
%
%\item Create a plugin for Chinese word segmentation, which is
% based on our work of using machine learning algorithms for the Sighan-05 
%Chinese word segmentation task. It can learn a model from the segmented text.
% It can also apply a learned model to segment Chinese text. In addition
% several learned models are available with the plugin, which can be used
%to segment text. For details about the 
%plugin and those learned models see Section \ref{sec:misc-creole:chineseSeg}.
%
%\end{itemize}
%
%%%%%%%%%%%%%%%%%%%%%%%%%%%%%%%%%%%%%%%%
%\sect{March 2009}
%
%\begin{itemize}
%
%\item {\bf Important change}: ANNIE plugin is no more loaded by default so you
%need to load it first before to use any of its processing resources. This is
%done in order to be able to use GATE without ANNIE.
%
%\item Correct the output of 'Save Preserving Format' when
%two HTML entities are close. Only the first one was replaced: A \& B \& C
%was output as A \&amp; B \& C instead of A \&amp; B \&amp; C. This
%correction is also valid for the flexible exporter but only if the standoff
%annotations parameter is set to false.
%
%\item Create a plugin for computing the BDM scores for an ontology. The BDM score
%can be used for evaluating the ontology based information extraction and
%classification. For details about the plugin see Section \ref{sec:gui:bdmplugin}.
%
%\item Update the IAA plugin such that it can compute the BDM based F-measures now.
%For more see Section \ref{sec:gui:iaaplugin}.
%\end{itemize}
%
%\subsubsect{GUI improvements}
%
%\begin{itemize}
%\item New dialog box for error messages with user oriented messages,
%optional display of the configuration and proposing some useful
%actions. This will progressively replace the old stack trace dump into the
%message panel which is still here for the moment but should be hide by
%default in the future.
%
%\item Add read-only document mode that can be enable from the Options menu.
%
%\item Add a selection filter in the status bar of the annotations list table
%to easily select rows based on the text you enter.
%
%\item Add the last five applications loaded/saved in the context menu of the
%language resources in the resources tree.
%
%\item Display more informations on what going's on in the waiting dialog box
%when running an application. The goal is to improve it to get a global
%progress bar and estimated time.
%\end{itemize}
%
%%%%%%%%%%%%%%%%%%%%%%%%%%%%%%%%%%%%%%%%
%\sect{February 2009}
%
%Improved the scalability of the machine learning PR (i.e. it can use large 
%training data without memory problem and also faster) in three ways: improved
%the memory usage by optimising the codes; implemented the PAUM learning algorithm
%which can achieve the similar performance as SVM but is much faster and needs
%less memory; providing one option of running the SVM learning outside of the GATE
%(i.e. the learning engine {\em SVMExec} in the configuration file). For more details
%about the last two ways, see Section \ref{subsect-settings}.
%
%Added a mechanism to assemble a saved GATE application along with all the
%resource files it uses into a single self-contained package to run on another
%machine (e.g. as a service in \htlink{http://gate.ac.uk/teamware}{GATE
%Teamware}).  This is available as a menu option
%(section~\ref{sec:howto:export}) which will work for most common cases, but for
%complex cases you can use the underlying Ant task described in
%section~\ref{sec:ant:packagegapp}.
%
%\subsubsect{GUI improvements}
%
%\begin{itemize}
%\item Remember the last annotations selected in the annotation set
%view when you shift click on the annotation set view button.
%
%\item Improved context menu and when possible added drag and drop in:
%resource tree, annotation set view, annotation list view, corpus view,
%controller view. Context menu key can be now used if you have Java 1.6.
%\end{itemize}

%%%%%%%%%%%%%%%%%%%%%%%%%%%%%%%%%%%%%%%%
%\sect{October 2008}
%
%\subsect{iaaPlugin}
%
%Added a new plugin `iaaPlugin', which computes inner-annotator agreement (IAA)
%for two or more annotators for the same annotation task on the same documents.
%For text classification tasks such as document, sentence or token classification,
%it computes the IAA measures such as observed agreement, Cohen's kappa, and Scott's pi,
%among others (see Appendix \ref{chap:iaakappa} for the explanations about those
%measures). For named entity annotation task, it computes the F-measures as IAA.
%For full details see Section \ref{sec:gui:iaaplugin}.
%
%\subsect{Tools for Alignment Tasks}
%
%A better alignment editor with several advanced features. API that allows
%adding more features to the editor is explained under the section 
%\ref{sec:alignment:editor:advanced-features}.
%
%%%%%%%%%%%%%%%%%%%%%%%%%%%%%%%%%%%%%%%%%
%\sect{September 2008}
%
%Added new `getCovering' method to AnnotationSet.  This method returns
%annotations that completely span the provided range.  An optional annotation
%type parameter can be provided to further limit the returned set.
%
%\subsect{JAPE Updates}
%Added annotation `meta-property' accessors. The three that are built in are
%length, string, and cleanString.  See chapter~\ref{chap:jape} for full
%details.
%
%Added custom boolean operators.  In addition to the standard \verb|==|,
%\verb|<|, \verb|>|, etc, two new built in operators are \texttt{contains} and
%\texttt{within}.  See section \ref{sec:jape:operators:contextual} for full
%details.
%
%Add support for a range Kleene operator. e.g. \verb|({Token})[1,3]| matches one
%to three Tokens in a row.
%
%Added parameters to Transducer objects to allow additional annotation
%accessors and custom operators to be referenced without having to change
%the JAPE language.
%
%%%%%%%%%%%%%%%%%%%%%%%%%%%%%%%%%%%%%%%%%
%\sect{August 2008}
%
%Introduced an alternative style for supplying resource configuration
%information via Java 5 annotations rather than in {\tt creole.xml}.  The
%previous approach is still fully supported as well, and the two styles can be
%freely mixed.  See section \ref{sec:creole-model:config} for full details.
%
%%%%%%%%%%%%%%%%%%%%%%%%%%%%%%%%%%%%%%%%%
%\sect{July 2008}
%Bug fix: the HTML document format parser now respects the "Add space on markup
%unpack" configuration option. Previously it would always behave as if the
%option was set to true (the default).
%
%\subsect{JAPE Updates}
%A major update to the JAPE language, incorporating features inspired by (though
%not exactly the same as) the Montreal Transducer.  Full details are in
%chapter~\ref{chap:jape}, but briefly:
%\begin{itemize}
%\item Negative constraints, that prevent a rule from matching if certain other
%      annotations are present (section~\ref{sec:jape:negation}).
%\item Additional matching operators for feature values, so you can now look for
%      \verb|{Token.length < 5}|, \verb|{Lookup.minorType != "ignore"}|, etc.~as
%      well as simple equality (section~\ref{sec:jape:operators}).
%\end{itemize}
%
%If you are already familiar with the Montreal Transducer, you should first look
%at section~\ref{sec:jape:montrealdifferences} which summarises the differences.
%
%%%%%%%%%%%%%%%%%%%%%%%%%%%%%%%%%%%%%%%%%
%\sect{May 2008}
%%
%Added a new plugin Ontology\_Based\_Gazetteer, which contains OntoRoot
%Gazetteer -- a dynamically created gazetteer which is, in combination with few
%other generic GATE resources, capable of producing ontology-aware annotations
%over the given content with regards to the given ontology. For more details see
%section~\ref{sec:gazetteers:ontoRootGaz}.
%
%Added the new Real-time corpus controller. See
%section~\ref{sec:applications}.
%%
%\sect{March 2008}
%%
%Added a new plugin for the
%\htlink{http://nlp.stanford.edu/software/lex-parser.shtml}{Stanford
%  Parser}; see section~\ref{sec:parsers:stanford}.
%
%Added a new plugin for using the RASP NLP tools
%(\htlink{http://www.informatics.sussex.ac.uk/research/groups/nlp/rasp/}{RASP
%  web page}) and moved the old one into Obsolete/rasp; see
%section~\ref{sec:parsers:rasp}.
%%%%%%%%%%%%%%%%%%%%%%%%%%%%%%%%%%%%%%%%%
%\sect{January 2008}
%Added some new functions to the ML API: it now can produce the n-gram based language model
%from a corpus and also the so-called document-term matrix described in
%Section \ref{sect:ir}. It also provides the facilities for SVM active learning,
%mainly ranking the unlabelled documents according to their confidence scores
%to the current learned SVM models. Also add one option to speed up
%the processing of extracting feature vector from linguistic features.
%
%%%%%%%%%%%%%%%%%%%%%%%%%%%%%%%%%%%%%%%%%
%\sect{October 2007}
%A new plugin called {\em Annotation\_Merging} has been added to GATE. It
%can be used for merging annotations from different annotators about the
%same subject on the same document. It uses one of the
%two implemented merging methods, the majority voting method and another one based
%on the minimal number of annotators agreeing on the merged annotation.
%See section~\ref{sec:misc-creole:merging} for more details.
%
%%%%%%%%%%%%%%%%%%%%%%%%%%%%%%%%%%%%%%%%%
%\sect{September 2007}
%A new sentence splitter, based on regular expressions, has been added to the
%ANNIE plugin. More details in section~\ref{sec:regex-splitter}.
%
%%%%%%%%%%%%%%%%%%%%%%%%%%%%%%%%%%%%%%%%%
%\sect{August 2007}
%
%Fixed a bug
%(\htlink{https://sourceforge.net/support/tracker.php?aid=1754749}{\#1754749})
%in the HTML document format parser where it would fail with a null pointer
%exception if the document contained certain characters illegal in HTML.
%
%Fixed a bug in the JAPE parser which showed up in certain circumstances where
%the input and output annotation sets are different.
%
%%%%%%%%%%%%%%%%%%%%%%%%%%%%%%%%%%%%%%%%
\sect[sec:changes:4.0]{Version 4.0 (July 2007)}

\subsect{Major New Features}

\subsubsect{ANNIC}
ANNotations In Context: a full-featured annotation indexing and retrieval
system designed to support corpus querying and JAPE rule authoring.  It is
provided as part of an extension of the Serial Datastores, called Searchable
Serial Datastore (SSD). See Section~\ref{chap:annic} for more
details.

\subsubsect{New Machine Learning API}
A brand new machine learning layer specifically targetted at NLP tasks
including text classification, chunk learning (e.g. for named entity
recognition) and relation learning.  See \Chapthing~\ref{chap:ml} for more
details.

\subsubsect{Ontology API}
A new ontology API, based on \htlink{http://www.ontotext.com/owlim/}{OWL
In Memory} (OWLIM), which offers a better API, revised ontology event model and
an improved ontology editor to name but few. See
\Chapthing~\ref{chap:ontologies} for more details.

\subsubsect{OCAT}
Ontology-based Corpus Annotation Tool to help annotators to manually annotate
documents using ontologies. For more details please see
Section~\ref{sec:ontologies:ocat}.

\subsubsect{Alignment Tools}
A new set of components (e.g. CompoundDocument, AlignmentEditor etc.) that help
in building alignment tools and in carrying out cross-document processing.  See
\Chapthing~\ref{chap:alignment} for more details.

\subsubsect{New HTML Parser}
A new HTML document format parser, based on Andy Clark's
\htlink{http://people.apache.org/~andyc/neko/doc/html/}{NekoHTML}.  This parser
is much better than the old one at handling modern HTML and XHTML constructs,
JavaScript blocks, etc., though the old parser is still available for existing
applications that depend on its behaviour.

\subsubsect{Java 5.0 Support}
GATE now requires Java 5.0 or later to compile and run.
This brings a number of benefits:
\begin{itemize}
\item Java 5.0 syntax is now available on the right hand side of JAPE rules
  with the default Eclipse compiler.  See
  Section~\ref{sec:jape:javarhs} for details.
\item \texttt{enum} types are now supported for resource parameters.  see
  Section \ref{sec:api:bootstrap} for details on defining the parameters
  of a resource.
\item \texttt{AnnotationSet} and the \texttt{CreoleRegister} take advantage of
  generic types.  The \texttt{AnnotationSet} interface is now an extension of
  \texttt{Set<Annotation>} rather than just \texttt{Set}, which should make for
  cleaner and more type-safe code when programming to the API, and the
  \texttt{CreoleRegister} now uses parameterized types, which are
  backwards-compatible but provide better type-safety for new code.
\end{itemize}

\subsect{Other New Features and Improvements}
\begin{itemize}

\item Hiding the view for a particular resource (by right clicking on its tab
and selecting `Hide this view') will now completely close the associated
viewers and dispose them. Re-selecting the same resource at a later time will
lead to re-creating the necessary viewers and displaying them. This has two
advantages: firstly it offers a mechanism for disposing views that are not
needed any more without actually closing the resource and secondly it provides
a way to refresh the view of a resource in the situations where it becomes
corrupted.

\item The DataStore viewer now allows multiple selections. This lets users load
or delete an arbitrarily large number of resources in one operation.

\item The Corpus editor has been completely overhauled. It now allows
re-ordering of documents as well as sorting the document list by either index
or document name.

\item Support has been added for resource parameters of type
  \texttt{gate.FeatureMap}, and it is also possible to specify a default value
  for parameters whose type is \texttt{Collection}, \texttt{List} or
  \texttt{Set}.  See Section~\ref{sec:api:plugins} for details.

\item (Feature Request
  \htlink{https://sourceforge.net/support/tracker.php?aid=1446642}{\#1446642})
  After several requests, a mechanism has been added to allow overriding of
  GATE's document format detection routine.  A new creation-time parameter
  {\tt mimeType} has been added to the standard document implementation, which
  forces a document to be interpreted as a specific MIME type and prevents the
  usual detection based on file name extension and other information.  See
  Section~\ref{sec:corpora:detecting-reader} for details.

\item A capability has been added to specify arbitrary sets of additional
  features on individual gazetteer entries.  These features are passed forward
  into the Lookup annotations generated by the gazetteer.  See
  Section~\ref{sec:annie:gazetteer} for details.

\item As an alternative to the Google plugin, a new plugin called
  \textit{yahoo} has been added to to allow users to submit their query to
  the Yahoo search engine and to load the found pages as GATE documents.  See
  Section~\ref{sec:misc-creole:yahoo} for more details.

\item It is now easier to run a corpus pipeline over a single document in the
  GATE Developer GUI -- documents now provide a right-click menu item to create a
  singleton corpus containing just this document.  See Section
  \ref{sec:developer:loadlr} for details.

\item A new interface has been added that lets PRs receive notification at the
  start and end of execution of their containing controller.  This is useful
  for PRs that need to do cleanup or other processing after a whole corpus has
  been processed.  See Section~\ref{sec:creole-model:applications} for details.

\item The GATE Developer GUI does not call System.exit() any more when it is closed.
  Instead an effort is made to stop all active threads and to release all
  GUI resources, which leads to the JVM exiting gracefully. This is
  particularly useful when GATE is embedded in other systems as closing the
  main GATE window will not kill the JVM process any more.

\item The set of AnnotationSchemas that used to be included in the core
  gate.jar and loaded as builtins have now been moved to the ANNIE plugin. When
  the plugin is loaded, the default annotation schemas are instantiated
  automatically and are available when doing manual annotation.

\item There is now support in creole.xml files for automatically creating
  instances of a resource that are hidden (i.e. do not show in the GUI). One
  example of this can be seen in the creole.xml file of the ANNIE plugin where
  the default annotation schemas are defined.

\item A couple of helper classes have been added to assist in using GATE within
  a \htlink{http://www.springframework.org}{Spring} application.
  Section~\ref{sec:api:spring} explains the details.

\item Improvements have been made to the thread-safety of some internal
  components, which mean that it is now safe to create resources in
  multiple threads (though it is not safe to use the same resource instance in
  more than one thread).  This is a big advantage when using GATE in a
  multithreaded environment, such as a web application.  See Section
  \ref{sec:api:multithread} for details.

\item Plugins can now provide custom icons for their PRs and LRs in the plugin
  JAR file.  See Section \ref{sec:api:bootstrap} for details.

\item It is now possible to override the default location for the saved session
  file using a system property.  See Section~\ref{sec:gettingstarted:sysprop} for
  details.

\item The TreeTagger plugin (`Tagger\_TreeTagger') supports a system property
to specify the location of the shell interpreter used for the tagger shell
script.  In combination
  with Cygwin this makes it much easier to use the tagger on Windows.  
  % See Section~\ref{sec:parsers:treetagger} for details.

\item The {\tt Buchart} plugin has been removed.  It is superseded by SUPPLE,
  and instructions on how to upgrade your applications from Buchart to SUPPLE
  are given in Section~\ref{sec:parsers:supple}.  The probability finder
  plugin has also been removed, as it is no longer maintained.

\item The bootstrap wizard now creates a basic plugin that builds with Ant.
  Since a Unix-style make command is no longer required this means that the
  generated plugin will build on Windows without needing Cygwin or MinGW.

\item The GATE source code has moved from CVS into Subversion.  See
  Section~\ref{sec:gettingstarted:svn} for details of how to check out the code from the
  new repository.

\item An optional parameter, keepOriginalMarkupsAS, has been added to the
  DocumentReset PR which allows users to decide whether to keep the Original
  Markups AS or not while reseting the document. See
  Section~\ref{sec:misc-creole:reset} for more details.

\end{itemize}

\subsect{Bug Fixes and Optimizations}
\begin{itemize}
\item The Morphological Analyser has been optimized.  A new FSM based, although
  with minor alteration to the basic FSM algorithm, has been implemented to
  optimize the Morphological Analyser. The previous profiling figures show
  that the morpher when integrated with ANNIE application used to take upto
  60\% of the overall processing time.  The optimized version only takes 7.6\%
  of the total processing time. See Section~\ref{sec:parsers:morpher} for
  more details on the morpher.

\item The ANNIE Sentence Splitter was optimised. The new version is about twice
  as fast as the previous one. The actual speed increase varies widely
  depending on the nature of the document.

\item The imlementation of the \textit{OrthoMatcher} component has been
  improved.  This resources takes significantly less time on large documents.

\item The implementation of \texttt{AnnotationSets} has been improved. GATE now
  requires up to 40\% less memory to run and is also 20\% faster on average.
  The \texttt{get} methods of \texttt{AnnotationSet} return instances of
  \texttt{ImmutableAnnotationSet}. Any attempt at modifying the content of
  these objects will trigger an \texttt{Exception}. An empty
  \texttt{ImmutableAnnotationSet} is returned instead of \texttt{null}.

\item The Chemistry tagger (Section~\ref{sec:parsers:chemistrytagger}) has
  been updated with a number of bugfixes and improvements.

\item The Document user interface has been optimised to deal better with large
bursts of events which tend to occur when the document that is currently
displayed gets modified. The main advantages brought by this new implementation
are:
\begin{itemize}
  \item The document UI refreshes faster than before.
  \item The presence of the GUI for a document induces a smaller performance
  penalty than it used to. Due to a better threading implementation, machines
  benefiting from multiple CPUs (e.g. dual CPU, dual core or hyperthreading
  machines) should only see a negligible increase in processing time when a
  document is displayed compared to the situations where the document view is
  not shown. In the previous version, displaying a document while it was
  processed used to increase execution time by an order of magnitude.
  \item The GUI is more responsive now when a large number of annotations are
  displayed, hidden or deleted.
  \item The strange exceptions that used to occur occasionally while working
  with the document GUI should not happen any more.
\end{itemize}

\end{itemize}
And as always there are many smaller bugfixes too numerous to list here...

%%%%%%%%%%%%%%%%%%%%%%%%%%%%%%%%%%%%%%%%
% \sect{March 2007}
%
% A new corpus annotation tool, OCAT, has been added to GATE, which assists
% annotators to manually annotate documents using ontologies. For more details
% please see section~\ref{sec:ontologies:ocat}.
%
% A new ontology API, based on OWLIM\footnote{http://www.ontotext.com/owlim/} has
% been added to GATE. It offers a better API, revised ontology event model and an
% improved ontology editor to name but few. See chapter~\ref{chap:ontologies} for
% more details.
%
% A new corpus query and rule creation tool, ANNIC (Annotations in Context), has
% been added to GATE. ANNIC is a full-featured annotation indexing and retrieval
% system. It is provided as part of an extention of the Serial Datastores, called
% Searchable Serial Datastore (SSD). See section~\ref{sec:misc-creole:annic} for
% more details.
%
% The ANNIE Sentence Splitter was optimised. The new version is about twice as
% fast as the previous one. The actual speed increase varies widely depending on
% the nature of the document.
%
%
% %%%%%%%%%%%%%%%%%%%%%%%%%%%%%%%%%%%%%%%%
% \sect{February 2007}
%
% A new plugin called \textit{Web\_Search\_Yahoo} has been added to GATE to
% allow users to
% submit their query to Yahoo and to load the found pages as GATE documents.  See
% section~\ref{sec:misc-creole:yahoo} for more details.
%
% The imlementation of the \textit{OrthoMatcher} component has been improved.
% This resources takes significantly less time on large documents.
%
% The Corpus editor has been completely overhauled. It now allows re-ordering of
% documents as well as sorting the document list by either index or document name.
%
% The GATE GUI does not call System.exit() any more when it is closed. Instead an
% effort is made to stop all active GATE threads and to release all GUI resources,
% which leads to the JVM exiting gracefully. This is particularly useful when GATE
% is embedded in other systems as closing the main GATE window will not kill the
% JVM process any more.
%
% %%%%%%%%%%%%%%%%%%%%%%%%%%%%%%%%%%%%%%%%
% \sect{January 2007}
%
% Support has been added for resource parameters of type
% \texttt{gate.FeatureMap}.  See section~\ref{sec:howto:creoleconfig} for
% details.
%
% %%%%%%%%%%%%%%%%%%%%%%%%%%%%%%%%%%%%%%%%
% \sect{November 2006}
%
% A couple of subtle bugs relating to thread-safety of CREOLE
% \texttt{ResourceData} have been fixed.  It is now safe to create resources
% in multiple threads (though it is not safe to use the same resource instance in
% more than one thread).  See section \ref{sec:howto:multithread} for details.
%
% A capability has been added to specify arbitrary sets of additional features on
% individual gazetteer entries.  These features are passed forward into the
% Lookup annotations generated by the gazetteer.  See section~\ref{sec:gazetteer}
% for details.
%
% The implementation of \texttt{AnnotationSets} has been improved. GATE now requires
% up to 40\% less memory to run and is also 20\% faster on average.
%
% The \texttt{get} methods of \texttt{AnnotationSet} return instances of
% \texttt{ImmutableAnnotationSet}. Any attempt at modifying the content of these objects
% will trigger an \texttt{Exception}. An empty \texttt{ImmutableAnnotationSet} is returned
% instead of \texttt{null}.
%
% %%%%%%%%%%%%%%%%%%%%%%%%%%%%%%%%%%%%%%%%
% \sect{October 2006}
% GATE has officially become version 4.0-alpha1.
%
% The \texttt{AnnotationSet} interface is now an extension of
% \texttt{Set<Annotation>} rather than just \texttt{Set}, which should make for
% cleaner and more type-safe code when programming to the API.  Existing code
% that uses \texttt{AnnotationSet}s does not need to change, but if you have a
% class that {\em implements} \texttt{AnnotationSet} you will need to update it.
% Similarly, the \texttt{CreoleRegister} now uses parameterized types, which are
% backwards-compatible but provide better type-safety for new code.
%
% Java 5.0 \texttt{enum} types are now supported for resource parameters.  see
% section \ref{sec:howto:creoleconfig} for details on defining the parameters of
% a resource.
%
% It is now easier to run a corpus pipeline over a single document in the GATE
% GUI -- documents now provide a right-click menu item to create a singleton
% corpus containing just this document.  See section \ref{sec:howto:loadlr} for
% details.
%
% %%%%%%%%%%%%%%%%%%%%%%%%%%%%%%%%%%%%%%%%
% \sect{August 2006}
%
% GATE now (as of subversion revision 7592) requires Java 5.0 to build and run.
%
% Java 5.0 syntax is now available on the right hand side of JAPE rules with the
% default Eclipse compiler.  See section~\ref{sec:japeimpl:javacompiler} for
% details.
%
% A couple of helper classes have been added to assist in using GATE within a
% \htlink{http://www.springframework.org}{Spring} application.
% Section~\ref{sec:howto:spring} explains the details.
%
% %%%%%%%%%%%%%%%%%%%%%%%%%%%%%%%%%%%%%%%%
% \sect{July 2006}
%
% (Feature Request \htlink{https://sourceforge.net/support/tracker.php?aid=1446642}{\#1446642})
% After several requests, a mechanism has been added to allow overriding of
% GATE's document format detection routine.  A new creation-time parameter
% {\tt mimeType} has been added to the standard document implementation, which
% forces a document to be interpreted as a specific MIME type and prevents the
% usual detection based on file name extension and other information.  See
% section~\ref{subsect:DetectingTheReader} for details.
%
% \subsect{Optimization of the GATE Morphological Analyzer}
% A new FSM based, although with minor alteration to the basic FSM algorithm,
% has been implemented to optimize the GATE Morphological Analyser. The previous
% profiling figures show that the morpher when integrated with ANNIE application
% used to take upto 60\% of the overall processing time.  The optimized version
% only takes 7.6\% of the total processing time. See
% section~\ref{sec:misc-creole:morpher} for more details on the morpher.
%
% %%%%%%%%%%%%%%%%%%%%%%%%%%%%%%%%%%%%%%%%
% \sect{June 2006}
%
% The GATE source code has moved from CVS into Subversion.  See
% section~\ref{sec:howto:svn} for details of how to check out the code from the
% new repository.
%
% Plugins can now provide custom icons for their PRs and LRs in the plugin JAR
% file.  See section \ref{sec:howto:creoleconfig} for details.
%
% The {\tt Buchart} plugin has been removed.  It is superseded by SUPPLE, and
% instructions on how to upgrade your applications from Buchart to SUPPLE are
% given in section~\ref{sec:parsers:supple}.  The probability finder plugin
% has also been removed, as it is no longer maintained.
%
% The Chemistry tagger (section~\ref{sec:parsers:chemistrytagger}) has been
% updated with a number of bugfixes and improvements.
%
% %%%%%%%%%%%%%%%%%%%%%%%%%%%%%%%%%%%%%%%%
% \sect{April 2006 (after 3.1)}
%
% Osamu Masutani from Denso IT Laboratory in Tokyo has contributed a GATE wrapper
% for the Sen morphological analyser for Japanese.  See
% \htlinkplain{http://gate.ac.uk/gate/doc/plugins.html} for details.
%
% The bootstrap wizard now creates a basic plugin that builds with Ant.  Since a
% Unix-style make command is no longer required this means that the generated
% plugin will build on Windows without needing Cygwin or MinGW.
%
%%%%%%%%%%%%%%%%%%%%%%%%%%%%%%%%%%%%%%%%
\sect{Version 3.1 (April 2006)}

\subsect{Major New Features}

\subsubsect{Support for UIMA}
UIMA (\htlinkplain{http://www.research.ibm.com/UIMA/}) is a language
processing framework developed by IBM.  UIMA and GATE share some
functionality but are complementary in most respects. GATE now provides an
interoperability layer to allow UIMA applications to include GATE components
in their processing and vice-versa.  For full information, see \Chapthing
\ref{chap:uima}.

\subsubsect{New Ontology API} The ontology layer has been rewritten in order to
provide an abstraction layer between the model representation and the tools used
for input and output of the various representation formats. An implementation
that uses Jena 2 (\htlinkplain{http://jena.sourceforge.net/ontology}) for
reading and writing OWL and RDF(S) is provided.

\subsubsect{Ontotext Japec Compiler}
Japec is a compiler for JAPE grammars developed by Ontotext Lab.  It has
some limitations compared to the standard JAPE transducer implementation,
but can run JAPE grammars up to five times as fast.  By default, GATE still
uses the stable JAPE implementation, but if you want to experiment with
Japec, see Section \ref{sec:misc-creole:japec}.

\subsect{Other New Features and Improvements}

\begin{itemize}
\item Addition of a new JAPE matching style `all'. This is similar to Brill,
  but once all rules from a given start point have matched, the matching will
  continue from the next offset to the current one, rather than from the
  position in the document where the longest match finishes. More details can
  be found in Section \ref{sec:jape:priority}.
\item Limited support for loading PDF and Microsoft Word document formats.
  Only the text is extracted from the documents, no formatting information is
  preserved.
\item The Buchart parser has been deprecated and replaced by a new plugin
  called SUPPLE - the Sheffield University Prolog Parser for Language
  Engineering.  Full details, including information on how to move your
  application from Buchart to SUPPLE, is in Section
  \ref{sec:parsers:supple}.
\item The Hepple POS Tagger is now open-source.  The source code has been
  included in the GATE Developer/Embedded distribution, under
  src/hepple/postag. More information about the POS Tagger can be
  found in Section \ref{sec:annie:tagger}.
\item Minipar is now supported on Windows.  \textit{minipar-windows.exe}, a
  modified version of \textit{pdemo.cpp} is added under the
  gate/plugins/Parser\_Minipar directory to allow users to run Minipar on
  windows platform. While using Minipar on Windows, this binary should be
  provided as a value for \textit{miniparBinary} parameter.  (The Minipar plugin
  has been subsequently retired.)
  %For full information on Minipar
  %in GATE, see Section \ref{sec:parsers:minipar}.
  % This \ref is dead (Minipar removed in 2016)
\item The XmlGateFormat writer(Save As Xml from GATE Developer GUI, gate.Document.toXml()
  from GATE Embedded API) and reader have been modified to write and read GATE
  annotation IDs. For backward compatibility reasons the old reader
  has been kept. This change fixes a bug which manifested in the
  following situation: If a GATE document had annotations carrying
  features of which values were numbers representing other GATE
  annotation IDs, after a save and a reload of the document to and
  from XML, the former values of the features could have become
  invalid by pointing to other annotations. By saving and restoring
  the GATE annotation ID, the former consistency of the GATE document
  is maintained. For more information, see
  Section \ref{sec:corpora:saveasxml}.
\item The NP chunker and chemistry tagger plugins have been updated.  Mark A.
  Greenwood has relicenced them under the LGPL, so their source code has been
  moved into the GATE Developer/Embedded distribution.  See Sections
  \ref{sec:parsers:npchunker} and \ref{sec:parsers:chemistrytagger} for
  details.
\item The Tree Tagger wrapper has been updated with an option to be less strict
  when characters that cannot be represented in the tagger's encoding are
  encountered in the document. 
  % Details are in Section \ref{sec:parsers:treetagger}.
\item JAPE Transducers can be serialized into binary files.  The option to load
  serialized version of JAPE Transducer (an init-time parameter
  \textit{binaryGrammarURL}) is also implemented which can be used as an
  alternative to the parameter \textit{grammarURL}. More information can be
  found in Section \ref{sec:jape:serialization}.
\item On Mac OS, GATE Developer now behaves more `naturally'.  The application menu items
  and keyboard shortcuts for {\it About} and {\it Preferences} now do
  what you would expect, and exiting GATE Developer with command-Q or the {\it
  Quit} menu item properly saves your options and current session.
\item Updated versions of Weka(3.4.6) and Maxent(2.4.0).
\item Optimisation in {\it gate.creole.ml}: the conversion of AnnotationSet
  into ML examples is now faster.
\item It is now possible to create your own implementation of {\tt Annotation},
  and have GATE use this instead of the default
  implementation.  See {\tt AnnotationFactory} and {\tt
  AnnotationSetImpl} in the {\tt gate.annotation} package for details.
\end{itemize}

\subsect{Bug Fixes}

\begin{itemize}
\item The Tree Tagger wrapper has been updated in order to run under Windows.
% See \ref{sec:parsers:treetagger}.
\item The SUPPLE parser has been made more user-friendly.  It now produces more
helpful error messages if things go wrong.  Note that you will need to update
any saved applications that include SUPPLE to work with this version - see
Section \ref{sec:parsers:supple} for details.
\item Miscellaneous fixes in the Ontotext JapeC compiler.
\item Optimization : the creation of a Document is much faster.
\item Google plugin: The optional pagesToExclude parameter was causing a
  NullPointerException when left empty at run time. Full details about the
  plugin functionality can be found in Section \ref{sec:misc-creole:google}.
\item Minipar, SUPPLE, TreeTagger: These plugins that call external processes
  have been fixed to cope better with path names that contain spaces.  Note that
  some of the external tools themselves still have problems handling spaces in
  file names, but these are beyond our control to fix.  If you want to use any
  of these plugins, be sure to read the documentation to see if they have any
  such restrictions.  (The Minipar plugin has been subsequently retired.)
\item When using a non-default location for GATE configuration files, the
  configuration data is saved back to the correct location when GATE exits.
  Previously the default locations were always used.
\item Jape Debugger:  ConcurrentModificationException in JAPE debugger.
  The JAPE debugger was generating a ConcurrentModificationException during an
  attempt to run ANNIE. There is no exception when running without the debugger
  enabled. As result of fixing one unnecessary and incorrect callback to
  debugger was removed from SinglePhaseTransducer class.
\item Plus many other small bugfixes...
\end{itemize}

%%%%%%%%%%%%%%%%%%%%%%%%%%%%%%%%%%%%%%%%%%%%

\sect{January 2005}
Release of version 3.

New plugins for processing in various languages (see
\ref{sec:misc-creole:language-plugins}). These are not full IE systems but are
designed as starting points for further development (French, German, Spanish,
etc.), or as sample or toy applications (Cebuano, Hindi, etc.).

Other new plugins:
\begin{itemize}
\item Chemistry Tagger \ref{sec:parsers:chemistrytagger}
\item Montreal Transducer (since retired)
%\ref{sec:misc-creole:montrealtransducer}
\item RASP Parser \ref{sec:parsers:rasp}
\item MiniPar (since retired)
\item Buchart Parser \ref{sec:parsers:buchart}
\item MinorThird (Version 5.1: removed)
\item NP Chunker \ref{sec:parsers:npchunker}
\item Stemmer \ref{sec:parsers:stemmer}
\item TreeTagger %\ref{sec:parsers:treetagger}
\item Probability Finder %\ref{sec:misc-creole:probability}
\item Crawler \ref{sec:misc-creole:crawler}
\item Google PR \ref{sec:misc-creole:google}
\end{itemize}

Support for SVM Light, a support vector machine implementation, has been added
to the machine learning plugin `Learning' (see section \ref{sec:ml:svmlight}).


\sect{December 2004}

GATE no longer depends on the Sun Java compiler to run, which means it will now
work on any Java runtime environment of at least version 1.4.  JAPE grammars
are now compiled using the Eclipse JDT Java compiler by default.

A welcome side-effect of this change is that it is now much easier to integrate
GATE-based processing into web applications in Tomcat.  See
Section~\ref{sec:api:tomcat} for details.

%%%%%%%%%%%%%%%%%%%%%%%%%%%%%%%%%%%%%%%%%
\sect{September 2004}

GATE applications are now saved in XML format using the XStream library,
rather than by using native java
serialization. On loading an application, GATE
will automatically detect whether it is in the old or the new format, and so
applications in both formats can be loaded. However, older
versions of GATE will be unable to load applications saved in the XML format. (A
{\tt java.io.StreamCorruptedException: invalid stream header exception} will
occcur.) It is possible to get new versions of GATE to use the old format by
setting a flag in the source code. (See the Gate.java file for details.) This
change has been made because it allows the details of an application to be
viewed and edited in a text editor, which is sometimes easier than loading the
application into GATE.

%%%%%%%%%%%%%%%%%%%%%%%%%%%%%%%%%%%%%%%%%
\sect{Version 3 Beta 1 (August 2004)}

Version 3 incorporates a lot of new functionality and some reorganisation of
existing components.

Note that Beta 1 is feature-complete but needs further debugging (please
send us bug reports!).

Highlights include: completely rewritten document viewer/editor;
extensive ontology support; a new plugin management
system; separate .jar files and a Tomcat classloading fix; lots more CREOLE
components (and some more to come soon).

Almost all the changes are backwards-compatible; some recent classes have
been renamed (particularly the ontologies support classes) and a few events
added (see below); datastores created by version 3 will probably not read
properly in version 2.  If you have problems use the mailing list and we'll
help you fix your code!

The gorey details:
\begin{itemize}
%
\item
Anonymous CVS is now available.  See Section \ref{sec:gettingstarted:svn} for details.
%
\item
CREOLE repositories and the components they contain are now managed as
plugins. You can select the plugins the system knows about (and add new
ones) by going to `Manage CREOLE Plugins' on the file menu.
%
\item
The {\tt gate.jar} file no longer contains all the subsidiary libraries and
CREOLE component resources. This makes it easier to replace library versions
and/or not load them when not required (libraries used by CREOLE builtins
will now not be loaded unless you ask for them from the plugins manager
console).
%
\item
ANNIE and other bundled components now have their resource files (e.g.
pattern files, gazetteer lists) in a separate directory in the distribution
-- {\tt gate/plugins}.
%
\item
Some testing with Sun's JDK 1.5 pre-releases has been done and no problems
reported.
%
\item
The {\tt gate://} URL system used to load CREOLE and ANNIE resources in past
releases is no longer needed. This means that loading in systems like Tomcat
is now much easier.
%
\item
MAC OS X is now properly supported by the installed and the runtime.
%
\item
An Ontology-based Corpus Annotation Tool (OCAT) has been implemented as a
plugin. Documentation of its functionality is in Section
\ref{sec:ontologies:ocat}.
%
\item
The NLG Lexical tools from the MIAKT project have now been released.
% See documentation in Section \ref{sec:misc-creole:nlg}.
%
\item
The Features viewer/editor has been completely updated -- see Section \ref{sec:developer:edit} for details.
%
\item
The Document editor has been completely rewritten
-- see Section \ref{sec:developer:documents} for more information.
%
\item
The datastore viewer is now a full-size VR -- see Section
\ref{sec:developer:datastores} for more information.
%
\end{itemize}
%


%%%%%%%%%%%%%%%%%%%%%%%%%%%%%%%%%%%%%%%%%
\sect{July 2004}

GATE documents now fire events when the document content is edited.  This
was added in order to support the new facility of editing documents from the
GUI. This change will break backwards compatibility by requiring all
DocumentListener implementations to implement a new method: \\
{\tt public void contentEdited(DocumentEvent e);}


%%%%%%%%%%%%%%%%%%%%%%%%%%%%%%%%%%%%%%%%%
\sect{June 2004}

A new algorithm has been implemented for the AnnotationDiff function. A new,
more usable, GUI is included, and an `Export to HTML' option added. More
details about the AnnotationDiff tool are in Section \ref{sec:eval:adiff}.

A new build process, based on ANT (http://ant.apache.org/) is now
available. The old build process, based on make, is now
unsupported. See Section \ref{sec:gettingstarted:build} for details of the new
build process.

A Jape Debugger from Ontos AG has been integrated. You can turn
integration ON with command line option `-j'. If you run GATE Developer with
this option, the new menu item for Jape Debugger GUI will appear in the
Tools menu. The default value of integration is OFF. We are currently
awaiting documentation for this.

NOTE! Keep in mind there is ClassCastException if you try to debug
ConditionalCorpusPipeline. Jape Debugger is designed for Corpus Pipeline
only. The Ontos code needs to be changed to allow debugging of
ConditionalCorpusPipeline.


%%%%%%%%%%%%%%%%%%%%%%%%%%%%%%%%%%%%%%%%%
\sect{April 2004}

There are now two alternative strategies for ontology-aware grammar
transduction:
%
\begin{itemize}
%
\item using the [ontology] feature both in grammars and annotations; with the
default Transducer.
%
\item using the ontology aware transducer -- passing an ontology LR to a new
subsume method in the SimpleFeatureMapImpl. the latter strategy does not
check for ontology features (this will make the writing of grammars
easier -- no need to specify ontology).
%
\end{itemize}

The changes are in:
%
\begin{itemize}
%
\item SinglePhaseTransducer (always call subsume with ontology -- if null
then the ordinary subsumption takes place)
%
\item SimpleFeatureMapImpl (new subsume method using an ontology LR)
%
\end{itemize}
%
More information about the ontology-aware transducer can be found in Section
\ref{sec:ontologies:ontology-aware-jape}.

A morphological analyser PR has been added. This finds the root and
affix values of a token and adds them as features to that token.

A flexible gazetteer PR has been added. This performs lookup over a
document based on the values of an arbitrary feature of an arbitrary
annotation type, by using an externally provided gazetteer. See
\ref{sec:gazetteers:flexgazetteer} for details.


%%%%%%%%%%%%%%%%%%%%%%%%%%%%%%%%%%%%%%%%%
\sect{March 2004}

Support was added for the MAXENT machine learning library. (See
\ref{sec:ml:maxent} for details.)


%%%%%%%%%%%%%%%%%%%%%%%%%%%%%%%%%%%%%%%%%
\sect{Version 2.2 -- August 2003}

Note that GATE 2.2 works with JDK 1.4.0 or above. Version 1.4.2 is
recommended, and is the one included with the latest installers.

GATE has been adapted to work with Postgres 7.3. The compatibility with
PostgreSQL 7.2 has been preserved. 

Note that as of Version 5.1 PostgreSQL is no longer supported.

%See \ref{chap:oraclepostgre} for more
%details.

New library version -- Lucene 1.3 (rc1)

A bug in gate.util.Javac has been fixed in order to account for situations
when String literals require an encoding different from the platform
default.

Temporary .java files used to compile JAPE RHS actions are now saved using
UTF-8 and the `-encoding UTF-8' option is passed to the javac compiler.

A custom tools.jar is no longer necessary

Minor changes have been made to the look and feel of GATE Developer to
improve its appearance with JDK 1.4.2

%%%%%%%%%%%%%%%%%%%%%%%%%%%%%%%%%%%%%%%%%
\sect{Version 2.1 --  February 2003}

Integration of Machine Learning PR and WEKA wrapper (see Section
\ref{sec:ml:machine-learning-pr}).

Addition of DAML+OIL exporter.

Integration of WordNet (see Section \ref{sec:misc-creole:wn}).

The syntax tree viewer has been updated to fix some bugs.


%%%%%%%%%%%%%%%%%%%%%%%%%%%%%%%%%%%%%%%%%
\sect{June 2002}

Conditional versions of the controllers are now available (see Section
\ref{sec:developer:cond}). These allow processing resources to be run
conditionally on document features.

PostgreSQL Datastores are now supported.
% (see Section
%\ref{chap:oraclepostgre}). 

These store data into a PostgreSQL RDBMS.

(As of Version 5.1 PostgreSQL is no longer supported.)

Addition of OntoGazetteer (see Section \ref{sec:gazetteers:ontogaz}), an
interface which makes ontologies visible within GATE Developer, and
supports basic methods for hierarchy management and traversal.

Integration of Prot\'eg\'e, so that people
with developed Prot\'eg\'e ontologies can use them within GATE.

Addition of IR facilities in GATE (see Section \ref{sec:misc-creole:ir}).

Modification of the corpus benchmark tool (see Section
\ref{sec:eval:howto-benchmarktool}), which now takes an application as a
parameter.

See also \htlink{http://gate.ac.uk/gate/doc/bugs.html} for details of other
recent bug fixes.
