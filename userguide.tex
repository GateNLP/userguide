% html: Beginning of file: `userguide.html'

\section{Introduction}

\label{f0}
GATE, a General Architecture for Text Engineering
\cite{Cun96b,Cun97a,Cun98,Cun99a,Cun00a,Cun01b},
is a software architecture for Language Engineering \cite{Cun99b}.
More specifically, it is three things: an {\bf architecture}; 
a {\bf framework}; a {\bf development environment}.

By architecture we mean an abstract description of how a language processing
system may usefully be constructed, the types of component typically
used and so on.
By framework we mean an object-oriented class library that implements the
architecture and provides a range of services that are useable in a variety
of application contexts. One such application is a development environment
built on top of the framework. The development environment is analogous to
systems like Mathematica for Mathematicians, or JBuilder for Java
programmers: it provides a convenient graphical environment for research and
development of language processing software.

Version 1 of GATE was released in 1996. It was written in C++ and Tcl, has
been licensed by several hundred organisations, and used in a wide range of
language analysis contexts including Information Extraction (IE -
\cite{Gai98a,Cun99c}) in English, Greek, Spanish, Swedish, German, Italian and
French.

Version 2 of GATE was released in Spring 2001. It is written in Java,
and is available as open source free software under the GNU licence
at http://gate.ac.uk/\footnote{See URL http://gate.ac.uk/}.

For more details about human language processing in general see
Sheffield NLP group\footnote{See URL http://www.dcs.shef.ac.uk/nlp/}
or this paper on
Language
Engineering\footnote{See URL http://www.dcs.shef.ac.uk/.hamish/LeIntro.html}.
For more details about Information Extraction see this
User Guide to IE\footnote{See URL http://www.dcs.shef.ac.uk/.hamish/IE/}
or the Sheffield
IE pages\footnote{See URL http://www.dcs.shef.ac.uk/nlp/extraction}.

The rest of this section gives a general introduction to the system.
The rest of the document then covers:
\begin{itemize}
\item 
how to use the
development environment (cf.\ Section~\ref{f0:the-development-environment})
\item 
how to use the framework (cf.\ Section~\ref{f0:the-framework})
\item 
the design principles (cf.\ Section~\ref{f0:design}) of the architecture and framework.
\end{itemize}
%%%%%%%%%%%%%%%%%%%%%%%%%%%%%%%%%%%%%%%%%%%%%%%%%%%%%%%%%%%%%%%%%%%%%%%%%

\subsection{Architectural principles}


A central idea behind the GATE architecture is that there should be
no requirement for users to commit to any particular theory of
language processing: the architecture strives to be non-prescriptive and
theory-neutral. Therefore there is a very general model of components and the
data structures they share. This is, of course, both a strength and a
weakness.

(Almost) everything in GATE is a component. Components are
reusable software chunks with well-defined interfaces
that are conceptually separate from GATE itself.
All component sets are user-extensible and together are called CREOLE - a
Collection of REusable Objects for Language Engineering.
%%%%%%%%%%%%%%%%%%%%%%%%%%%%%%%%%%%%%%%%%%%%%%%%%%%%%%%%%%%%%%%%%%%%%%%%%

\subsection{GATE-based development}


The framework is a backplane into which plug CREOLE components.
The user gives the system a list of URLs to search when it starts up,
and components at those locations are loaded by the system. (To be precise
only their configuration data is loaded to begin with; the actual classes
are loaded when the user requests the instantiation of a resource.)

The backplane performs these functions:
\begin{itemize}
\item 
component discovery, bootstrapping, loading and reloading;
\item 
native data structures for common information types;
\item 
generalised data storage and process execution.
\end{itemize}

A set of components plus the framework is a deployment unit which can be
embedded in another application.

The key task of the 
development environment is to facilitate constructing components.
%%%%%%%%%%%%%%%%%%%%%%%%%%%%%%%%%%%%%%%%%%%%%%%%%%%%%%%%%%%%%%%%%%%%%%%%%

\subsection{Component types}


GATE components are one of three types of specialised Java Beans:

{\bf Resource:}\newline
The top-level interface, which describes all components. What all 
components share in common is that they can be loaded at runtime,
and that the set of components is extendable by clients. They
have Features, which are represented externally to the system as
{\tt{}"{}}meta-data{\tt{}"{}} in a format such as RDF, plain XML, or Java properties.
Resources should probably all be Java beans.

{\bf ProcessingResource:}\newline
Is a resource that is runnable, may be invoked remotely (via RMI),
and lives in class files. In order to load a PR the system just
needs to know where to find the class or jar files (which will 
also include the metadata).

{\bf LanguageResource:}\newline
Is a resource that consists of data, accessed via a Java abstraction
layer. They live in relational databases.

{\bf VisualResource:}\newline
Is a visual Java bean, component of GUIs, including of the main GATE 
gui.  Like PRs they live in .class or .jar files.
%%%%%%%%%%%%%%%%%%%%%%%%%%%%%%%%%%%%%%%%%%%%%%%%%%%%%%%%%%%%%%%%%%%%%%%%%

\subsection{Bits and pieces}

There are built in components for common processing and data visualisation
tasks.
There is a finite state transduction language operating over
annotations on text, called JAPE, a Jolly Advanced Pattern Eater. JAPE 
is based on Doug Appelt's TextPro language.
There is automated measurement: precision, recall, diff over annotations
on text.
Support for documents in XML, SGML, HTML, RTF, email, PDF as well as
support for documents from both Microsoft Office and OpenOffice.
Full Unicode support including editing in a number of languages (not
supported by native JDK; thanks to Mark Leisher for help with this).
%%%%%%%%%%%%%%%%%%%%%%%%%%%%%%%%%%%%%%%%%%%%%%%%%%%%%%%%%%%%%%%%%%%%%%%%%
%%%%%%%%%%%%%%%%%%%%%%%%%%%%%%%%%%%%%%%%%%%%%%%%%%%%%%%%%%%%%%%%%%%%%%%%%

\section{Development Environment}

\label{f0:the-development-environment}
The GATE development environment is designed to facilitate the creation,
development and testing of components for language processing R\&D.
We describe here how to perform these tasks, and
how to use the tools for named entity recognition and results evaluation.

There are 6 main steps to using GATE.
\begin{enumerate}
\item Bootstrap the basic software for new resources
\item Instantiate the desired language resource(s)
\item Instantiate appropriate processing resource(s)
\item Create and run an application (a set of components)
\item View the results of the application
\item Apply further tools, e.g. evaluation of the results.
\end{enumerate}
%%%%%%%%%%%%%%%%%%%%%%%%%%%%%%%%%%%%%%%%%%%%%%%%%%%%%%%%%%%%%%%%%%%%%%%%%

\subsection{Bootstrapping New Resources}

GATE components may be implemented by a variety of programming languages and
databases, but in each case they are represented to the system as a Java
class. This class may do nothing other than call the underlying program, or
provide an access layer to a database; on the other hand 
it may implement the whole component.

The development environment will dump out the basic form of a new resource
Java class to disk for you: select `BootStrap Wizard' from the `Tools' menu.
%%%%%%%%%%%%%%%%%%%%%%%%%%%%%%%%%%%%%%%%%%%%%%%%%%%%%%%%%%%%%%%%%%%%%%%%%

\subsection{Loading Language Resources}


Load a language resource by right clicking on {\tt{}"{}}Language Resources{\tt{}"{}} and
selecting {\tt{}"{}}Create Language Resource{\tt{}"{}}. Select {\tt{}"{}}GATE document{\tt{}"{}} and a
pop-up window will appear. Choose a name for the resource, and select
a file or url as the value of {\tt{}"{}}sourceUrl{\tt{}"{}}. Note that double clicking
in the {\tt{}"{}}values{\tt{}"{}} box brings up a tree structure to enable selection of
documents.
 Make any
changes to default settings as required (e.g. encoding type used) and
click OK. The document name and icon should appear in the left hand
pane, and can be viewed in the main window by double clicking on the
icon. The right hand pane enables annotations to be selected and
viewed. At this stage, the only annotations displayed will be those
which are produced as a result of the text structure analysis which
transforms a text into a GATE document, e.g. xml or html
tags. Additional language resources can be loaded by repeating the
procedure.
%%%%%%%%%%%%%%%%%%%%%%%%%%%%%%%%%%%%%%%%%%%%%%%%%%%%%%%%%%%%%%%%%%%%%%%%%

\subsection{Loading Processing Resources}


Right click on `Processing Resources' and select `Create processing
resource'. Select the type of resource (e.g. tokeniser, gazetteer,
etc.) from the list of options. In the pop-up box, choose a name for
the resource, and either select the default value for the resource, or
select a new one. Select any other values as appropriate
(e.g. encoding). In the case where it is not indicated that a value is
`required', the box may be left blank and the system default will be
used. When all necessary boxes are completed, click `OK'. An icon should appear
under `Processing Resources' in the left hand pane. Note that it may
take a few seconds for the resource to be loaded. Repeat this
procedure until all necessary resources have been loaded.

The following individual processing resources are currently available: 
\begin{itemize}
\item Tokeniser
\item Sentence Splitter
\item Hepple POS Tagger
\item Gazetteer
\item JAPE Transducer
\item Namematcher
\end{itemize}

Note that the resources must be run in this order (with the exception
of the gazetteer, which, can be run at any point preceding the JAPE
transducer). If not using all the PRs, some caution is advised, as
certain modules rely on previous ones and may either not work as
expected or fail completely. For example, the JAPE transducer requires
all previous PRs for best performance, but can be run without the
tagger, splitter, and gazetteer. The tagger, on the other hand, will
fail to produce results if the splitter is not run first. 

The first 5 of these PRs are also combined into an additional
PR, the NERC (the Named Entity Recognition Component). This can be run
as one unit instead of the individual PRs. The Namematcher can
optionally be run following this. Note that while the default
annotation set used for the individual PRs is the Default set, the
default set for the NERC is the nercAS set. if the Namematcher is run
following the NERC, nercAS must therefore be specified as the name of the AnnotationSet.
%%%%%%%%%%%%%%%%%%%%%%%%%%%%%%%%%%%%%%%%%%%%%%%%%%%%%%%%%%%%%%%%%%%%%%%%%

\subsection{Running an Application}

Once all the resources have been loaded, an application can be created
and run. Right click on `Applications' and create a new
one. Then double click on it and the `Design' tab will appear. Here you can
select the resources needed to run the application (these may
not be necessarily be all those which have been loaded). Transfer the
necessary components from the set of `available components' displayed on
the right hand side of the main window to the set of `used components'
on the left, by selecting each component and clicking on the left and
right arrows. Ensure that the components are listed on the left in the
correct order for processing (starting from the top). If not, select a
component and move it up or down the list using the up/down arrows at the
bottom of the pane. Once this is complete, move to the left hand pane,
select the language resource to be used (using a left click), and
finally right click on the application and select `Run'.


%%%%%%%%%%%%%%%%%%%%%%%%%%%%%%%%%%%%%%%%%%%%%%%%%%%%%%%%%%%%%%%%%%%%%%%%%
\subsection{Viewing the Results}
%%%%%%%%%%%%%%%%%%%%%%%%%%%%%%%%%%%%%%%%%%%%%%%%%%%%%%%%%%%%%%%%%%%%%%%%%

Once the system has run, open the document to be
viewed with a double click on the icon in the left window (if it is
not already displayed). Note that it may take a few seconds for the
text to be displayed if it is long.
The annotation types belonging to each annotation set are displayed to
the right of the text. Select the annotation types to be viewed by
clicking on the appropriate checkbox(es). The text segments
corresponding to these annotations will be highlighted in the main text window.  

Descriptions of the annotations are simultaneously displayed in the
bottom pane. These lists can be sorted in ascending and descending
order by any column, by clicking on the corresponding column
heading. An arrow will appear indicating the direction of the sorting.
Clicking on an entry in the table will also highlight the respective
matching text portion.

Right clicking on some part of the text in the main window will bring
up a box containing a list of the annotations associated with
it. Selecting one of these annotation types will highlight the
relevant annotation description in the lower pane, if present. If not
present (because the corresponding annotation on the right hand pane
has not been selected), this annotation on the right will then be
automatically selected and all relevant text in the main window will be
appropriately highlighted.

Although there is no cursor displayed in the various windows, they 
can all be scrolled using the keyboard arrows, as well as by using the
scrollbars.

At any time, the main viewer can also be used to display other
information, such as Messages, by clicking on the header
at the top of the main window. If an error occurs in processing, the
messages tab will flash red, and an additional popup error message
may also occur.


%%%%%%%%%%%%%%%%%%%%%%%%%%%%%%%%%%%%%%%%%%%%%%%%%%%%%%%%%%%%%%%%%%%%%%%%%
\subsection{Adding Annotations}
%%%%%%%%%%%%%%%%%%%%%%%%%%%%%%%%%%%%%%%%%%%%%%%%%%%%%%%%%%%%%%%%%%%%%%%%%

In order to be able to add/edit annotations in GATE, the relevant
Annotation Schemata must first be loaded. If only the Default
annotation types are required, this is done automatically. If other
types of annotation are required, an Annotation Schema (which is an
xml file) must be selected from the Language Resources. This process
must be repeated for each annotation type. It does not, however, need
to be repeated for each document.

Once the Annotation Schemata have been loaded, the annotation types that
have a Schema present inside GATE can be added or edited. 
To add a new annotation, select the text, right click, and select an
annotation set. Then select the name of the
annotation to be created. If the annotation can have features, another
window will automatically open. Select a feature from the list of
possible features, and click the arrow to transfer it to the list of
current features. The feature values can be edited by clicking on
them. The new annotation will be added to the annotation set, and will
appear in the annotation description table.

An existing annotation can be modified by selecting it from the table
and double clicking on it to bring up the features
window. If, however, the schema has no features defined,
then the selected annotation cannot be edited (since there are no
features to edit). All that can be done is to add or delete the
annotation. An annotation can be deleted by selecting it from the
table, right clicking on it, and selecting Delete.

An annotated text can be saved in a datastore. Create a datastore  %
by right clicking on `Datastores' and selecting the option
`Create Datastore'. Select
`Serial DataStore' as the datastore type. Create a directory to be
used as the datastore (note that the datastore is a directory and
not a file). Save the text to the datastore by right clicking on the
document name and selecting the `Save to' option (giving the name of
the datastore created earlier).

A modified file can also be saved as an xml file, by right clicking on
the document icon in the left window and choosing the
option `Save as XML'. Once saved, it can be reloaded in the usual way,
by selecting `New Language Resource' from the `File' menu or by
selecting `New' and then `GATE Document' from the Language Resources icon.

To load a document from a datastore, do not try to load it as a
language resource. Instead, open the datastore, and double click on
it to view its contents. Double click on the relevant file to display
the text. Once the text has appeared in the main window, it can be
treated in the same way as any other document.


%%%%%%%%%%%%%%%%%%%%%%%%%%%%%%%%%%%%%%%%%%%%%%%%%%%%%%%%%%%%%%%%%%%%%%%%%
\subsection{The Evaluation Tool (Annotation Diff)}
%%%%%%%%%%%%%%%%%%%%%%%%%%%%%%%%%%%%%%%%%%%%%%%%%%%%%%%%%%%%%%%%%%%%%%%%%

The annotation tool is activated by selecting it from the 
`Tools' menu at the top of the window. It will appear in a new
window. Select the key and response documents to be used (note that
both must have been previously loaded into the system), the annotation
sets to be used for each, and the annotation
type to be evaluated. If false positives are to be measured, select
the annotation type (and relevant annotation set) to be used as the
denominator (normally, Token or Sentence). The weight for the
F-Measure can also be changed - by default it is set to 0.5 (i.e. to
give precision and recall equal weight). Finally, click
on `Evaluate' to display the results. Note that the window may need to
be resized manually, by dragging the window edges or internal bars
as appropriate). 

In the main window, the key and response annotations will be
displayed. They can be sorted by any category by clicking on the
relevant column header. The key and response annotations will be aligned if
their indices are identical, and are colour coded according to the legend
displayed. 


%%%%%%%%%%%%%%%%%%%%%%%%%%%%%%%%%%%%%%%%%%%%%%%%%%%%%%%%%%%%%%%%%%%%%%%%%
\subsubsection{Evaluation metrics}
%%%%%%%%%%%%%%%%%%%%%%%%%%%%%%%%%%%%%%%%%%%%%%%%%%%%%%%%%%%%%%%%%%%%%%%%%

Precision, recall, F-measure and
false positives are also displayed below the annotation tables, each
according to 3 criteria - strict, lenient and average. The reason for
these 3 criteria is to deal with partially correct responses in
different ways.
\begin{itemize}
\item The Strict measure considers all partially correct
responses as incorrect (spurious).
\item The Lenient measure considers all partially correct
responses as correct.
\item The Average measure allocates a half weight to
partially correct responses (i.e. it takes the average of strict and
lenient).
\end{itemize}


%%%%%%%%%%%%%%%%%%%%%%%%%%%%%%%%%%%%%%%%%%%%%%%%%%%%%%%%%%%%%%%%%%%%%%%%%
\section{The Framework}
%%%%%%%%%%%%%%%%%%%%%%%%%%%%%%%%%%%%%%%%%%%%%%%%%%%%%%%%%%%%%%%%%%%%%%%%%

\label{f0:the-framework}
This section gives documentation for the framework; see also the 
JavaDoc pages. 
The
CookBook class gives example code for
using the GATE API.

The GATE framework models language processing components and the language
data they operate on as Resources
Resources. The set of all resources is
known as CREOLE, a Collection or REusable Objects for Language Engineering.

The terms component, resource and CREOLE object are largely synonymous.
%%%%%%%%%%%%%%%%%%%%%%%%%%%%%%%%%%%%%%%%%%%%%%%%%%%%%%%%%%%%%%%%%%%%%%%%%

\subsection{The Processing Model}


Any resource whose primary characteristics are algorithmic, such as parsers,
generators and so on, is modelled as a 
ProcessingResource (PR).
A PR is a Resource that implements the Java Runnable interface.
%%%%%%%%%%%%%%%%%%%%%%%%%%%%%%%%%%%%%%%%%%%%%%%%%%%%%%%%%%%%%%%%%%%%%%%%%

\subsection{The Visualisation Model}


Resources whose task is to display and edit other resources are modelled as 
VisualResources (VRs).
%%%%%%%%%%%%%%%%%%%%%%%%%%%%%%%%%%%%%%%%%%%%%%%%%%%%%%%%%%%%%%%%%%%%%%%%%

\subsection{The Corpus Model}


A Corpus
in GATE is a Java Set whose members are 
Documents.
Both Corpora and Documents are types of 
LanguageResource (LR); all
LRs have a 
FeatureMap
(a Java Map) associated with them that
stored attribute/value information about the resource. FeatureMaps are also
used to associate arbitrary information with ranges of documents (e.g. pieces
of text) via the annotation model (see below).

Documents have a
DocumentContent which is
a text at present (future versions may add support for audiovisual content)
and one or more
AnnotationSets which are Java
Sets.
%%%%%%%%%%%%%%%%%%%%%%%%%%%%%%%%%%%%%%%%%%%%%%%%%%%%%%%%%%%%%%%%%%%%%%%%%

\subsection{The Annotation Model}


Annotations are organised in graphs, which are modelled as 
Java sets of 
Annotation.
Annotations may be considered as the arcs in the graph; they have a start 
Node and an end Node, an ID, a type and a
FeatureMap. Nodes have pointers into the sources document, e.g. character
offsets.

The rest of this section
shows some simple examples of annotated documents.

This material is adapted from \cite{Gri96b}, the TIPSTER Architecture Design
document upon which GATE version 1 was based. Version 2 has a similar model,
although annotations are now graphs, and instead of multiple spans per
annotation each annotation now has a single start/end node pair. The current
model is largely compatible with \cite{Bir99}, and roughly isomorphic with
`stand-off markup' as latterly adopted by the SGML/XML community.

Each example is shown in the form of a table. At the top of the table is the
document being annotated; immediately below the line with the document is a
ruler showing the position (byte offset) of each character.
({\bf NOTE:} the ruler doesn't scale very well in HTML; for a better picture 
see the original
TIPSTER Architecture Design Document\footnote{See URL http://www.itl.nist.gov/div894/894.02/related.projects/tipster/docs/arch31.doc}.)
Underneath this
appear the annotations, one annotation per line. For each annotation is shown
its Id, Type, Span (start/end offsets derived from the start/end nodes),
and Features. Integers are used as the annotation Ids.  The
features are shown in the form name = value.

The first example shows a single sentence and the result of three
annotation procedures: tokenization with part-of-speech assignment, name
recognition, and sentence boundary recognition. Each token has a single
feature, its part of speech (pos), using the tag set from the University of
Pennsylvania Tree Bank; each name also has a single feature, indicating the
type of name: person, company, etc.

{\bf Table 1. Result of annotation on a single sentence}TextCyndi savored the
            soup.\mbox{$|$}0...\mbox{$|$}5...\mbox{$|$}10..\mbox{$|$}15..\mbox{$|$}20AnnotationsIdTypeSpan
            StartSpan
            EndFeatures1token05pos=NP2token613pos=VBD3token1417pos=DT4token1822pos=NN5token2223~6name05name\_type=person7sentence023~

Annotations will typically be organized to describe a hierarchical
      decomposition of a text. A simple illustration would be the decomposition of a
      sentence into tokens. A more complex case would be a full syntactic analysis,
      in which a sentence is decomposed into a noun phrase and a verb phrase, a verb
      phrase into a verb and its complement, etc. down to the level of individual
      tokens. Such decompositions can be represented by annotations on nested sets of
      spans. Both of these are illustrated in the second example, which is an
      elaboration of our first example to include parse information. Each
      non-terminal node in the parse tree is represented by an annotation of type
      parse.

{\bf Table 2. Result of annotations including parse information}TextCyndi savored the
            soup.\mbox{$|$}0...\mbox{$|$}5...\mbox{$|$}10..\mbox{$|$}15..\mbox{$|$}20AnnotationsIdTypeSpan
            StartSpan
            EndFeatures1token05pos=NP2token613pos=VBD3token1417pos=DT4token1822pos=NN5token2223~6name05name\_type=person7sentence023constituents=\mbox{$[$}1\mbox{$]$},\mbox{$[$}2\mbox{$]$},\mbox{$[$}3\mbox{$]$}.\mbox{$[$}4\mbox{$]$},\mbox{$[$}5\mbox{$]$}8parse05symbol={\tt{}"{}}NP{\tt{}"{}},constituents=
            \mbox{$[$}1\mbox{$]$}9parse1422symbol={\tt{}"{}}NP{\tt{}"{}},constituents=\mbox{$[$}3\mbox{$]$},\mbox{$[$}4\mbox{$]$}10parse622symbol={\tt{}"{}}VP{\tt{}"{}},constituents=\mbox{$[$}2\mbox{$]$},\mbox{$[$}9\mbox{$]$}11parse022symbol={\tt{}"{}}S{\tt{}"{}},constituents=\mbox{$[$}8\mbox{$]$},\mbox{$[$}10\mbox{$]$}

In most cases, the hierarchical structure could be recovered from the
      spans. However, it may be desirable to record this structure directly through a
      constituents feature whose value is a sequence of annotations representing
      the immediate constituents of the initial annotation. For the annotations of
      type parse, the constituents are either non-terminals (other annotations in the
      parse group) or tokens. For the sentence annotation, the constituents feature
      points to the constituent tokens. A reference to another annotation is
      represented in the table as {\tt{}"{}}\mbox{$[$} Annotation Id\mbox{$]$}{\tt{}"{}}; for example, {\tt{}"{}}\mbox{$[$}3\mbox{$]$}{\tt{}"{}} represents a
      reference to annotation 3. Where the value of an feature is a sequence of
      items, these items are separated by commas. No special operations are provided
      in the current architecture for manipulating constituents. At a less esoteric
      level, annotations can be used to record the overall structure of documents,
      including in particular documents which have structured headers, as is shown in
      the third example (Table 3).

{\bf Table 3. Annotation showing overall document structure}TextTo: All Barnyard Animals\mbox{$|$}0...\mbox{$|$}5...\mbox{$|$}10..\mbox{$|$}15..\mbox{$|$}20.. From: Chicken
            Little\mbox{$|$}25..\mbox{$|$}30..\mbox{$|$}35..\mbox{$|$}40..\mbox{$|$}45..  Date: November 10,1194....\mbox{$|$}50..\mbox{$|$}55..\mbox{$|$}60..\mbox{$|$}65.. Subject: Descending
            Firmament\mbox{$|$}70..\mbox{$|$}75..\mbox{$|$}80..\mbox{$|$}85..\mbox{$|$}90..\mbox{$|$}95.. Priority :
            Urgent.\mbox{$|$}100.\mbox{$|$}105.\mbox{$|$}110.\mbox{$|$}115. The sky is falling. The sky is
            falling.
            ....\mbox{$|$}120.\mbox{$|$}125.\mbox{$|$}130.\mbox{$|$}135.\mbox{$|$}140.\mbox{$|$}145.\mbox{$|$}150.AnnotationsIdTypeSpan
            StartSpan
            EndFeatures1Addressee424~2Source3145~3Date5369ddmmyy=1011944Subject7898~5Priority109115~6Body116155~7Sentence116135~8Sentence136155~

 If the Addressee, Source, ... annotations are recorded when the
      document is indexed for retrieval, it will be possible to perform retrieval
      selectively on information in particular fields. Our final example (Table 4)
      involves an annotation which effectively modifies the document. The current
      architecture does not make any specific provision for the modification of the
      original text. However, some allowance must be made for processes such as
      spelling correction. This information will be recorded as a correction
      feature on token annotations and possibly on name
      annotations:

{\bf Table 4. Annotation modifying the document}TextTopster tackles 2 terrorbytes.\mbox{$|$}0...\mbox{$|$}5...\mbox{$|$}10..\mbox{$|$}15..\mbox{$|$}20..\mbox{$|$}25..AnnotationsIdTypeSpan
            StartSpan EndFeatures1token07pos=NP
            correction=TIPSTER2token815pos=VBZ3token1617pos=CD4token1829pos=NNS correction=terabytes5token2930~%  END TEXT GENERATED WITH DOCBOOK
%%%%%%%%%%%%%%%%%%%%%%%%%%%%%%%%%%%%%%%%%%%%%%%%%%%%%%%%%%%%%%%%%%%%%%%%%
%%%%%%%%%%%%%%%%%%%%%%%%%%%%%%%%%%%%%%%%%%%%%%%%%%%%%%%%%%%%%%%%%%%%%%%%%

\section{Design}

\label{f0:design}
GATE is a backplane into which specialised Java Beans plug.
These beans are loose-coupled with respect to each other - they communicate
entirely by means of the GATE framework.
Inter-component communication is handled by model components -
LanguageResources, and events.

Components are defined by conformance to various interfaces (e.g.
LanguageResource), ensuring
separation of interface and implementation.

Distribution and parallelism (NOT fully working as yet)
is handled by controller components (and by distributing data
over HTTP and JDBC).

The reason for adding to the normal bean initialisation mech is that
LRs, PRs and VRs all have characteristic parameterisation
phases; the GATE resources/components model makes explicit these phases.
%%%%%%%%%%%%%%%%%%%%%%%%%%%%%%%%%%%%%%%%%%%%%%%%%%%%%%%%%%%%%%%%%%%%%%%%%

\subsection{Patterns}


GATE is structured around a number of what we might call principles, or
patterns, or alternatively, clever ideas stolen from better minds than
mine. These patterns are:
\begin{itemize}
\item 
modelling most things as extensible sets of components (cf.\ Section~\ref{f0:components});
\item 
separating components into
model, view, or controller (cf.\ Section~\ref{f0:mvc}) types;
\item 
hiding implementation behind interfaces (cf.\ Section~\ref{f0:interfaces}).
\end{itemize}

Four interfaces in the top-level package describe the GATE view of
components:
Resource, ProcessingResource, LanguageResource and VisualResource.
%%%%%%%%%%%%%%%%%%%%%%%%%%%%%%%%%%%%%%%%%%%%%%%%%%%%%%%%%%%%%%%%%%%%%%%%%

\subsection{Components}

\label{f0:components}%%%%%%%%%%%%%%%%%%%%%%%%%%%%%%%%%%%%%%%%%%%%%%%%%%%%%%%%%%%%%%%%%%%%%%%%%

\subsubsection{Architectural Principle}


Wherever users of the architecture may wish to extend the set of a
particular type of entity, those types should be expressed as components.

Another way to express this is to say that the architecture is based on
{\em agents}. I've avoided this in the past because of an association
between this term and the idea of bits of code moving around between
machines of their own volition. I take this to be somewhat pointless, and
probably the result of an anthropomorphic obsession with mobility as a
correlate of intelligence. If we drop this connotation, however, we can
say that GATE is an agent-based architecture. If we want to, that is.
%%%%%%%%%%%%%%%%%%%%%%%%%%%%%%%%%%%%%%%%%%%%%%%%%%%%%%%%%%%%%%%%%%%%%%%%%

\subsubsection{Framework Expression}


Many of the classes in the framework are components, by which we mean
classes that conform to an interface with certain standard properties. In
our case these properties are based on the Java Beans component
architecture, with the addition of component metadata, automated loading
and standardised storage, threading and distribution.

All components inherit from Resource, via one of:
\begin{itemize}
\item 
LanguageResource (LR) represents entities such as lexicons, corpora or
ontologies;
\item 
VisualResource (VR) represents visualisation and editing components that
participate in GUIs;
\item 
ProcessingResource (PR) represents entities that are primarily algorithmic,
such as parsers, generators or ngram modellers.
\end{itemize}
%%%%%%%%%%%%%%%%%%%%%%%%%%%%%%%%%%%%%%%%%%%%%%%%%%%%%%%%%%%%%%%%%%%%%%%%%

\subsection{Model, view, controller}

\label{f0:mvc}
According to Buschmann et al
(Pattern-Oriented Software Architecture, 1996),
the Model-View-Controller (MVC)
pattern 
\begin{quotation} 
...divides an interactive application into three components. The model
contains the core functionality and data. Views display information to the
user. Controllers handle user input. Views and controllers together comprise
the user interface. A change-propagation mechanism ensures consistency between
the user interface and the model. \mbox{$[$}p.125\mbox{$]$}\end{quotation}
A variant of MVC, the Document-View pattern, 
\begin{quotation} 
...relaxes the separation of view and controller... The View component of
Document-View combines the responsibilities of controller and view in MVC, and
implements the user interface of the system.\end{quotation}
A benefit of both arrangements is that 
\begin{quotation} 
...loose coupling of the document and view components enables multiple
simultaneous synchronized but different views of the same document.\end{quotation}

Geary (Graphic Java 2, 3rd Edtn., 1999) gives a slightly different view:
\begin{quotation} 
MVC separates applications into three types of objects:
\begin{itemize}
\item 
{\bf Models:} Maintain data and provide data accessor methods
\item 
{\bf Views:} Paint a visual representation of some or all of a model's data
\item 
{\bf Controllers:} Handle events
... By encapsulating what other architectures intertwine, MVC applications are
much more flexible and reusable than their traditional counterparts.
\end{itemize}
\mbox{$[$}pp. 71, 75\mbox{$]$}\end{quotation}
Swing, the Java user interface framework, uses
\begin{quotation} 
a specialised version of the classic MVC meant to support pluggable look and
feel instead of applications in general.
\mbox{$[$}p. 75\mbox{$]$}\end{quotation}

GATE may be regarded as an MVC architecture in two ways:
\begin{itemize}
\item 
directly, because we use the Swing toolkit for the GUIs;
\item 
by analogy, where LRs are models, VRs are views and PRs are controllers.
Of these, the latter sits least
easily with the MVC scheme, as PRs may indeed be controllers but may also
not be.
\end{itemize}
%%%%%%%%%%%%%%%%%%%%%%%%%%%%%%%%%%%%%%%%%%%%%%%%%%%%%%%%%%%%%%%%%%%%%%%%%

\subsection{Interfaces}

\label{f0:interfaces}%%%%%%%%%%%%%%%%%%%%%%%%%%%%%%%%%%%%%%%%%%%%%%%%%%%%%%%%%%%%%%%%%%%%%%%%%

\subsubsection{Architectural Principle}


The implementation of types should generally be hidden from the clients of
the architecture.
%%%%%%%%%%%%%%%%%%%%%%%%%%%%%%%%%%%%%%%%%%%%%%%%%%%%%%%%%%%%%%%%%%%%%%%%%

\subsubsection{Framework Expression}


With a few exceptions (such as for utility classes),
clients of the framework work with the {\tt gate.*} package.
This
package is mostly composed of interface definitions.
Instantiations of these interfaces are obtained via the {\tt Factory}
class.

The subsidiary packages of GATE provide the implementations of the
{\tt gate.*} interfaces that are accessed via the factory. They
themselves avoid directly constructing classes from other packages
(with a few exceptions, such as JAPE's need for unattached annotation
sets). Instead they use the factory.
%%%%%%%%%%%%%%%%%%%%%%%%%%%%%%%%%%%%%%%%%%%%%%%%%%%%%%%%%%%%%%%%%%%%%%%%%
%%%%%%%%%%%%%%%%%%%%%%%%%%%%%%%%%%%%%%%%%%%%%%%%%%%%%%%%%%%%%%%%%%%%%%%%%


\section{Development Notes}

%%%%%%%%%%%%%%%%%%%%%%%%%%%%%%%%%%%%%%%%%%%%%%%%%%%%%%%%%%%%%%%%%%%%%%%%%

\subsection{Integrating Sicstus Prolog programs}


Sicstus provide a nice interface for Java, called Jasper, based on a native
code library that is available for different platforms as part of the
Sicstus distribution.  Linking native code with Java is a slightly risky
business - unlike the gentle degradation often available via Java
exceptions, native code problems are likely to crash the whole application. 
It is also difficult to get a bug-free implementation of this type of
language mixing to work identically on different platforms. For example:

In Sicstus 3.8.6 on NT,
when calling Sicstus from Java, if the memory allocation directive
{\tt -Xmx200m} is given to the JVM, the sicstus runtime throws
this error:
\begin{verbatim} 
{ERROR: Memory allocation failed (upper 4 bits do not match MallocBase)}
Signal 127
\end{verbatim}

Using a slightly different version of jasper.jar than the one in the 
Sicstus distribution you're getting the native code from
(e.g. NT version 3.8.4 vs. Linux version 3.8.6) can crash the
system, unsurprisingly. But if we include jasper.jar in the GATE libraries
then this is likely to happen quite often.

For these reasons we don't include Sicstus support in GATE itself, but will
be happy to supply example code from our own modules that integrate Sicstus
code with GATE on request.
\vspace{1mm}\hrule 
% html: End of file: `userguide.html'
