
%%%%%%%%%%%%%%%%%%%%%%%%%%%%%%%%%%%%%%%%%%%%%%%%%%%%%%%%%%%%%%%%%%%%%%%%%%%%%
\chapt[sec:misc-creole:language-plugins]{Non-English Language Support}
\markboth{Non-English Language Support}{Non-English Language Support}
%%%%%%%%%%%%%%%%%%%%%%%%%%%%%%%%%%%%%%%%%%%%%%%%%%%%%%%%%%%%%%%%%%%%%%%%%%%%%
\nnormalsize

There are plugins available for processing the following languages:
French, German, Italian, Danish, Chinese, Arabic, Romanian, Hindi, Russian,
Welsh and Cebuano. Some of the applications are quite basic and just contain
some useful processing resources to get you started when developing a full
application. Others (Cebuano and Hindi) are more like toy systems built as
part of an exercise in language portability.

Note that if you wish to use individual language processing resources
without loading the whole application, you will need to load the
relevant plugin for that language in most cases. The plugins all
follow the same kind of format. Load the plugin using the plugin
manager in GATE Developer, and the relevant resources will be
available in the Processing Resources set.

Some plugins just contain a list of resources which can be added ad
hoc to other applications. For example, the Italian plugin simply
contains a lexicon which can be used to replace the English lexicon in
the default English POS tagger: this will provide a reasonable basic
POS tagger for Italian.

In most cases you will also find a directory in the relevant plugin
directory called data which contains some sample texts (in some cases,
these are annotated with NEs).

There are also a number of plugins, documented elsewhere in this manual that
while they default to processing English can be configured to support other
languages. These include the TaggerFramework
(Section~\ref{sec:parsers:taggerframework}), the OpenNLP plugin
(Section~\ref{sec:misc-creole:opennlp}), the Numbers Tagger
(Section~\ref{sec:misc-creole:numbers:numbers}), and the Snowball based stemmer
(Section~\ref{sec:parsers:stemmer}).  The LingPipe POS Tagger PR
(Section~\ref{sec:misc-creole:lingpipe:postagger}) now includes two models for
Bulgarian.

\sect[sec:misc-creole:language-identification]{Language Identification}

A common problem when handling multiple languages is determining the language of a
document or section of document. For example, patent documents often contain
the abstract in more than one language. In such cases you may want
to only process those sections written in English, or you may want to
run different processing resources over the different sections dependent upon
the language they are written in. Once documents or sections are annotated with
their language then it is easy to apply different processing resources to the
different sections using either a Conditional Corpus Pipeline or via the
Section-By-Section PR (Section \ref{sec:alignment:segment-processing}).
The problem is, of course, identifying the language.

The \texttt{Language\_Identification} plugin contains a
\htlink{http://textcat.sourceforge.net/}{TextCat} based PR for
performing language identification. The choice of languages used for
categorization is specified through a configuration file, the URL of which
is the PRs only initialization parameter.


The PR has the following runtime parameters.
%%
\begin{description}
\item[annotationType] If this is supplied, the PR classifies the text underlying
  each annotation of the specified type and stores the result as a feature on
  that annotation.  If this is left blank (null or empty), the PR classifies the
  text of each document and stores the result as a document feature.
\item[annotationSetName] The annotation set used for input and output; ignored
  if \emph{annotationType} is blank.
\item[languageFeatureName] The name of the document or annotation feature used
  to store the results.
\end{description}

Unlike most other PRs (which produce annotations), this one adds either document
features or annotation features.  (To classify both whole documents and spans
within them, use two instances of this PR.)  Note that classification accuracy
is better over long spans of text (paragraphs rather than sentences, for
example).

\textit{Note that an alternative language identification PR is available in the LingPipe plugin,
which is documented in Section \ref{sec:misc-creole:lingpipe:langid}.}

\subsect[sec:misc-creole:language-identification:fingerprints]{Fingerprint Generation}
Whilst the TextCat based PR supports a number of languages (not all of which are enabled in the default
configuration file), there may be occasiosn where you need to support a new language, or where the
language of domain specific documents affects the classification. In these situations you can use the
Fingerprint Generation PR included in the \texttt{Language\_Identification} to build new fingerprints
from a corpus of documents.

The PR has no initialization parameters and is configured through the following runtime parameters:
%%
\begin{description}
\item[annotationType] If this is supplied, the PR uses only the text underlying
  each annotation of the specified type to build the language fingerprint.
  If this is left blank (null or empty), the PR will instead use the whole of
  each document to create the fingerprint.
\item[annotationSetName] The annotation set used for input; ignored
  if \emph{annotationType} is blank.
\item[fingerprintURL] The URL to a file in which the fingerprint should be
  stored -- note that this must be a file URL.
\end{description}

\sect[sec:misc-creole:language-plugins:french]{French Plugin}

The French plugin contains two applications for NE recognition: one
which includes the TreeTagger for POS tagging in French
(french+tagger.gapp) , and one which does not (french.gapp). Simply
load the application required from the plugins/Lang\_French directory. You
do not need to load the plugin itself from the GATE Developer's Plugin
Management Console. Note that the TreeTagger must first be installed and set up
correctly (see Section \ref{sec:parsers:taggerframework} for
details). Check that the runtime parameters are set correctly for your
TreeTagger in your application. The applications both contain
resources for tokenisation, sentence splitting, gazetteer lookup, NE
recognition (via JAPE grammars) and orthographic coreference. Note
that they are not intended to produce high quality results, they are
simply a starting point for a developer working on French. Some sample
texts are contained in the plugins/Lang\_French/data directory.

Additionally, a more sophisticated application for French NE recognition is
available at \htlinkplain{https://github.com/GateNLP/gateapplication-French}.
This is designed to be flexible in the choice of POS tagger, by mapping the output POS 
tags to the \htlink{http://universaldependencies.org/u/pos/}{universal tagset}, 
which is used in the JAPE grammars. The application uses the Stanford POS tagger, 
but this can be easily replaced with any other tagger.

The GATE Cloud version of the plugin can be found here: \\
\htlinkplain{https://cloud.gate.ac.uk/shopfront/displayItem/french-named-entity-recognizer}

\sect[sec:misc-creole:language-plugins:german]{German Plugin}

The German plugin contains two applications for NE recognition: one
which includes the TreeTagger for POS tagging in German
(german+tagger.gapp) , and one which does not (german.gapp). Simply
load the application required from the plugins/Lang\_German/resources
directory. You do not need to load the plugin itself from the GATE
Developer's Plugin Management Console. Note that the TreeTagger must first be
installed and set up correctly (see
Section \ref{sec:parsers:taggerframework} for details). Check that the
runtime parameters are set correctly for your TreeTagger in your
application. The applications both contain resources for tokenisation,
sentence splitting, gazetteer lookup, compound analysis, NE
recognition (via JAPE grammars) and orthographic coreference. Some
sample texts are contained in the plugins/Lang\_German/data directory. We
are grateful to Fabio Ciravegna and
the \htlink{http://nlp.shef.ac.uk/dot.kom/}{Dot.KOM} project for use
of some of the components for the German plugin. 

Additionally, a more sophisticated application for German NE recognition is
available at \htlinkplain{https://github.com/GateNLP/gateapplication-German}.
This is designed to be flexible in the choice of POS tagger, by mapping the output POS 
tags to the \htlink{http://universaldependencies.org/u/pos/}{universal tagset}, 
which is used in the JAPE grammars. The application uses the Stanford POS tagger, 
but this can be easily replaced with any other tagger.

The GATE Cloud version of the plugin can be found here: \\
\htlinkplain{https://cloud.gate.ac.uk/shopfront/displayItem/german-named-entity-recognizer}

\sect[sec:misc-creole:language-plugins:romanian]{Romanian Plugin}

The Romanian plugin contains an application for Romanian NE
recognition (romanian.gapp). Simply load the application from the
plugins/Lang\_Romanian/resources directory. You do not need to load the
plugin itself from the GATE Developer's Plugin
Management Console. The application
contains resources for tokenisation, gazetteer lookup, NE recognition
(via JAPE grammars) and orthographic coreference. Some sample texts
are contained in the plugins/romanian/corpus directory.

The GATE Cloud version of the plugin can be found here: \\
\htlinkplain{https://cloud.gate.ac.uk/shopfront/displayItem/romanian-named-entity-recognizer}

\sect[sec:misc-creole:language-plugins:arabic]{Arabic Plugin}

The Arabic plugin contains a simple application for Arabic NE
recognition (arabic.gapp). Simply load the application from the
plugins/Lang\_Arabic/resources directory. You do not need to load the plugin
itself from the GATE Developer's Plugin
Management Console. The application contains
resources for tokenisation, gazetteer lookup, NE recognition (via JAPE
grammars) and orthographic coreference. Note that there are two types
of gazetteer used in this application: one which was derived
automatically from training data (Arabic inferred gazetteer), and one
which was created manually. Note that there are some other
applications included which perform quite specific tasks (but can
generally be ignored). For example, arabic-for-bbn.gapp and
arabic-for-muse.gapp make use of a very specific set of training data
and convert the result to a special format. There is also an
application to collect new gazetteer lists from training data
(arabic\_lists\_collector.gapp). For details of the gazetteer list
collector please see Section \ref{sec:gazetteers:listscollector}.


\sect[sec:misc-creole:language-plugins:chinese]{Chinese Plugin}

The Chinese plugin contains two components: a simple application for Chinese 
NE recognition (chinese.gapp) and a component called ``Chinese Segmenter''.

In order to use the former, simply load the application from the\linebreak 
plugins/Lang\_Chinese/resources directory. You do not need to load the plugin 
itself from the GATE Developer's Plugin
Management Console. The application contains resources 
for tokenisation, gazetteer lookup, NE recognition (via JAPE grammars) and 
orthographic coreference. The application makes use of some gazetteer lists (and
a grammar to process them) derived automatically from training data, as well as 
regular hand-crafted gazetteer lists. There are also applications
(listscollector.gapp, adj\_collector.gapp and
nounperson\_collector.gapp) to create such lists, and various other
application to perform special tasks such as coreference evaluation
(coreference\_eval.gapp) and converting the output to a different
format (ace-to-muse.gapp).


%%%%%%%%%%%%%%%%%%%%%%%%%%%%%%%%%%%%%%%%%%%%%%%%%%%%%%%%%%%%%%%%%%%%%%%%%%%%%
\subsect[sec:misc-creole:chineseSeg]{Chinese Word Segmentation}
%%%%%%%%%%%%%%%%%%%%%%%%%%%%%%%%%%%%%%%%%%%%%%%%%%%%%%%%%%%%%%%%%%%%%%%%%%%%%

Unlike English, Chinese text does not have a symbol (or delimiter)
 such as blank space to explicitly separate a word from the
 surrounding words.  Therefore, for automatic Chinese text processing,
 we may need a system to recognise the words in Chinese text, a
 problem known as Chinese word segmentation. The plugin described in
 this section performs the task of Chinese word segmentation. It is
 based on our work using the Perceptron learning algorithm for the Chinese
 word segmentation task of the Sighan 2005\footnote{See
 http://www.sighan.org/bakeoff2005/ for the Sighan-05 task}.
\cite{Yaoyong05b}. Our Perceptron based system has achieved very good
performance in the Sighan-05 task.

The plugin is called {\em Lang\_Chinese} and is available in the GATE
distribution. The corresponding processing resource's name is {\em Chinese
  Segmenter PR}. Once you load the PR into GATE, you may put it into a {\em
  Pipeline} application. Note that it does not process a corpus of documents,
but a directory of documents provided as a parameter (see description of
parameters below). The plugin can be used to learn a model from segmented
Chinese text as training data. It can also use the learned model to segment
Chinese text.  The plugin can use different learning algorithms to learn
different models.  It can deal with different character encodings for
Chinese text, such as UTF-8, GB2312 or BIG5.  These options can
be selected by setting the run-time parameters of the plugin.

The plugin has five run-time parameters, which are described in the following.
\begin{itemize}
\item {\bf learningAlg} is a String variable, which specifies the learning
  algorithm used for producing the model. Currently it has two values, {\em
    PAUM} and {\em SVM}, representing the two popular learning algorithms
  Perceptron and SVM, respectively. The default value is {\em PAUM}.\\
  Generally speaking, SVM may perform better than Perceptron, in particular
  for small training sets. On the other hand, Perceptron's learning is much
  faster than SVM's. Hence, if you have a small training set, you may want to
  use SVM to obtain a better model. However, if you have a big training set
  which is typical for the Chinese word segmentation task, you may want to use
  Perceptron for learning, because the SVM's learning may take too long
  time. In addition, using a big training set, the performance of the
  Perceptron model is quite similar to that of the SVM model. See
  \cite{Yaoyong05b} for the experimental comparison of SVM and Perceptron on
  Chinese word segmentation.

\item {\bf learningMode} determines the two modes of using the plugin, either
  learning a model from training data or applying a learned model to segment
  Chinese text.  Accordingly it has two values, {\em SEGMENTING} and {\em
    LEARNING}. The default value is
  {\em SEGMENTING}, meaning segmenting the Chinese text.\\
  Note that you first need to learn a model and then you can use the learned
  model to segment the text. Several models using the training data used in
  the Sighan-05 Bakeoff are available for this plugin, which you can use to
  segment your Chinese text. More descriptions about the provided models will
  be given below.
\item {\bf modelURL} specifies an URL referring to a directory containing the
  model. If the plugin is in the {\em LEARNING} runmode, the model learned
  will be put into the directory. If it is in the {\em SEGMENTING} runmode,
  the plugin will use the model stored in the directory to segment the
  text. The models learned from the Sighan-05 bakeoff training data will be
  discussed below.
\item {\bf textCode} specifies the encoding of the text used. For example it can
  be UTF-8, BIG5, GB2312 or any other encoding for Chinese text. Note that, when
  you segment some Chinese text using a learned model, the Chinese text should
  use the same encoding as the one used by the training text for obtaining the
  model.
\item {\bf textFilesURL} specifies an URL referring to a directory containing
  the Chinese documents. All the documents contained in this directory (but
  not those documents contained in its sub-directory if there is any) will be
  used as input data. In the {\em LEARNING} runmode, those documents contain
  the segmented Chinese text as training data. In the {\em SEGMENTING}
  runmode, the text in those documents will be segmented. The segmented text
  will be stored in the corresponding documents in the sub-directory called
  {\em segmented}.
\end{itemize}

The following PAUM models are distributed with plugins and are available as
compressed zip files under the plugins/Lang\_Chinese/resources/models 
directory. Please unzip them to use. In detail, those models were learned using 
the PAUM learning algorithm from the corpora provided by Sighan-05 bakeoff task.
\begin{itemize}
\item the PAUM model learned from PKU training data, using the PAUM
  learning algorithm and the {\em UTF-8} encoding, is available as 
  model-paum-pku-utf8.zip.
\item the PAUM model learned from PKU training data, using the PAUM
  learning algorithm and the {\em GB2312} encoding, is available as 
  model-paum-pku-gb.zip.
\item the PAUM model learned from AS training data, using the PAUM
  learning algorithm and the {\em UTF-8} encoding, is available as 
  model-as-utf8.zip.
\item the PAUM model learned from AS training data, using the PAUM
  learning algorithm and the {\em BIG5} encoding, is available as
  model-as-big5.zip.
\end{itemize}

As you can see, those models were learned using different training data and
different Chinese text encodings of the same training data. The PKU training data
are news articles published in mainland China and use simplified Chinese,
while the AS training data are news articles published in Taiwan and use
traditional Chinese. If your text are in simplified Chinese, you can use the
models trained by the PKU data. If your text are in traditional Chinese, you
need to use the models trained by the AS data. If your data are in GB2312 encoding
or any compatible encoding, you need use the model trained by the corpus in GB2312
encoding.

Note that the segmented Chinese text (either used as training data or produced
by this plugin) use the blank space to separate a word from its surrounding
words. Hence, if your data are in Unicode such as UTF-8, you can use the {\em
  GATE Unicode Tokeniser} to process the segmented text to add the Token
annotations into your text to represent the Chinese words.  Once you get the
annotations for all the Chinese words, you can perform further processing such
as POS tagging and named entity recognition.

\sect[sec:misc-creole:language-plugins:hindi]{Hindi Plugin}

The Hindi plugin (`Lang\_Hindi') contains a set of resources for basic Hindi NE
recognition which mirror the ANNIE resources but are customised to the Hindi
language. You need to have the ANNIE plugin loaded first in order to load any of
these PRs. With the Hindi, you can create an application similar to ANNIE but
replacing the ANNIE PRs with the default PRs from the plugin.

\sect[sec:misc-creole:language-plugins:russian]{Russian Plugin}

The Russian plugin (\verb|Lang_Russian|) contains a set of resource for a Russian
IE application which mirrors the construction of ANNIE. This includes custom
components for part-of-speech tagging, morphological analysis and gazetteer
lookup. A number of ready-made applications are also available which combine
these resources together in a number of ways.

The GATE Cloud version of the plugin can be found here: \\
\htlinkplain{https://cloud.gate.ac.uk/shopfront/displayItem/russian-named-entity-recognizer-basic}

\sect[sec:misc-creole:language-plugins:bulgarian]{Bulgarian Plugin}
The Bulgarian plugin (\verb|Lang_Bulgarian|) containts a GATE PR which 
integrates the \htlink{http://lml.bas.bg/~nakov/bulstem/index.html}{BulStem stemmer}
into GATE. Currently no other Bulgarian specific PRs are available so
the stemmer should be used with the Unicode tokenizer and a sentence splitter
to process Bulgarian language documents.

\sect[sec:misc-creole:language-plugins:danish]{Danish Plugin}
The Danish plugin (\verb|Lang_Danish|) contains resources for a Danish IE
application. As well as a set of tokeniser rules and gazetteer lists tuned for
Danish, the plugin includes models for the Stanford CoreNLP POS tagger and
named entity recogniser trained on the Danish PAROLE corpus and the Copenhagen
Dependency Treebank respectively.  Full details can be found in the EACL 2014
paper \cite{Derczynski2014d}.

The Java code in this plugin (the tokeniser and gazetteer) is released under
the same LGPL licence as GATE itself, but the POS tagger and NER models are
subject to the full GPL as this is the licence of the data used for training.

\sect[sec:misc-creole:language-plugins:welsh]{Welsh Plugin}
The Welsh plugin (\verb|Lang_Welsh|) is the result of the Welsh Natural
Language Toolkit
project\footnote{\htlinkplain{http://hypermedia.research.southwales.ac.uk/kos/wnlt/}},
funded by the Welsh Government.  It contains a set of resources that mirror the
English-language ANNIE application, but adapted to the Welsh language.  The
plugin includes a tokeniser, sentence splitter, POS tagger, morphological
analyser, gazetteers and named entity JAPE grammars, with a ready-made
application called \emph{CYMRIE} to combine them all.

The GATE Cloud version of the plugin can be found here: \\
\htlinkplain{https://cloud.gate.ac.uk/shopfront/displayItem/cymrie-welsh-named-entity-recogniser}
