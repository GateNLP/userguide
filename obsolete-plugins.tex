%%%%%%%%%%%%%%%%%%%%%%%%%%%%%%%%%%%%%%%%%%%%%%%%%%%%%%%%%%%%%%%%%%%
\chapt[chap:obsolete-plugins]{Obsolete CREOLE Plugins}
\markboth{Obsolete CREOLE Plugins}{Obsolete CREOLE Plugins}
%%%%%%%%%%%%%%%%%%%%%%%%%%%%%%%%%%%%%%%%%%%%%%%%%%%%%%%%%%%%%%%%%%%

These plugins should not be needed for new development with GATE, but are documented here
in case they are required by an old application.  Note that the obsolete plugins do not
appear in GATE's plugin manager by default.

% The remainder of this appendix is maintained simply by moving the documentation of
% obsolete plugins from where they were originally included to here. Note that you
% must ensure that section labels etc. are maintained so that help URLs are not
% broken when the documentation is moved here.


%%%%%%%%%%%%%%%%%%%%%%%%%%%%%%%%%%%%%%%%%%%%%%%%%%%%%%%%%%%%%%%%%%%%%%%%%%%%%
\sect[sec:misc-creole:japec]{Ontotext JapeC Compiler}
%%%%%%%%%%%%%%%%%%%%%%%%%%%%%%%%%%%%%%%%%%%%%%%%%%%%%%%%%%%%%%%%%%%%%%%%%%%%%

\textit{Note: the JapeC compiler does not currently support the new JAPE
language features introduced in July--September 2008.  If you need to use
negation, the \texttt{@length} and \texttt{@string} accessors, the contextual
operators \texttt{within} and \texttt{contains}, or any comparison operators
other than \texttt{==}, then you will need to use the standard JAPE
transducer instead of JapeC.}

JapeC is an alternative implementation of the JAPE language which works by
compiling JAPE grammars into Java code.  Compared to the standard
implementation, these compiled grammars can be several times faster to run.  At
Ontotext, a modified version of the ANNIE sentence splitter using compiled
grammars has been found to run up to five times as fast as the standard
version.  The compiler can be invoked manually from the command line, or used
through the `Ontotext Japec Compiler' PR in the {\it Jape\_Compiler} plugin.

The `Ontotext Japec Transducer' (com.ontotext.gate.japec.JapecTransducer) is
a processing resource that is designed to be an alternative to the original
Jape Transducer. You can simply replace gate.creole.Transducer with
com.ontotext.gate.japec.JapecTransducer in your gate application and it should
work as expected.

The Japec transducer takes the same parameters as the standard JAPE transducer:
\begin{description}
\item[grammarURL] the URL from which the grammar is to be loaded.  Note that
the Japec Transducer will {\it only} work on {\tt file:} URLs.  Also, the
alternative {\it binaryGrammarURL} parameter of the standard transducer is not
supported.
\item[encoding] the character encoding used to load the grammars.
\item[ontology] the ontology used for ontolog-aware transduction.
\end{description}

Its runtime parameters are likewise the same as those of the standard
transducer:
\begin{description}
\item[document] the document to process.
\item[inputASName] name of the AnnotationSet from which input annotations to
the transducer are read.
\item[outputASName] name of the AnnotationSet to which output annotations from
the transducer are written.
\end{description}

The Japec compiler itself is written in Haskell.  Compiled binaries are
provided for Windows, Linux (x86) and Mac OS X (PowerPC), so no Haskell
interpreter is required to run Japec on these platforms.  For other platforms,
or if you make changes to the compiler source code, you can build the compiler
yourself using the Ant build file in the Jape\_Compiler plugin directory.  You
will need to install the latest version of the Glasgow Haskell
Compiler\footnote{GHC version 6.4.1 was used to build the supplied binaries for
Windows, Linux and Mac} and associated libraries.  The japec compiler can then
be built by running:
\begin{verbatim}
../../bin/ant japec.clean japec
\end{verbatim}
from the Jape\_Compiler plugin directory.

%%%%%%%%%%%%%%%%%%%%%%%%%%%%%%%%%%%%%%%%%%%%%%%%%%%%%%%%%%%%%%%%%%%%%%%%%%%%%
\sect[sec:misc-creole:google]{Google Plugin}
%%%%%%%%%%%%%%%%%%%%%%%%%%%%%%%%%%%%%%%%%%%%%%%%%%%%%%%%%%%%%%%%%%%%%%%%%%%%%

This plugin is no longer operational because the functionality, provided by
Google, on which it depends, is no longer available.

%%%%%%%%%%%%%%%%%%%%%%%%%%%%%%%%%%%%%%%%%%%%%%%%%%%%%%%%%%%%%%%%%%%%%%%%%%%%%
\sect[sec:misc-creole:yahoo]{Yahoo Plugin}
%%%%%%%%%%%%%%%%%%%%%%%%%%%%%%%%%%%%%%%%%%%%%%%%%%%%%%%%%%%%%%%%%%%%%%%%%%%%%

The Yahoo API is now integrated with GATE, and can be used as a PR-based
plugin.  This plugin, `Web\_Search\_Yahoo', allows the user to query Yahoo and
build a document corpus that contains the search results returned by Yahoo for
the query.  For more information about the Yahoo API please refer to
\htlinkplain{http://developer.yahoo.com/search/}. In order to use the Yahoo
PR, you need to obtain an application ID.

The Yahoo PR can be used for a number of different application scenarios. For
example, one use case is where a user wants to find the different named
entities that can be associated with a particular individual. In this example,
the user could build a collection of documents by querying Yahoo with the
individual's name and then running ANNIE over the collection. This would
annotate the results and show the different Organization, Location and other
entities that are associated with the query.

%%%%%%%%%%%%%%%%%%%%%%%%%%%%%%%%%%%%%%%%%%%%%%%%%%%%%%%%%%%%%%%%%%%%%%%%%%%%%
\subsect{Using the YahooPR}
%%%%%%%%%%%%%%%%%%%%%%%%%%%%%%%%%%%%%%%%%%%%%%%%%%%%%%%%%%%%%%%%%%%%%%%%%%%%%

In order to use the PR, you first need to load the plugin using the
GATE Developer plugin manager. Once the PR is loaded, it can be
initialized by creating an instance of a new PR. Here you need to
specify the Yahoo Application ID. Please use the license key assigned
to you by registering with Yahoo.

Once the Yahoo PR is initialized, it can be placed in a pipeline 
or a conditional pipeline application. This pipeline would contain the instance 
of the Yahoo PR just initialized as above. There are a number of
parameters to be set at runtime:

\begin{itemize}
\item \verb|corpus|: The corpus used by the plugin to add or append
documents from the Web.

\item \verb|corpusAppendMode|: If set to \verb|true|, will append documents
to the corpus. If set to \verb|false|, will remove preexisting
documents from the corpus, before adding the documents newly fetched by the PR

\item \verb|limit|: A limit on the results returned by the search. Default set to
10.

\item \verb|pagesToExclude|: This is an optional parameter. It is a list with
URLs not to be included in the search.

\item \verb|query|: The query sent to Yahoo. It is in the format
accepted by Yahoo.

\end{itemize}

Once the required parameters are set we can run the pipeline. This
will then download all the URLs in the results and create a document
for each. These documents would be added to the corpus.


%%%%%%%%%%%%%%%%%%%%%%%%%%%%%%%%%%%%%%%%%%%%%%%%%%%%%%%%%%%%%%%%%%%%%%%%%%%%%
\sect[sec:gazetteers:gaze]{Gazetteer Visual Resource - GAZE}
%%%%%%%%%%%%%%%%%%%%%%%%%%%%%%%%%%%%%%%%%%%%%%%%%%%%%%%%%%%%%%%%%%%%%%%%%%%%%

Gaze is a tool for editing the gazetteer lists , definitions and mapping to
ontology. It is suitable for use both for Plain/Linear Gazetteers (Default and
Hash Gazetteers) and Ontology-enabled Gazetteers (OntoGazetteer). The Gazetteer
PR associated with the viewer is reinitialised every time a save operation is
performed. Note that GAZE does not scale up to very large lists (we suggest not
using it to view over 40,000 entries and not to copy inside more than 10, 000
entries).

Gaze is part of and provided by the ANNIE plugin. To make it possible
to visualize gazetteers with the Gaze visualizer, the ANNIE plugin must
be loaded first.
Double clicking on a gazetteer PR that uses a gazetteer definition (index)
file  will display
the contents of the gazetteer in the main window. The first pane will
display the definition file, while the right pane will display
whichever gazetteer list has been selected from it.

A gazetteer list can be modified simply by typing in it. it can be
saved by clicking the Save button. When a list is saved, the whole
gazetteer is automatically reinitialised (and will be ready for use in
GATE immediately).

To edit the definition file, right click inside the pane and choose
from the options (Inset, Edit, Remove). A pop-up menu will appear to
guide you through the remaining process. Save the definition file by
selecting Save. Again, the gazetteer will be reinitialised
automatically.

%%%%%%%%%%%%%%%%%%%%%%%%%%%%%%%%%%%%%%%%%%%%%%%%%%%%%%%%%%%%%%%%%%%%%%%%%%%%%
\subsect{Display Modes}
%%%%%%%%%%%%%%%%%%%%%%%%%%%%%%%%%%%%%%%%%%%%%%%%%%%%%%%%%%%%%%%%%%%%%%%%%%%%%

 The display mode depends on the type of gazetteer loaded in the VR. The mode in
 which Linear/Plain Gazetteers are loaded is called Linear/Plain Mode. In
 this mode, the Linear Definition is displayed in the left pane, and the
 Gazetteer List is displayed in the right pane. The Ontology/Extended
 mode is on when the displayed gazetteer is ontology-aware, which means that
 there exists a mapping between classes in the ontology and lists of phrases.
 Two more panes are displayed when in this mode. On the top in the left-most
 pane there is a tree view of the ontology hierarchy, and at the bottom the
 mapping definition is displayed. This section describes the Linear/Plain
 display mode, the Ontology/Extended mode is described in 
 section~\ref{sec:gazetteers:onto_gaze}.
	
Whenever a gazetteer PR that uses a gazetteer definition (index) file is loaded, 
the Gaze gazetteer visualisation  will
appear on double-click over the gazetteer in the Processing Resources branch of
the Resources Tree.

%%%%%%%%%%%%%%%%%%%%%%%%%%%%%%%%%%%%%%%%%%%%%%%%%%%%%%%%%%%%%%%%%%%%%%%%%%%%%
\subsect{Linear Definition Pane}
%%%%%%%%%%%%%%%%%%%%%%%%%%%%%%%%%%%%%%%%%%%%%%%%%%%%%%%%%%%%%%%%%%%%%%%%%%%%%

This pane displays the nodes of the linear definition, and allows manipulation
of the whole definition as a file, as well as the single nodes. Whenever a
gazetteer list is modified, its node in the linear definition is coloured in red.

%%%%%%%%%%%%%%%%%%%%%%%%%%%%%%%%%%%%%%%%%%%%%%%%%%%%%%%%%%%%%%%%%%%%%%%%%%%%%
\subsect{Linear Definition Toolbar}
%%%%%%%%%%%%%%%%%%%%%%%%%%%%%%%%%%%%%%%%%%%%%%%%%%%%%%%%%%%%%%%%%%%%%%%%%%%%%

All the functionality explained in this section (New, Load, Save, Save As) is
accessible also via File | Linear Definition in the menu bar of Gaze.

\textbf{New} -- Pressing New invokes a file dialog where the location of the new
definition is specified.

\textbf{Load} -- Pressing Load invokes a file dialog, and after locating the new
definition it is loaded by pressing Open.

\textbf{Save} -- Pressing Save saves the definition to the location from which it has
been read.
	
\textbf{Save As} -- Pressing Save As allows another location to be chosen, and
the definition saved there.
		
%%%%%%%%%%%%%%%%%%%%%%%%%%%%%%%%%%%%%%%%%%%%%%%%%%%%%%%%%%%%%%%%%%%%%%%%%%%%%
\subsect{Operations on Linear Definition Nodes}
%%%%%%%%%%%%%%%%%%%%%%%%%%%%%%%%%%%%%%%%%%%%%%%%%%%%%%%%%%%%%%%%%%%%%%%%%%%%%

\textbf{Double-click node} -- Double-clicking on a definition node forces
the displaying of the gazetteer list of the node in the right-most pane of the
viewer.

\textbf{Insert} -- On right-click over a node and choosing Insert, a dialog is
displayed, requesting List, Major Type, Minor Type and Languages. The mandatory
fields are List and Major Type. After pressing OK, a new linear node is added to
the definition.

\textbf{Remove} -- On right-click over a node and choosing Remove, the selected
linear node is removed from the definition.

\textbf{Edit} -- On right-click over a node and choosing Edit a dialog is
displayed allowing changes of the fields List, Major Type, Minor Type and
Languages. 

%%%%%%%%%%%%%%%%%%%%%%%%%%%%%%%%%%%%%%%%%%%%%%%%%%%%%%%%%%%%%%%%%%%%%%%%%%%%%
\subsect{Gazetteer List Pane}
%%%%%%%%%%%%%%%%%%%%%%%%%%%%%%%%%%%%%%%%%%%%%%%%%%%%%%%%%%%%%%%%%%%%%%%%%%%%%

The gazetteer list pane has a toolbar with similar to the linear definition's
buttons (New, Load, Save, Save As). They work as predicted by their names and as
explained in the Linear Definition Pane section, and are also accessible from
File / Gazetteer List in the menu bar of Gaze. The only addition is Save All
which saves all modified gazetteer lists. The editing of the gazetteer list is
as simple as editing a text file. One could use Ctrl+A to select the whole list,
Ctrl+C to copy the selected, Ctrl+V to paste it, Del to delete the selected text
or a single character, etc.
	
%%%%%%%%%%%%%%%%%%%%%%%%%%%%%%%%%%%%%%%%%%%%%%%%%%%%%%%%%%%%%%%%%%%%%%%%%%%%%
\subsect{Mapping Definition Pane}
%%%%%%%%%%%%%%%%%%%%%%%%%%%%%%%%%%%%%%%%%%%%%%%%%%%%%%%%%%%%%%%%%%%%%%%%%%%%%

The mapping definition is displayed one mapping node per row. It consists of a
gazetteer list, ontology URL, and class id. The content of the gazetteer list in
the node is accessible through double-clicking. It is displayed in the Gazetteer
List Pane. The toolbar allows the creation of a new definition (New), the
loading of an existing one (Load), saving to the same or new location (Save/Save
As). The functionality of the toolbar buttons is also available via File.

%%%%%%%%%%%%%%%%%%%%%%%%%%%%%%%%%%%%%%%%%%%%%%%%%%%%%%%%%%%%%%%%%%%%%%%%%%%%%
\sect[sec:misc-creole:google-translate]{Google Translator PR}
%%%%%%%%%%%%%%%%%%%%%%%%%%%%%%%%%%%%%%%%%%%%%%%%%%%%%%%%%%%%%%%%%%%%%%%%%%%%%

The Google Translator PR allows users to translate their documents into many 
other languages using the Google translation service. It is based on the library
called {\tt google-translate-api-java} which is distributed under the LGPL 
licence and is available to download from 
\htlinkplain{http://code.google.com/p/google-api-translate-java/}.

The PR is included in the plugin called Web\_Translate\_Google and depends on 
the Alignment plugin. (chapter \ref{chap:alignment}).

If a user wants to translate an English document into French using the Google 
Translator PR. The first thing user needs to do is to create an instance of 
CompoundDocument with the English document as a member of it.  The 
CompoundDocument in GATE provides a convenient way to group parallel documents 
that are translations of one other (see chapter \ref{chap:alignment} for more
information).  The idea is to use text from one of the members of the provided 
compound document, translate it using the Google translation service and create
another member with the translated text.  In the process, the PR also aligns the
chunks of parallel texts.  Here, a chunk could be a sentence, paragraph, section
or the entire document.

{\tt siteReferrer} is the only init-time parameter required to instantiate the
PR. It has to be a valid website address. The value of this parameter is 
required to inform Google about the users using their service. There are seven
run-time parameters:

\begin{itemize}
\item document - an instance of the compound document with a member document 
containing source text.
\item sourceDocumentId - id of the source member document that needs to be 
translated.
\item targetDocumentId - id of the target member document. This document is 
created by the PR and contains the translated text.
\item sourceLanguage - the language of the source document.
\item targetLanguage - the language into which the source document should be 
translated.
\item unitOfTranslation - annotation type used for identifying chunks of texts 
to be translated and aligned.
\item inputASName - name of the annotation set which contains unit of 
translations.
\item alignmentFeatureName - name of the alignment feature used for storing the
alignment information.  The alignment feature is a document feature stored on
the compound document.
\end{itemize}
