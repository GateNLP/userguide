%'nested' switch is used to make section headings format correctly whether
%they are in the appendix or the introduction.
%
% Within this file you should use \rcSect[label]{Heading} instead of
% \sect[sec:changes:label]{Heading}, and \rcSubsect{Heading} instead of
% \subsect{Heading}.  These commands are redefined appropriately depending on
% whether we are processing the intro subsection or the full changelog
% appendix.
%
\ifnested
  \def\rcSectNoLabel#1{\subsect{#1}}
  \def\rcSect[#1]#2{\subsect[subsec:changes:#1]{#2}}
  \def\rcSubsect#1{\subsubsect{#1}}
  \def\rcSubsubsect#1{\subsubsubsect{#1}}
\else
  \def\rcSectNoLabel#1{\sect{#1}}
  \def\rcSect[#1]#2{\sect[sec:changes:#1]{#2}}
  \def\rcSubsect#1{\subsect{#1}}
  \def\rcSubsubsect#1{\subsubsect{#1}}
\fi

\rcSect[8.6]{Version 8.6 (June 2019)}

GATE Developer 8.6 is mainly a maintenance and stability release, but there are
some important new features, in particular around the processing of
Twitter data:

\begin{itemize}
\item The \verb!Format_Twitter! plugin can now correctly handle extended 280
  character tweets and the latest Twitter JSON format. See
  Section~\ref{sec:social:twitter:format} for full details.
\item The new \verb!Format_JSON! plugin provides import/export support for GATE
  JSON.  This is essentially the old style Twitter format, but it no longer
  needs to track changes to the Twitter JSON format so should be more suitable
  for long term storage of GATE documents as JSON files. See
  Section~\ref{sec:creole:gatejson} for more details.  This plugin makes use of
  a new mechanism whereby document format parsers can take parameters via the
  document MIME type, which may be useful to third party formats too.
\end{itemize}

Many bugs have been fixed and documentation improved, in particular:

\begin{itemize}
\item The plugin loading mechanism now properly respects the user's Maven
  \verb!settings.xml!:
  \begin{itemize}
  \item HTTP proxy and ``mirror'' repository settings now work properly,
  including authentication.  Also plugin resolution will now use the system
  proxy (if there is one) by default if there is no proxy specified in the
  Maven settings.
  \item The ``offline'' setting is respected, and will prevent GATE from trying
  to fetch plugins from remote repositories altogether -- for this to work, all
  the plugins you want to use must already be cached locally, or you can use
  ``Export for GATE Cloud'' to make a self-contained copy of an application
  including all its plugins.
  \end{itemize}
\item Upgraded many dependencies including Tika and Jackson to avoid known
  security bugs in the previous versions.
\item Documentation improvements for the Kea plugin, the Corpus QA and
  annotation diff tools, and the default GATE XML and inline XML formats
  (section~\ref{sec:developer:dump})
\item For plugin developers, the standard plugin testing framework generates a
  report detailing all the plugin-to-plugin dependencies, including those that
  are only expressed in the plugin's example saved applications
  (section~\ref{sec:api:bootstrap:dependencies}).
\end{itemize}

Some obsolete plugins have been removed (Websphinx web crawler, which depends
on an unmaintained library, and the RASP parser, whose external binary is no
longer available for modern operating systems), and there are many smaller bug
fixes and improvements.

Note: following changes to Oracle's JDK licensing scheme, we now recommend
running GATE using the freely-available OpenJDK.  The
\htlink{https://adoptopenjdk.net}{AdoptOpenJDK} project offers simple
installers for all major platforms, and major Linux distributions such as
Ubuntu and CentOS offer OpenJDK packages as standard.  See
section~\ref{sec:gettingstarted:install} for full installation instructions.

\rcSect[8.5.1]{Version 8.5.1 (June 2018)}

Version 8.5.1 is a minor release to fix a few critical bugs in 8.5:

\begin{itemize}
\item Fixed an exception that prevented the ANNIC search GUI from opening.
\item Fixed a problem with ``Export for GATE Cloud'' that meant some resources
  were not getting included in the output ZIP file.
\item Fixed the XML schema in the \verb!gate-spring! library.
\end{itemize}

\rcSect[8.5]{Version 8.5 (May 2018)}

GATE Developer and Embedded 8.5 introduces a number of significant internal
changes to the way plugins are managed, but with the exception of the plugin
manager most users will not see significant changes in the way they use GATE.

\begin{itemize}
\item The GATE plugins are no longer bundled with the GATE Developer
  distribution, instead each plugin is downloaded from a repository at runtime,
  the first time it is used.  This means the distribution is much smaller than
  previous versions.
\item Most plugins are now distributed as a single JAR file through the
  Java-standard ``Central Repository'', and resource files such as gazetteers
  and JAPE grammars are bundled inside the plugin JAR rather than being
  separate files on disk.  If you want to modify the resources of a plugin
  then GATE provides a tool to extract an editable copy of the files from
  a plugin onto your disk -- it is no longer possible to edit plugin grammars
  in place.
\item This makes dependencies between plugins much easier to manage -- a plugin
  can specify its dependencies declaratively by name and version number rather
  than by fragile relative paths between plugin directories.
\end{itemize}

GATE 8.5 remains backwards compatible with existing third-party plugins, though
we encourage you to convert your plugins to the new style where possible.

Further details on these changes can be found in sections
\ref{sec:developer:plugins} (the plugin manager in GATE Developer),
\ref{sec:api:plugins} (loading plugins via the GATE Embedded API),
\ref{sec:api:bootstrap} (creating a new plugin from scratch), and
\ref{sec:api:updateplugins} (converting an existing plugin to the new style).

If you have an existing saved application from GATE version 8.4.1 or earlier it
will be necessary to ``upgrade'' it to use the new core plugins.  An upgrade
tool is provided on the ``Tools'' menu of GATE Developer, and is described in
section Section~\ref{sec:developer:convertxgapp}.

\rcSubsect{For developers}

As part of this release, GATE development has moved from SourceForge to GitHub
-- bug reports, patches and feature requests should now use the GitHub issue
tracker as described in section~\ref{sec:development:bugs}.

% vim:ft=tex
