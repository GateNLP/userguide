%'nested' switch is used to make section headings format correctly whether
%they are in the appendix or the introduction.
%
% Within this file you should use \rcSect[label]{Heading} instead of
% \sect[sec:changes:label]{Heading}, and \rcSubsect{Heading} instead of
% \subsect{Heading}.  These commands are redefined appropriately depending on
% whether we are processing the intro subsection or the full changelog
% appendix.
%
\ifnested
  \def\rcSectNoLabel#1{\subsect{#1}}
  \def\rcSect[#1]#2{\subsect[subsec:changes:#1]{#2}}
  \def\rcSubsect#1{\subsubsect{#1}}
  \def\rcSubsubsect#1{\subsubsubsect{#1}}
\else
  \def\rcSectNoLabel#1{\sect{#1}}
  \def\rcSect[#1]#2{\sect[sec:changes:#1]{#2}}
  \def\rcSubsect#1{\subsect{#1}}
  \def\rcSubsubsect#1{\subsubsect{#1}}
\fi

\rcSect[8.6-SNAPSHOT]{Version 8.6-SNAPSHOT}

\begin{itemize}
\item Removed the RASP plugin.
\item The \verb!Format_Twitter! plugin has been updated to correctly handle 280
  character tweets and the latest Twitter JSON format. See
  \ref{sec:social:twitter:format} for full details.
\item The new \verb!Format_JSON! plugin provides import/export support for GATE
  JSON.  This is essentially the old style Twitter format, but it no longer
  needs to track changes to the Twitter JSON format so should be more suitable
  for long term strage of GATE documents as JSON files. See
  \ref{sec:creole:gatejson} for more details.
\end{itemize}

\rcSect[8.5.1]{Version 8.5.1 (June 2018)}

Version 8.5.1 is a minor release to fix a few critical bugs in 8.5:

\begin{itemize}
\item Fixed an exception that prevented the ANNIC search GUI from opening.
\item Fixed a problem with ``Export for GATE Cloud'' that meant some resources
  were not getting included in the output ZIP file.
\item Fixed the XML schema in the \verb!gate-spring! library.
\end{itemize}

\rcSect[8.5]{Version 8.5 (May 2018)}

GATE Developer and Embedded 8.5 introduces a number of significant internal
changes to the way plugins are managed, but with the exception of the plugin
manager most users will not see significant changes in the way they use GATE.

\begin{itemize}
\item The GATE plugins are no longer bundled with the GATE Developer
  distribution, instead each plugin is downloaded from a repository at runtime,
  the first time it is used.  This means the distribution is much smaller than
  previous versions.
\item Most plugins are now distributed as a single JAR file through the
  Java-standard ``Central Repository'', and resource files such as gazetteers
  and JAPE grammars are bundled inside the plugin JAR rather than being
  separate files on disk.  If you want to modify the resources of a plugin
  then GATE provides a tool to extract an editable copy of the files from
  a plugin onto your disk -- it is no longer possible to edit plugin grammars
  in place.
\item This makes dependencies between plugins much easier to manage -- a plugin
  can specify its dependencies declaratively by name and version number rather
  than by fragile relative paths between plugin directories.
\end{itemize}

GATE 8.5 remains backwards compatible with existing third-party plugins, though
we encourage you to convert your plugins to the new style where possible.

Further details on these changes can be found in sections
\ref{sec:developer:plugins} (the plugin manager in GATE Developer),
\ref{sec:api:plugins} (loading plugins via the GATE Embedded API),
\ref{sec:api:bootstrap} (creating a new plugin from scratch), and
\ref{sec:api:updateplugins} (converting an existing plugin to the new style).

If you have an existing saved application from GATE version 8.4.1 or earlier it
will be necessary to ``upgrade'' it to use the new core plugins.  An upgrade
tool is provided on the ``Tools'' menu of GATE Developer, and is described in
section Section~\ref{sec:developer:convertxgapp}.

\rcSubsect{For developers}

As part of this release, GATE development has moved from SourceForge to GitHub
-- bug reports, patches and feature requests should now use the GitHub issue
tracker as described in section~\ref{sec:development:bugs}.

% vim:ft=tex
