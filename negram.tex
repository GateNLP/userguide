%%%%%%%%%%%%%%%%%%%%%%%%%%%%%%%%%%%%%%%%%%%%%%%%%%%%%%%%%%%%%%%%%%%%%%%%%%%%%
%
% negram.tex
%
% diana, 8/1
%
% $Id: negram.tex,v 1.9 2005/02/10 17:08:59 ian Exp $
%
%%%%%%%%%%%%%%%%%%%%%%%%%%%%%%%%%%%%%%%%%%%%%%%%%%%%%%%%%%%%%%%%%%%%%%%%%%%%%


%%%%%%%%%%%%%%%%%%%%%%%%%%%%%%%%%%%%%%%%%%%%%%%%%%%%%%%%%%%%%%%%%%%%%%%%%%%%%
\chapt[chap:negram]{Named-Entity State Machine Patterns}
\markboth{Named-Entity State Machine Patterns}{Named-Entity State Machine
Patterns}
%%%%%%%%%%%%%%%%%%%%%%%%%%%%%%%%%%%%%%%%%%%%%%%%%%%%%%%%%%%%%%%%%%%%%%%%%%%%%

%%%% qqqqqqqqqqqqqqqqqqqqqqqqq %%%%
\begin{quote}
There are, it seems to me, two basic reasons why minds aren't computers...
The first... is that human beings are organisms. Because
of this we have all sorts of needs - for food, shelter, clothing, sex
etc - and capacities - for locomotion, manipulation, articulate
speech etc, and so on - to which there are no real analogies in computers.
These needs and capacities underlie and interact with our
mental activities. This is important, not simply because we can't
understand how humans behave except in the light of these needs
and capacities, but because any historical explanation of how human
mental life developed can only do so by looking at how this
process interacted with the evolution of these needs and capacities
in successive species of hominids. 

\ldots

The second reason... is that... brains don't work like computers. 

{\it Minds, Machines and Evolution}, Alex Callinicos, 1997 (ISJ 74, p.103).
\end{quote}
%%%% qqqqqqqqqqqqqqqqqqqqqqqqq %%%%


This \chapthing\ describes the individual grammars used in GATE for Named
Entity Recognition, and how they are combined together. It relates to
the default NE grammar for ANNIE, but should also provide guidelines
for those adapting or creating new grammars. For documentation about
specific grammars other than this core set, use this document in
combination with the comments in the relevant grammar
files. \chapthing\ \ref{chap:jape} also provides information
about designing new grammar rules and tips for ensuring maximum
processing speed.


\sect{Main.jape}

This file contains a list of the grammars to be used, in the
correct processing order. The ordering of the grammars is crucial,
because they are processed in series, and later grammars may depend on
annotations produced by earlier grammars.

The default grammar consists of the following phases:
\begin{itemize}
\item first.jape
\item firstname.jape
\item name.jape
\item name\_post.jape
\item date\_pre.jape
\item date.jape
\item reldate.jape
\item number.jape
\item address.jape
\item url.jape
\item identifier.jape
\item jobtitle.jape
\item final.jape
\item unknown.jape
\item name\_context.jape
\item org\_context.jape
\item loc\_context.jape
\item clean.jape
\end{itemize}


\sect{first.jape}

This grammar must always be processed first. It can contain any
general macros needed for the whole grammar set. This should consist
of a macro defining how space and control characters are to be
processed (and may consequently be different for each grammar set,
depending on the text type). Because this is defined first of all, it
is not necessary to restate this in later grammars. This has a big
advantage -- it means that default grammars can be used for specialised
grammar sets, without having to be adapted to deal with e.g. different
treatment of spaces and control characters. In this way, only the
first.jape file needs to be changed for each grammar set, rather than
every individual grammar.

The first.jape grammar also has a dummy rule in. This is never
intended to fire -- it is simply added because every grammar set must
contain rules, but there are no specific rules we wish to add
here. Even if the rule were to match the pattern defined, it is
designed not to produce any output (due to the empty RHS).

\sect{firstname.jape}
This grammar contains rules to identify first names and titles via the
gazetteer lists. It adds a gender feature where appropriate from the
gazetteer list. This gender feature is used later in order to improve
co-reference between names and pronouns. The grammar creates
separate annotations of type FirstPerson and Title.

\sect{name.jape}

This grammar contains initial rules for organization, location and
person entities. These rules all create temporary annotations, some of
which will be discarded later, but the majority of which will be
converted into final annotations in later grammars. Rules beginning
with `Not' are negative rules -- this means that we detect something
and give it a special annotation (or no annotation at all) in order to
prevent it being recognised as a name. This is because we have no
negative operator (we have `=' but not `!=').

\subsect{Person}

We first define macros for initials, first names, surnames, and
endings. We then use these to recognise combinations of first names
from the previous phase, and surnames from their POS tags
or case information. Persons get marked with the
annotation `TempPerson'. We also percolate feature information about
the gender from the previous annotations if known.

\subsect{Location}

The rules for Location are fairly straightforward, but we define them
in this grammar so that any ambiguity can be resolved at the top
level. Locations are often combined with other entity types, such as
Organisations. This is dealt with by annotating the two entity types
separately, and them combining them in a later phase. Locations are
recognised mainly by gazetteer lookup, using not only lists of known
places, but also key words such as mountain, lake, river, city
etc. Locations are annotated as TempLocation in this phase.

\subsect{Organization}

Organizations tend to be defined either by straight lookup from
the gazetteer lists, or, for the majority, by a combination of POS or
case information and key words such as `company', `bank',
`Services' `Ltd.' etc. Many organizations are also identified by
contextual information in the later phase org\_context.jape. In this
phase, organizations are annotated as TempOrganization.

\subsect{Ambiguities}

Some ambiguities are resolved immediately in this grammar, while
others are left until later phases. For example, a Christian name
followed by a possible Location is resolved by default to a person rather than
a Location (e.g. `Ken London'). On the other hand, a Christian name
followed by a possible organisation ending is resolved to an
Organisation (e.g. `Alexandra Pottery'), though this is a slightly
less sure rule.

\subsect{Contextual information}
Although most of the rules involving contextual information are
invoked in a much later phase, there are a few which are invoked here,
such as `X joined Y' where X is annotated as a Person and Y as an
Organization. This is so that both annotations types can be handled at once.

\sect{name\_post.jape}
This grammar runs after the name grammar to fix some erroneous
annotations that may have been created. Of course, a more elegant
solution would be not to create the problem in the first instance, but
this is a workaround. For example, if the surname of a Person contains
certain stop words, e.g. `Mary And' then only the first name should be
recognised as a Person. However, it might be that the firstname is
also an Organization (and has been tagged with TempOrganization
already), e.g. `U.N.' If this is the case, then the annotation is left
untouched, because this is correct.

\sect{date\_pre.jape}
This grammar precedes the date phase, because it includes extra context
to prevent dates being recognised erroneously in the middle of longer
expressions. It mainly treats the case where an expression is already
tagged as a Person, but could also be tagged as a date (e.g. 16th
Jan).

\sect{date.jape}
This grammar contains the base rules for recognising times and
dates. Given the complexity of potential patterns representing such
expressions, there are a large number of rules and macros.

Although times and dates can be mutually ambiguous, we try to
distinguish between them as early as possible. Dates, times and years
are generally tagged separately (as TempDate, TempTime and TempYear
respectively) and then recombined to form a final Date annotation in a
later phase. This is because dates, times and years can be combined
together in many different ways, and also because there can be much
ambiguity between the three. For example, 1312 could be a time or a
year, while 9-10 could be a span of time or date, or a fixed time
or date.

\sect{reldate.jape}
This grammar handles relative rather than absolute date and time
sequences, such as `yesterday morning', `2 hours ago', `the first
9 months of the financial year'etc. It uses mainly explicit key words
such as `ago' and items from the gazetteer lists.


\sect{number.jape}

This grammar covers rules concerning money and percentages.
The rules are fairly straightforward, using keywords from the
gazetteer lists, and there is little ambiguity here, except for
example where `Pound' can be money or weight, or where there is no
explicit currency denominator.

 
\sect{address.jape}

Rules for Address cover ip addresses, phone and fax
numbers, and postal addresses. In general, these are not highly ambiguous, and can be
covered with simple pattern matching, although phone numbers can
require use of contextual information. Currently only UK formats are
really handled, though handling of foreign zipcodes and
phone number formats is envisaged in future. The annotations produced
are of type Email, Phone etc. and are then replaced in a later phase
with final Address annotations with `phone' etc. as features.

\sect{url.jape}
Rules for email addresses and Urls are in a separate grammar from the
other address types, for the simple reason that SpaceTokens need to be
identified for these rules to operate, whereas this is not necessary
for the other Address types. For speed of processing, we place them in
separate grammars so that SpaceTokens can be eliminated from the Input
when they are not required.

\sect{identifier.jape}

This grammar identifies `Identifiers' which basically means any
combination of numbers and letters acting as an ID, reference number
etc. not recognised as any other entity type. 

\sect{jobtitle.jape}

This grammar simply identifies Jobtitles from the gazetteer lists, and
adds a JobTitle annotation, which is used in later phases to aid
recognition of other entity types such as Person and Organization. It
may then be discarded in the Clean phase if not required as a final
annotation type.


\sect{final.jape}

This grammar uses the temporary annotations previously assigned in
the earlier phases, and converts them into final annotations. The
reason for this is that we need to be able to resolve ambiguities
between different entity types, so we need to have all the different
entity types handled in a single grammar somewhere. Ambiguities can be
resolved using prioritisation techniques. Also, we may need to combine
previously annotated elements, such as dates and times, into a single
entity.

The rules in this grammar use Java code on the RHS to
remove the existing temporary annotations, and replace them with new
annotations. This is because we want to retain the features associated
with the temporary annotations. For example, we might need to keep
track of whether a person is male or female, or whether a location is
a city or country. It also enables us to keep track of which rules
have been used, for debugging purposes.

For the sake of obfuscation, although this phase is called final, it
is not the final phase!

\sect{unknown.jape}

This short grammar finds proper nouns not previously recognised,
and gives them an Unknown annotation. This is then used by the
namematcher -- if an Unknown annotation can be matched with a
previously categorised entity, its annotation is changed to that of
the matched entity. Any remaining Unknown annotations are useful
for debugging purposes, and can also be used as input for additional
grammars or processing resources.


\sect{name\_context.jape}

This grammar looks for Unknown annotations occurring in certain
contexts which indicate they might belong to Person. This is a typical
example of a grammar that would benefit from learning or automatic
context generation, because useful contexts are (a) hard to find
manually and may require large volumes of training data, and (b) often
very domain--specific. In this core grammar, we confine the use of
contexts to fairly general uses, since this grammar should not be
domain--dependent.

\sect{org\_context.jape}

This grammar operates on a similar principle to name\_context.jape. It
is slightly oriented towards business texts, so does not quite fulfil
the generality criteria of the previous grammar. It does, however,
provide some insight into more detailed use of contexts.</p>

\sect{loc\_context.jape}

This grammar also operates in a similar manner to the preceding two,
using general context such as coordinated pairs of locations, and
hyponymic types of information.

\sect{clean.jape}

This grammar comes last of all, and simply aims to clean up (remove)
some of the temporary annotations that may not have been deleted along
the way.

%\section{Appendix - Default Grammars}
\subsection{main.jape}
\begin{verbatim}

/*
*  main.jape
*
* Copyright (c) 1998-2001, The University of Sheffield.
*
*  This file is part of GATE (see http://gate.ac.uk/), and is free
*  software, licenced under the GNU Library General Public License,
*  Version 2, June 1991 (in the distribution as file licence.html,
*  and also available at http://gate.ac.uk/gate/licence.html).
*
*  Diana Maynard, 02 Aug 2001
*
*  $Id: grammars.tex,v 1.2 2001/11/29 18:07:27 diana Exp $
*/

MultiPhase:	TestTheGrammars
Phases: 
first
firstname
name
name_post
date_pre
date
reldate
number
address
url
identifier
jobtitle
final
unknown
name_context
org_context
loc_context
clean/*

*  first.jape
*
* Copyright (c) 1998-2001, The University of Sheffield.
*
*  This file is part of GATE (see http://gate.ac.uk/), and is free
*  software, licenced under the GNU Library General Public License,
*  Version 2, June 1991 (in the distribution as file licence.html,
*  and also available at http://gate.ac.uk/gate/licence.html).
*
*  Diana Maynard, 10 Sep 2001
* 
*  $Id: grammars.tex,v 1.2 2001/11/29 18:07:27 diana Exp $
*/

Phase:	First
Input: Token Lookup
Options: control = appelt

// this has to be run first of all 
// contains any macros etc needed only for standard grammars

//////////////////////////////////////////////////////////////
Macro: SPACE
// space
// control
// space control
// control space

( 
 ({SpaceToken.kind == space}
  ({SpaceToken.kind == control})?
  ({SpaceToken.kind == control})?
 )
|
 ({SpaceToken.kind == control}
  ({SpaceToken.kind == control})?
  ({SpaceToken.kind == space})?
 )
)


///////////////////////////////////////////////////////////////

Rule: Silly
// we have to have a rule here, so we'll just have something silly

(
 {Token.string == "afguahughaegarth"}
)
:silly
-->
 {}

/*
*  firstname.jape
*
* Copyright (c) 1998-2001, The University of Sheffield.
*
*  This file is part of GATE (see http://gate.ac.uk/), and is free
*  software, licenced under the GNU Library General Public License,
*  Version 2, June 1991 (in the distribution as file licence.html,
*  and also available at http://gate.ac.uk/gate/licence.html).
*
*  Diana Maynard, 02 Aug 2001
* 
*  $Id: grammars.tex,v 1.2 2001/11/29 18:07:27 diana Exp $
*/

Phase:	FirstName
Input: Token Lookup
Options: control = appelt

Rule: FirstName
// Fred

(
 {Lookup.majorType == person_first}
):person
-->
{
gate.AnnotationSet person = (gate.AnnotationSet)bindings.get("person");
gate.Annotation personAnn = (gate.Annotation)person.iterator().next();
gate.FeatureMap features = Factory.newFeatureMap();
features.put("gender", personAnn.getFeatures().get("minorType"));
features.put("rule", "FirstName");
annotations.add(person.firstNode(), person.lastNode(), "FirstPerson",
features);
}

Rule: TitleGender
Priority: 50
// Mr

(
 {Lookup.majorType == title, Lookup.minorType == male}|
 {Lookup.majorType == title, Lookup.minorType == female}
):person
-->
{
gate.AnnotationSet person = (gate.AnnotationSet)bindings.get("person");
gate.Annotation personAnn = (gate.Annotation)person.iterator().next();
gate.FeatureMap features = Factory.newFeatureMap();
features.put("gender", personAnn.getFeatures().get("minorType"));
features.put("rule", "TitleGender");
annotations.add(person.firstNode(), person.lastNode(), "Title",
features);
}

Rule: Title
// Dr

(
 {Lookup.majorType == title}
):person
-->
 :person.Title = {rule = "Title"}



/*
*  name.jape
*
* Copyright (c) 1998-2001, The University of Sheffield.
*
*  This file is part of GATE (see http://gate.ac.uk/), and is free
*  software, licenced under the GNU Library General Public License,
*  Version 2, June 1991 (in the distribution as file licence.html,
*  and also available at http://gate.ac.uk/gate/licence.html).
*
*  Diana Maynard, 10 Sep 2001
* 
*  $Id: grammars.tex,v 1.2 2001/11/29 18:07:27 diana Exp $
*/


Phase:	Name
Input: Token Lookup Title FirstPerson
Options: control = appelt

///////////////////////////////////////////////////////////////

// Person Rules

/////////////////////////////////////////////////////////////////
Macro: TITLE
(
 {Title}
 ({Token.string == "."})?
)
Macro: INITIALS
(
  ({Token.orth == upperInitial, Token.length =="1"}
  ({Token.string == "."})?
  )+
)

Macro: INITIALS2

(
 {Token.orth == allCaps, Token.length == "2"} |
 {Token.orth == allCaps, Token.length == "3"}
)


Macro: FIRSTNAME
(
 ({FirstPerson.gender == male} |
  {FirstPerson.gender == female})
 |
 (INITIALS)
)

Macro: FIRSTNAMEAMBIG
(
 {Lookup.majorType == person_first, Lookup.minorType == ambig}
)



Macro: UPPER
(
 ({Token.category == NNP}| 
 {Token.orth == upperInitial}|
 {Token.orth == mixedCaps} 
)
 ({Token.string == "-"}
  {Token.category == NNP}
 )?
)

Macro: PERSONENDING
(
 {Lookup.majorType == person_ending}
)

Macro: PREFIX
(
 ({Lookup.majorType == surname, Lookup.minorType == prefix}
 )|
 (({Token.string == "O"}|{Token.string == "D"})
  {Token.string == "'"}
 )
)




///////////////////////////////////////////////////////////


// Person Rules

Rule: Pronoun
Priority: 1000
//stops personal pronouns being recognised as Initials
(
 {Token.category == PP}|
 {Token.category == PRP}|
 {Token.category == RB}
):pro
-->
{}

 

Rule:	GazPerson
Priority: 50
(
 {Lookup.majorType == person_full, Lookup.minorType == normal}
)
:person -->
{
gate.AnnotationSet person = (gate.AnnotationSet)bindings.get("person");
gate.Annotation personAnn = (gate.Annotation)person.iterator().next();
gate.FeatureMap features = Factory.newFeatureMap();
features.put("kind", "personName");
features.put("rule", "GazPerson");
annotations.add(person.firstNode(), person.lastNode(), "TempPerson",
features);
}

Rule:	TheGazPersonFirst
Priority: 200
(
 {Token.category == DT}|
 {Token.category == PRP}|
 {Token.category == RB}
)
(
 {FirstPerson}
)
:person 
( 
 {Token.orth == upperInitial, Token.length == "1"}
)?
-->
{
gate.AnnotationSet person = (gate.AnnotationSet)bindings.get("person");
gate.Annotation personAnn = (gate.Annotation)person.iterator().next();
gate.FeatureMap features = Factory.newFeatureMap();
features.put("gender", personAnn.getFeatures().get("gender"));
features.put("kind", "personName");
features.put("rule", "GazPersonFirst");
annotations.add(person.firstNode(), person.lastNode(), "TempPerson",
features);
//annotations.removeAll(person);
}


Rule:	GazPersonFirst
Priority: 20
(
 {FirstPerson}
)
:person 
( 
 {Token.orth == upperInitial, Token.length == "1"}
)?
-->
{
gate.AnnotationSet person = (gate.AnnotationSet)bindings.get("person");
gate.Annotation personAnn = (gate.Annotation)person.iterator().next();
gate.FeatureMap features = Factory.newFeatureMap();
features.put("gender", personAnn.getFeatures().get("gender"));
features.put("kind", "personName");
features.put("rule", "GazPersonFirst");
annotations.add(person.firstNode(), person.lastNode(), "TempPerson",
features);
//annotations.removeAll(person);
}





Rule: PersonFirstContext
Priority: 30
// Anne and Kenton

(
 {FirstPerson}
):person1
(
 {Token.string == "and"}
)
({Token.orth == upperInitial})
:person2
 -->
{
//first deal with person1
 gate.FeatureMap features1 = Factory.newFeatureMap();
gate.AnnotationSet person1Set = (gate.AnnotationSet)bindings.get("person1");
gate.AnnotationSet firstPerson = (gate.AnnotationSet)person1Set.get("FirstPerson");
if (firstPerson != null && firstPerson.size()>0)
{
  gate.Annotation personAnn = (gate.Annotation)firstPerson.iterator().next();
  features1.put("gender", personAnn.getFeatures().get("gender"));
}
  features1.put("kind", "personName");
  features1.put("rule", "PersonFirstContext");
annotations.add(person1Set.firstNode(), person1Set.lastNode(), "TempPerson",
features1);
//now deal with person2
gate.FeatureMap features2 = Factory.newFeatureMap();
gate.AnnotationSet person2Set = (gate.AnnotationSet)bindings.get("person2");
  features2.put("kind", "personName");
  features2.put("rule", "PersonFirstContext");
annotations.add(person2Set.firstNode(), person2Set.lastNode(), "TempPerson",
features2);
}


Rule: PersonFirstContext2
Priority: 40
// Anne and I

(
 {FirstPerson}
):person
(
 {Token.string == "and"}
 {Token.length == "1"}
)
 -->
{
 gate.FeatureMap features = Factory.newFeatureMap();
gate.AnnotationSet personSet = (gate.AnnotationSet)bindings.get("person");
gate.AnnotationSet firstPerson = (gate.AnnotationSet)personSet.get("FirstPerson");
if (firstPerson != null && firstPerson.size()>0)
{
  gate.Annotation personAnn = (gate.Annotation)firstPerson.iterator().next();
  features.put("gender", personAnn.getFeatures().get("gender"));
}
  features.put("kind", "personName");
  features.put("rule", "PersonFirstContext2");
annotations.add(personSet.firstNode(), personSet.lastNode(), "TempPerson",
features);
}


Rule:	PersonTitle
Priority: 35
// Mr. Jones
// Mr Fred Jones
// note we only allow one first and surname, 
// but we can add more in a final phase if we find adjacent unknowns
 
(
 (TITLE)+
 ((FIRSTNAME | FIRSTNAMEAMBIG | INITIALS2)
 )?
  (PREFIX)* 
  (UPPER)
 (PERSONENDING)?
)
:person -->
{
 gate.FeatureMap features = Factory.newFeatureMap();
gate.AnnotationSet personSet = (gate.AnnotationSet)bindings.get("person");
  
 // get all Title annotations that have a gender feature
 HashSet fNames = new HashSet();
    fNames.add("gender");
    gate.AnnotationSet personTitle = personSet.get("Title", fNames);

// if the gender feature exists
 if (personTitle != null && personTitle.size()>0)
{
  //Out.prln("Titles found " +  personTitle);
  gate.Annotation personAnn = (gate.Annotation)personTitle.iterator().next();
  features.put("gender", personAnn.getFeatures().get("gender"));
}
else
{
  //get all firstPerson annotations that have a gender feature
  //  HashSet fNames = new HashSet();
   // fNames.add("gender");
    gate.AnnotationSet firstPerson = personSet.get("FirstPerson", fNames);

  if (firstPerson != null && firstPerson.size()>0)
 {
    //Out.prln("First persons found " +  firstPerson);
  gate.Annotation personAnn = (gate.Annotation)firstPerson.iterator().next();
  features.put("gender", personAnn.getFeatures().get("gender"));
 }
}
  features.put("kind", "personName");
  features.put("rule", "PersonTitle");
annotations.add(personSet.firstNode(), personSet.lastNode(), "TempPerson",
features);
}


Rule:	PersonFirstTitleGender
Priority: 55
// use this rule when we know what gender the title indicates
// Mr Fred

(
 ({Title.gender == male} | {Title.gender == female})
 ((FIRSTNAME | FIRSTNAMEAMBIG | INITIALS2)
 )
)
:person -->
{
 gate.FeatureMap features = Factory.newFeatureMap();
gate.AnnotationSet personSet = (gate.AnnotationSet)bindings.get("person");
gate.AnnotationSet title = (gate.AnnotationSet)personSet.get("Title");
if (title != null && title.size()>0)
{
  gate.Annotation personAnn = (gate.Annotation)title.iterator().next();
  features.put("gender", personAnn.getFeatures().get("gender"));
}
  features.put("kind", "personName");
  features.put("rule", "PersonFirstTitleGender");
annotations.add(personSet.firstNode(), personSet.lastNode(), "TempPerson",
features);
}


Rule: PersonTitleGender
Priority: 18
// use this rule if the title has a feature gender
// Miss F Smith
(
 ({Title.gender == male}|
  {Title.gender == female}
 ) 
 ((FIRSTNAME | FIRSTNAMEAMBIG | INITIALS2)
 )*
 (UPPER)
 (PERSONENDING)?
)
:person -->
{
 gate.FeatureMap features = Factory.newFeatureMap();
gate.AnnotationSet personSet = (gate.AnnotationSet)bindings.get("person");
gate.AnnotationSet title = (gate.AnnotationSet)personSet.get("Title");
// if the annotation type title doesn't exist, do nothing
if (title != null && title.size()>0)
{
// if it does exist, take the first element in the set
  gate.Annotation personAnn = (gate.Annotation)title.iterator().next();
//propagate gender feature (and value) from title
  features.put("gender", personAnn.getFeatures().get("gender"));
}
// create some new features
  features.put("kind", "personName");
  features.put("rule", "PersonTitleGender");
// create a TempPerson annotation and add the features we've created
annotations.add(personSet.firstNode(), personSet.lastNode(), "TempPerson",
features);
}


Rule: PersonJobTitle
Priority: 20
// note we include titles but not jobtitles in markup

(
 {Lookup.majorType == jobtitle}
):jobtitle
(
 (TITLE)?
 ((FIRSTNAME | FIRSTNAMEAMBIG | INITIALS2)
 )
 (PREFIX)* 
 (UPPER)
 (PERSONENDING)?
)
:person 
-->
    :person.TempPerson = {kind = "personName", rule = "PersonJobTitle"},
   :jobtitle.JobTitle = {rule = "PersonJobTitle"} 




Rule: NotFirstPersonStop
Priority: 50
// ambig first name and surname is stop word
// e.g. Will And

(
 ((FIRSTNAMEAMBIG)+ | 
  {Token.category == PRP}|
  {Token.category == DT}
 )
 ({Lookup.majorType == stop}
 )
)
:person -->
  {}


Rule: NotPersonFull
Priority: 50
// Det + Surname
(
 {Token.category == DT}|
 {Token.category == PRP}|
 {Token.category == RB}
)
(
 (PREFIX)* 
 (UPPER)
 (PERSONENDING)?
):foo
-->
{}


Rule: LocPersonAmbig
Priority: 50
// Location + Surname
(
 {Lookup.majorType == location}
):loc
(
 (PREFIX)* 
 (UPPER)
 (PERSONENDING)?
):foo
-->
:loc.TempLocation = {kind = "locName", rule = LocPersonAmbig}


Rule: 	PersonFull
Priority: 10
// F.W. Jones
// Fred Jones
(
 {Token.category == DT}
)?
(
 ((FIRSTNAME | FIRSTNAMEAMBIG) )+
 (PREFIX)*
 (UPPER)
 (PERSONENDING)?
)
:person -->
{
 gate.FeatureMap features = Factory.newFeatureMap();
gate.AnnotationSet personSet = (gate.AnnotationSet)bindings.get("person");
  
  //get all firstPerson annotations that have a gender feature
    HashSet fNames = new HashSet();
    fNames.add("gender");
    gate.AnnotationSet firstPerson = personSet.get("FirstPerson", fNames);

  if (firstPerson != null && firstPerson.size()>0)
 {
    //Out.prln("First persons found " +  firstPerson);
  gate.Annotation personAnn = (gate.Annotation)firstPerson.iterator().next();
  features.put("gender", personAnn.getFeatures().get("gender"));
}
  features.put("kind", "personName");
  features.put("rule", "PersonFull");
annotations.add(personSet.firstNode(), personSet.lastNode(), "TempPerson",
features);
}

Rule: PersonFullStop
Priority: 50
// G.Wilson Fri

(
 ((FIRSTNAME | FIRSTNAMEAMBIG) )
 (PREFIX)* 
 (UPPER)
):person
(
 {Lookup.majorType == date}
)
-->
 :person.TempPerson = {kind = "personName", rule = "PersonFullStop"}


Rule: NotPersonFullReverse
Priority: 20
// XYZ, I
(
 (UPPER)
 {Token.string == ","}
 {Token.category == PRP}
 (PERSONENDING)?
)
:unknown 
-->
{}

Rule: 	PersonFullReverse
Priority: 5
// Jones, F.W.
// don't allow Jones, Fred because too ambiguous
// Smith, TF

(
 {Token.category ==NNP}
 {Token.string == ","}
 (INITIALS )+ 
 (PERSONENDING)?
)
:person -->
{
 gate.FeatureMap features = Factory.newFeatureMap();
gate.AnnotationSet personSet = (gate.AnnotationSet)bindings.get("person");
gate.AnnotationSet firstPerson = (gate.AnnotationSet)personSet.get("FirstPerson");
if (firstPerson != null && firstPerson.size()>0)
{
  gate.Annotation personAnn = (gate.Annotation)firstPerson.iterator().next();
  features.put("gender", personAnn.getFeatures().get("gender"));
}
  features.put("kind", "personName");
  features.put("rule", "PersonFullReverse");
annotations.add(personSet.firstNode(), personSet.lastNode(), "TempPerson",
features);
}


Rule:  PersonSaint
Priority: 50
// Note: ensure that it's not a Saints Day first
(
 ({Token.string == "St"} ({Token.string == "."})? |
 {Token.string == "Saint"})
 (FIRSTNAME)
 )
:person -->
{
 gate.FeatureMap features = Factory.newFeatureMap();
gate.AnnotationSet personSet = (gate.AnnotationSet)bindings.get("person");
gate.AnnotationSet firstPerson = (gate.AnnotationSet)personSet.get("FirstPerson");
if (firstPerson != null && firstPerson.size()>0)
{
  gate.Annotation personAnn = (gate.Annotation)firstPerson.iterator().next();
  features.put("gender", personAnn.getFeatures().get("gender"));
}
  features.put("kind", "personName");
  features.put("rule", "PersonSaint");
annotations.add(personSet.firstNode(), personSet.lastNode(), "TempPerson",
features);
}


Rule: PersonLocAmbig
Priority: 40
// Ken London
// Susan Hampshire

// Christian name + Location --> Person's Name
(
 ({Lookup.majorType == person_first} |
  (INITIALS
   {Token.string == "."})
 )
  {Lookup.majorType == location}
)
:person -->
{
 gate.FeatureMap features = Factory.newFeatureMap();
gate.AnnotationSet personSet = (gate.AnnotationSet)bindings.get("person");
gate.AnnotationSet firstPerson = (gate.AnnotationSet)personSet.get("FirstPerson");
if (firstPerson != null && firstPerson.size()>0)
{
  gate.Annotation personAnn = (gate.Annotation)firstPerson.iterator().next();
  features.put("gender", personAnn.getFeatures().get("gender"));
}
  features.put("kind", "personName");
  features.put("rule", "PersonLocAmbig");
annotations.add(personSet.firstNode(), personSet.lastNode(), "TempPerson",
features);
}

///////////////////////////////////////////////////////////////////
// Organisation Rules

Macro:  CDG
// cdg is something like "Ltd."
 (
  ({Lookup.majorType == cdg})|
  ({Token.string == ","} 
  {Lookup.majorType == cdg})
 )


Macro: SAINT
(
 ({Token.string == "St"} ({Token.string == "."})? |
 {Token.string == "Saint"})
)

Macro: CHURCH
(
{Token.string == "Church"}|{Token.string == "church"}|
{Token.string == "Cathedral"}|{Token.string == "cathedral"}|
{Token.string == "Chapel"}|{Token.string == "chapel"}
)

/////////////////////////////////////////////////////////////
Rule:	TheGazOrganization
Priority: 245
(
 {Token.category == DT}
)
(
{Lookup.majorType == organization}
)
:orgName -->  
 {
 gate.FeatureMap features = Factory.newFeatureMap();
// create an annotation set consisting of all the annotations for org 
gate.AnnotationSet orgSet = (gate.AnnotationSet)bindings.get("orgName");
// create an annotation set consisting of the annotation matching Lookup
gate.AnnotationSet org = (gate.AnnotationSet)orgSet.get("Lookup");
// if the annotation type Lookup doesn't exist, do nothing
if (org != null && org.size()>0)
{
// if it does exist, take the first element in the set
  gate.Annotation orgAnn = (gate.Annotation)org.iterator().next();
//propagate minorType feature (and value) from org
  features.put("orgType", orgAnn.getFeatures().get("minorType"));
}
// create some new features
  features.put("rule", "GazOrganization");
// create a TempOrg annotation and add the features we've created
annotations.add(orgSet.firstNode(), orgSet.lastNode(), "TempOrganization",
features);
}


Rule:	GazOrganization
Priority: 145

(
{Lookup.majorType == organization}
)
:orgName -->  
 {
 gate.FeatureMap features = Factory.newFeatureMap();
// create an annotation set consisting of all the annotations for org 
gate.AnnotationSet orgSet = (gate.AnnotationSet)bindings.get("orgName");
// create an annotation set consisting of the annotation matching Lookup
gate.AnnotationSet org = (gate.AnnotationSet)orgSet.get("Lookup");
// if the annotation type Lookup doesn't exist, do nothing
if (org != null && org.size()>0)
{
// if it does exist, take the first element in the set
  gate.Annotation orgAnn = (gate.Annotation)org.iterator().next();
//propagate minorType feature (and value) from org
  features.put("orgType", orgAnn.getFeatures().get("minorType"));
}
// create some new features
  features.put("rule", "GazOrganization");
// create a TempOrg annotation and add the features we've created
annotations.add(orgSet.firstNode(), orgSet.lastNode(), "TempOrganization",
features);
}

Rule:	LocOrganization
Priority: 50
// Ealing Police
(
 ({Lookup.majorType == location} |
  {Lookup.majorType == country_adj})
{Lookup.majorType == organization}
({Lookup.majorType == organization})?
)
:orgName -->  
  :orgName.TempOrganization = {kind = "orgName", rule=LocOrganization}


Rule:	INOrgXandY
Priority: 200

// Bradford & Bingley
// Bradford & Bingley Ltd
(
 {Token.category == IN}
)

(
 ({Token.category == NNP}
  )+

 {Token.string == "&"}

 (
  {Token.orth == upperInitial}
 )+

 (CDG)?

)
:orgName -->
  :orgName.TempOrganization = {kind = "unknown", rule = "OrgXandY"}

Rule:	OrgXandY
Priority: 20

// Bradford & Bingley
// Bradford & Bingley Ltd


(
 ({Token.category == NNP}
  )+

 {Token.string == "&"}

 (
  {Token.orth == upperInitial}
 )+

 (CDG)?

)
:orgName -->
  :orgName.TempOrganization = {kind = "unknown", rule = "OrgXandY"}


Rule:OrgUni
Priority: 25
// University of Sheffield
// Sheffield University
// A Sheffield University
(
 {Token.string == "University"}
 {Token.string == "of"}
 (
 {Token.category == NNP})+
)
:orgName -->
  :orgName.TempOrganization = {kind = "org", rule = "OrgDept"}



Rule: OrgDept
Priority: 25
// Department of Pure Mathematics and Physics

(
 {Token.string == "Department"}
 
 {Token.string == "of"}
 (
 {Token.orth == upperInitial})+
 (
  {Token.string == "and"}
  ( 
   {Token.orth == upperInitial})+
 )?
)
:orgName -->
  :orgName.TempOrganization = {kind = "department", rule = "OrgDept"}

Rule:	TheOrgXKey
Priority: 500

// The Aaaa Ltd.
(
 {Token.category == DT}
)
(
  (UPPER)
  (UPPER)?
  (UPPER)?
  (UPPER)?
  (UPPER)?
 {Lookup.majorType == org_key}
 ({Lookup.majorType == org_ending})?
)
:org
-->
:org.TempOrganization = {kind = "unknown", rule = "TheOrgXKey"}

Rule:	OrgXKey
Priority: 125

// Aaaa Ltd.
(
  (UPPER)
  (UPPER)?
  (UPPER)?
  (UPPER)?
  (UPPER)?
 {Lookup.majorType == org_key}
 ({Lookup.majorType == org_ending})?
)
:org
-->
:org.TempOrganization = {kind = "unknown", rule = "TheOrgXKey"}


Rule: NotOrgXEnding
Priority: 500
// Very Limited

(
 {Token.category == DT}
)?
(
 {Token.category == RB}
 {Lookup.majorType == cdg}
)
:label
-->
{}
 
 Rule:	NotOrgXEnding2
Priority: 500

// The Coca Cola Co.

(
 {Token.category == DT}
)
(
  (UPPER)
  (UPPER)?
 {Lookup.majorType == cdg}
)
:orgName -->
  :orgName.TempOrganization = {kind = "unknown", rule = "OrgXEnding"}



Rule:	OrgXEnding
Priority: 120

// Coca Cola Co.

(
  (UPPER)
  (UPPER)?
 {Lookup.majorType == cdg}
)
:orgName -->
  :orgName.TempOrganization = {kind = "unknown", rule = "OrgXEnding"}

Rule:	TheOrgXandYKey
Priority: 220

(
 {Token.category == DT}
)
(
 (UPPER)
 (UPPER)?
  (({Token.string == "and"} | 
    {Token.string == "&"})
   (UPPER)?
   (UPPER)?
   (UPPER)?
  )
 {Lookup.majorType == org_key}
 ({Lookup.majorType == org_ending})?
)
:orgName -->
  :orgName.TempOrganization = {kind = "unknown", rule = "OrgXandYKey"}



Rule:	OrgXandYKey
Priority: 120

// Aaaa Ltd.
// Xxx Services Ltd. 
// AA and BB Services Ltd.
// but NOT A XXX Services Ltd.

(
 (UPPER)
 (UPPER)?
  (({Token.string == "and"} | 
    {Token.string == "&"})
   (UPPER)?
   (UPPER)?
   (UPPER)?
  )
 {Lookup.majorType == org_key}
 ({Lookup.majorType == org_ending})?
)
:orgName -->
  :orgName.TempOrganization = {kind = "unknown", rule = "OrgXandYKey"}


Rule:	OrgXsKeyBase
Priority: 120
 
// Gandy's Circus
// Queen's Ware

(
  (UPPER)?
  (UPPER)?
  ({Token.orth == upperInitial}
   {Token.string == "'"}
   ({Token.string == "s"})?
  )
 ({Lookup.majorType == org_key}|
  {Lookup.majorType == org_base})
)
:orgName -->
  :orgName.TempOrganization = {kind = "org", rule = "OrgXsKeybase"}



Rule: NotOrgXBase
Priority: 1000
// not things like British National
// or The University


(
 ({Token.category == DT} 
 )?
)
(
 ({Lookup.majorType == country_adj}|
  {Token.orth == lowercase})
 ({Lookup.majorType == org_base}|
  {Lookup.majorType == govern_key})
)
:orgName -->
  :orgName.Temp = {kind = "notorgName", rule = "NotOrgXBase"}


Rule:	TheOrgXBase
Priority: 230

(
 ({Token.category == DT}
 )
)
(
 (
  (UPPER)|
  {Lookup.majorType == organization}
 )
 (UPPER)?
 (UPPER)?
 ({Lookup.majorType == org_base}|
  {Lookup.majorType == govern_key}
 )
 (
  {Token.string == "of"}
  (UPPER)
  (UPPER)?
  (UPPER)?
 )?
)
:orgName -->
  :orgName.TempOrganization = {kind = "unknown", rule = "OrgXBase"}


Rule:	OrgXBase
Priority: 130

// same as OrgXKey but uses base instead of key
// includes govern_key e.g. academy
// Barclays Bank
// Royal Academy of Art

(
 (
  (UPPER)|
  {Lookup.majorType == organization}
 )
 (UPPER)?
 (UPPER)?
 ({Lookup.majorType == org_base}|
  {Lookup.majorType == govern_key}
 )
 (
  {Token.string == "of"}
  (UPPER)
  (UPPER)?
  (UPPER)?
 )?
)
:orgName -->
  :orgName.TempOrganization = {kind = "unknown", rule = "OrgXBase"}

Rule:	TheBaseofOrg
Priority: 230

(
 {Token.category == DT}
)
(
 ({Lookup.majorType == org_base}|
  {Lookup.majorType == govern_key}
 )
 
 {Token.string == "of"}
 ( 
  {Token.category == DT}
 )?
 (UPPER)
 (UPPER)?
)
:orgName -->
  :orgName.TempOrganization = {kind = "unknown", rule = "BaseofOrg"}




Rule:	BaseofOrg
Priority: 130

(
 ({Lookup.majorType == org_base}|
  {Lookup.majorType == govern_key}
 )
 
 {Token.string == "of"}
 ( 
  {Token.category == DT}
 )?
 (UPPER)
 (UPPER)?
)
:orgName -->
  :orgName.TempOrganization = {kind = "unknown", rule = "BaseofOrg"}



Rule:	OrgPreX
Priority: 130

// Royal Tuscan

(
 {Lookup.majorType == org_pre}
 (
  {Token.orth == upperInitial})+
 ({Lookup.majorType == org_ending})?
)
:orgName -->
  :orgName.TempOrganization = {kind = "unknown", rule = "OrgPreX"}




Rule: OrgChurch
Priority: 150
// St. Andrew's Church

(
  (SAINT)
  {Token.orth == upperInitial}
  {Token.string == "'"}({Token.string == "s"})?
  (CHURCH)
)
:orgName -->
  :orgName.TempOrganization = {kind = "org", rule = "OrgChurch"}


Rule:OrgPersonAmbig
Priority: 130
// Alexandra Pottery should be org not person
// overrides PersonFull

(
 (FIRSTNAME)
 ({Lookup.majorType == org_key}|
  {Lookup.majorType == org_base})
 ({Lookup.majorType == org_ending})?
)
:org 
-->
 :org.TempOrganization= {kind = "unknown", rule = "OrgPersonAmbig"}

 

/////////////////////////////////////////////////////////////////
// Location rules


Rule: 	Location1
Priority: 75
// Lookup = city, country, province, region, water

// Western Europe
// South China sea

(
 {Token.category == DT}
)?
(
 ({Lookup.majorType == loc_key, Lookup.minorType == pre}
 )?
 {Lookup.majorType == location}
 (
  {Lookup.majorType == loc_key, Lookup.minorType == post})?
)
:locName -->
{
 gate.FeatureMap features = Factory.newFeatureMap();
// create an annotation set consisting of all the annotations for org 
gate.AnnotationSet locSet = (gate.AnnotationSet)bindings.get("locName");
// create an annotation set consisting of the annotation matching Lookup
gate.AnnotationSet loc = (gate.AnnotationSet)locSet.get("Lookup");
// if the annotation type Lookup doesn't exist, do nothing
if (loc != null && loc.size()>0)
{
// if it does exist, take the first element in the set
  gate.Annotation locAnn = (gate.Annotation)loc.iterator().next();
//propagate minorType feature (and value) from loc
  features.put("locType", locAnn.getFeatures().get("minorType"));
}
// create some new features
  features.put("rule", "Location1");
// create a TempLoc annotation and add the features we've created
annotations.add(locSet.firstNode(), locSet.lastNode(), "TempLocation",
features);
}

Rule:	GazLocation
Priority: 70
(
 {Token.category == DT}
)?  
(
 {Lookup.majorType == location}
)
:locName
 --> 	
{
 gate.FeatureMap features = Factory.newFeatureMap();
// create an annotation set consisting of all the annotations for org 
gate.AnnotationSet locSet = (gate.AnnotationSet)bindings.get("locName");
// create an annotation set consisting of the annotation matching Lookup
gate.AnnotationSet loc = (gate.AnnotationSet)locSet.get("Lookup");
// if the annotation type Lookup doesn't exist, do nothing
if (loc != null && loc.size()>0)
{
// if it does exist, take the first element in the set
  gate.Annotation locAnn = (gate.Annotation)loc.iterator().next();
//propagate minorType feature (and value) from loc
  features.put("locType", locAnn.getFeatures().get("minorType"));
}
// create some new features
  features.put("rule", "GazLocation");
// create a TempLoc annotation and add the features we've created
annotations.add(locSet.firstNode(), locSet.lastNode(), "TempLocation",
features);
}




Rule: LocationPost
Priority: 50
(
 {Token.category == DT}
)?
(
 {Token.category == NNP}
 {Lookup.majorType == loc_key, Lookup.minorType == post}
)
:locName
-->
 :locName.TempLocation = {kind = "locName", rule = LocationPost}

Rule:LocKey
(
 {Token.category == DT}
)?
(
 ({Lookup.majorType == loc_key, Lookup.minorType == pre}
 )
 (UPPER)
 (
  {Lookup.majorType == loc_key, Lookup.minorType == post})?
)
:locName -->
:locName.TempLocation = {kind = "locName", rule = LocKey}
/////////////////////////////////////////////////////////////////

// Context-based Rules


Rule:InLoc1
(
 {Token.string == "in"}
)
(
 {Lookup.majorType == location}
)
:locName
-->
 :locName.TempLocation = {kind = "locName", rule = InLoc1}

Rule:LocGeneralKey
Priority: 30
(
 {Lookup.majorType == loc_general_key}
 {Token.string == "of"}
)
(
 (UPPER)
)
:loc
-->
 :loc.TempLocation = {kind = "locName", rule = LocGeneralKey}


Rule:OrgContext1
Priority: 1
// company X

(
 {Token.string == "company"}
)
(
 (UPPER)
 (UPPER)?
 (UPPER)? 
)
:org
-->
 :org.TempOrganization= {kind = "orgName", rule = "OrgContext1"}

Rule: OrgContext2
Priority: 5

// Telstar laboratory
// Medici offices

(
 (UPPER)
 (UPPER)?
 (UPPER)? 
)
: org
(
 ({Token.string == "offices"} |
 {Token.string == "Offices"} |
 {Token.string == "laboratory"} | 
 {Token.string == "Laboratory"} |
 {Token.string == "laboratories"} |
 {Token.string == "Laboratories"})
)
-->
 :org.TempOrganization= {kind = "orgName", rule = "OrgContext2"}



Rule:JoinOrg
Priority: 50
// Smith joined Energis

(
 ({Token.string == "joined"}|
  {Token.string == "joining"}|
  {Token.string == "joins"}|
  {Token.string == "join"}
 )
)
(
 (UPPER)
 (UPPER)?
 (UPPER)?
)
:org
-->
 :org.TempOrganization= {kind = "orgName", rule = "joinOrg"}


/*
*  name_post.jape
*
* Copyright (c) 1998-2001, The University of Sheffield.
*
*  This file is part of GATE (see http://gate.ac.uk/), and is free
*  software, licenced under the GNU Library General Public License,
*  Version 2, June 1991 (in the distribution as file licence.html,
*  and also available at http://gate.ac.uk/gate/licence.html).
*
*  Diana Maynard, 10 Sep 2001
* 
*  $Id: grammars.tex,v 1.2 2001/11/29 18:07:27 diana Exp $
*/

Phase:	NamePost
Input: Token Lookup FirstPerson
Options: control = appelt

// this runs after name.jape to fix some problems that may have been caused

Rule: 	FirstPersonStop
Priority: 20
/* if the surname contains stop words e.g. "Mary And" we don't want it to be a Person
However, it might be that the firstname is actually an Organization (and has been tagged with TempOrganization already), e.g. "U.N." If this is the case, then leave it as it is. Otherwise, just tag the first name as a person
*/

(
 (FIRSTNAME)+
):person
(
 (
  ({Lookup.majorType == stop}|
   {Token.category == DT})
 )
)
-->
{
 gate.FeatureMap features = Factory.newFeatureMap();
gate.AnnotationSet personSet = (gate.AnnotationSet)bindings.get("person");
gate.AnnotationSet firstPerson = (gate.AnnotationSet)personSet.get("FirstPerson");

// get the TempOrganization annotation (previously assigned)
gate.AnnotationSet orgSet =
annotations.get("TempOrganization",personSet.firstNode().getOffset(),
personSet.lastNode().getOffset());
// and if it's empty
if (orgSet.size()==0)
{
// then if the firstPerson annotation exists
if (firstPerson != null && firstPerson.size()>0)
{
  gate.Annotation personAnn = (gate.Annotation)firstPerson.iterator().next();
  features.put("gender", personAnn.getFeatures().get("gender"));
}
  features.put("kind", "personName");
  features.put("rule", "FirstPersonStop");
annotations.add(personSet.firstNode(), personSet.lastNode(), "TempPerson",
features);
}}



/*
*  date_pre.jape
*
* Copyright (c) 1998-2001, The University of Sheffield.
*
*  This file is part of GATE (see http://gate.ac.uk/), and is free
*  software, licenced under the GNU Library General Public License,
*  Version 2, June 1991 (in the distribution as file licence.html,
*  and also available at http://gate.ac.uk/gate/licence.html).
*
*  Diana Maynard, 10 Sep 2001
* 
*  $Id: grammars.tex,v 1.2 2001/11/29 18:07:27 diana Exp $
*/


Phase:	DatePre
Options: control = appelt

///////
// Note: this muse come before the date phase,  because it includes extra context
// to prevent dates being recognised erroneously in the middle of longer things
//////


Macro: ORDINAL
(
   ({Token.kind == number}
    ({Token.string == "th"}|
     {Token.string == "rd"}|
     {Token.string == "nd"}|
     {Token.string == "st"})
    |
   {Lookup.minorType == ordinal})
   (SPACE
    {Token.string == "of"})?
)

/////////////////////////////////////////////////


Rule: GazDate
(SPACE | {Token.kind == punctuation})
(
 ({Lookup.minorType == day}) |
 ({Lookup.minorType == month}) |
 ({Lookup.minorType == festival})
)
:date
(SPACE | {Token.kind == punctuation})
-->
 :date.TempDate = {rule = "GazDate"}


Rule: PersonDateAmbig
Priority: 100
(
 (ORDINAL)
)
:date

(SPACE)
(
 {TempPerson.kind == personName, TempPerson.rule == PersonFull}
):person
-->
:date.Date = {kind = date, rule = "PersonDateAmbig"},
{
//removes  TempPerson annotation, gets the rule feature and adds a new Person annotation
gate.AnnotationSet person = (gate.AnnotationSet)bindings.get("person");
gate.Annotation personAnn = (gate.Annotation)person.iterator().next();
gate.FeatureMap features = Factory.newFeatureMap();
features.put("rule1", personAnn.getFeatures().get("rule"));
features.put("rule2", "PersonDateAmbig");
annotations.add(person.firstNode(), person.lastNode(), "Person",
features);
annotations.removeAll(person);
}


/*
*  date.jape
*
* Copyright (c) 1998-2001, The University of Sheffield.
*
*  This file is part of GATE (see http://gate.ac.uk/), and is free
*  software, licenced under the GNU Library General Public License,
*  Version 2, June 1991 (in the distribution as file licence.html,
*  and also available at http://gate.ac.uk/gate/licence.html).
*
*  Diana Maynard, 10 Sep 2001
* 
*  $Id: grammars.tex,v 1.2 2001/11/29 18:07:27 diana Exp $
*/


Phase:	Date
Input: Token Lookup
Options: control = appelt


/////////////////////////////////////////////////

Macro: DAY_NAME 
({Lookup.minorType == day })

Macro: ONE_DIGIT
({Token.kind == number, Token.length == "1"})

Macro: TWO_DIGIT
({Token.kind == number, Token.length == "2"})

Macro: FOUR_DIGIT
({Token.kind == number, Token.length == "4"})

Macro: DAY_MONTH_NUM
(ONE_DIGIT | TWO_DIGIT)

Macro: DATE_PRE
// possible modifiers of dates, eg. "early October"
({Token.string == "early"}|
 {Token.string == "late"}|
 {Token.string == "mid"}|
 {Token.string == "mid-"}|
 {Token.string == "end"}
)

Macro: DAY 
(((DATE_PRE)?
  DAY_NAME) |
 DAY_MONTH_NUM)

Macro: MONTH_NAME
( (DATE_PRE)?
  {Lookup.minorType == month})

Macro: MONTH 
(MONTH_NAME | DAY_MONTH_NUM)

Macro: SLASH
  ({Token.string == "/"})
  
Macro: DASH
  {Token.string == "-"}

Macro: OF
  {Token.string == "of"}

Macro: AD_BC
	(  {Token.string == "ad"} | {Token.string == "AD"}
	|
	  ({Token.string == "a"} {Token.string == "."}
	   {Token.string == "d"} {Token.string == "."})
	|
	  ({Token.string == "A"} {Token.string == "."}
	   {Token.string == "D"} {Token.string == "."})
	|

	  {Token.string == "bc"} | {Token.string == "BC"}
	|
	  ({Token.string == "b"} {Token.string == "."}
	   {Token.string == "c"} {Token.string == "."})
	 
	|
 	  ({Token.string == "B"} {Token.string == "."}
	   {Token.string == "C"} {Token.string == "."})
	)

Macro: YEAR
(        
 {Lookup.majorType == year}|
 TWO_DIGIT | FOUR_DIGIT | 
 {Token.string == "'"}
 (TWO_DIGIT)
)


Macro:	XDAY
(
 ({Token.orth == upperInitial} |
  {Token.orth == allCaps})
 {Token.string == "Day"}
)


Macro: ORDINAL
(
   ({Token.kind == number}
    ({Token.string == "th"}|
     {Token.string == "rd"}|
     {Token.string == "nd"}|
     {Token.string == "st"})
    |
   {Lookup.minorType == ordinal})
   (
    {Token.string == "of"})?
)

Macro: NUM_OR_ORDINAL
  (ORDINAL | DAY_MONTH_NUM)


Macro: COMMA
({Token.string == ","})


Macro:  TIME_ZONE
(({Lookup.minorType == zone})|
 ({Token.string == "("}
  {Lookup.minorType == zone}
  {Token.string == ")"})
)

Macro: TIME_DIFF
(
 ({Token.string == "+"}|{Token.string == "-"})
 (FOUR_DIGIT)
)

Macro: TIME_AMPM
(
 {Lookup.minorType == ampm}
)




///////////////////////////////////////////////////////////////
// Time Rules 

Rule: TimeDigital1
// 20:14:25
(
 (ONE_DIGIT|TWO_DIGIT){Token.string == ":"} TWO_DIGIT 
({Token.string == ":"} TWO_DIGIT)?
(TIME_AMPM)?	
(TIME_DIFF)?
(TIME_ZONE)? 
)
:time
-->
:time.TempTime = {kind = "positive", rule = "TimeDigital1"}


Rule:	TimeDigital2

// 8:14 am
// 4.34 pm
// 6am

(
 (ONE_DIGIT|TWO_DIGIT) 
 (({Token.string == ":"}|{Token.string == "."} |{Token.string == "-"} )
  TWO_DIGIT)?
 (TIME_AMPM)
 (TIME_ZONE)?
)
:time
-->
:time.TempTime = {kind = "positive", rule = "TimeDigital"}


Rule: TimeOClock
// ten o'clock

(
 {Lookup.minorType == hour}
 {Token.string == "o"}
 {Token.string == "'"}
 {Token.string == "clock"}
)
:time 
-->
 :time.TempTime = {kind = "positive", rule = "TimeOClock"}

 
Rule: TimeAnalogue
// half past ten
// ten to twelve
// twenty six minutes to twelve

(
 (((({Lookup.majorType == number})?
    {Lookup.majorType == number}
   )
   {Token.string == "minutes"}
  ) |
  ({Token.string == "half"} | 
   {Token.string == "quarter"}) 
 )
 ({Token.string == "past"}|
  {Token.string == "to"})
 {Lookup.minorType == hour}
)
:time 
-->
 :time.TempTime = {kind = "positive", rule = "TimeAnalogue"}


Rule: TimeWordsContext
Priority: 50
// seven thirty tomorrow

(
 {Lookup.majorType == number}
 (
  {Lookup.majorType == number}
 )?
):time1
(
 {Lookup.minorType == time_key}
) 
-->
:time1.TempTime = {kind = "positive", rule = "TimeWordsContext"}


Rule: TimeWords

(
 {Lookup.majorType == number}
 (
  {Lookup.majorType == number}
 )?
)
:time
-->
  :time.TempTime = {kind = "timeWords", rule = "TimeWords"}

  


Rule: TimeDigitalContext1

(
(FOUR_DIGIT)
):time
{Lookup.minorType == time_key}
 -->
 :time.TempTime = {kind = "positive", rule = "TimeDigitalContext"}

Rule: NotTimeDigitalContext2
Priority: 100
// prevent things like "at 0.61 km/h"

(
 {Token.string == "at"}
)
({Token.string == "0"}
  ({Token.string == ":"}|{Token.string == "-"}|{Token.string == "."}) TWO_DIGIT
 (TIME_AMPM)?
 (TIME_ZONE)?
)
:time
 -->
 :time.Temp = {rule = "NotTimeDigitalContext2"}


Rule: TimeDigitalContext2

(
 {Token.string == "at"}
)
(
 FOUR_DIGIT | 
 ((ONE_DIGIT|TWO_DIGIT)
  ({Token.string == ":"}|{Token.string == "-"}|{Token.string == "."}) TWO_DIGIT
 )
 (TIME_AMPM)?
 (TIME_ZONE)?
)
:time
 -->
 :time.TempTime = {kind = "positive", rule = "TimeDigitalContext2"}

Rule: TimeDigitalTemp1

(
 FOUR_DIGIT | 
 ((ONE_DIGIT|TWO_DIGIT)
  ({Token.string == ":"}|{Token.string == "-"}|{Token.string == "."}) TWO_DIGIT
 )
)
:time
 -->
 :time.TempTime = {kind = "temp", rule = "TimeDigitalTemp"}

Rule: TimeDigitalContext1
(
 {Token.string == "in"}
)?
((ONE_DIGIT|TWO_DIGIT)
 ({Token.string == ":"}|{Token.string == "."})
 TWO_DIGIT
 ({Token.string == "seconds"}|
  {Token.string == "minutes"}|
  {Token.string == "hours"}
 )
):time
-->
:time.TempTime = {kind = "positive", rule = "TimeDigitalContext1"}



Rule: TimeDigitalContextConj

(
 {Token.string == "at"}
)
(
 FOUR_DIGIT | 
 ((ONE_DIGIT|TWO_DIGIT)
  ({Token.string == ":"}|{Token.string == "-"}|{Token.string == "."}) TWO_DIGIT
 )
)
:time1
(
 {Token.string == "and"}
)
(
 FOUR_DIGIT | 
 ((ONE_DIGIT|TWO_DIGIT)
  ({Token.string == ":"}|{Token.string == "-"}|{Token.string == "."}) TWO_DIGIT
 )
):time2
 -->
 :time1.TempTime = {kind = "positive", rule = "TimeDigitalContextConj"},

:time2.TempTime = {kind = "positive", rule = "TimeDigitalContextConj"}




//////////////////////////////////////////////////////////////////

// Date Rules

Rule: IgnoreDatePerson
Priority: 500
(
 {Date}
 {Person}
)
:date
-->
{}



Rule:	DateSlash           // UK only
// Wed, 10/July/00
// 10/July
// July/99

(
 ((DAY_NAME (COMMA)? )?
 NUM_OR_ORDINAL SLASH MONTH_NAME (SLASH YEAR)? )|
 (MONTH_NAME SLASH YEAR) 
)
:date
-->
 :date.TempDate = {rule = "DateSlash"}



Rule:	DateDash
// Wed 10-July-00
// 10-July 00
// 10-July

(  
 ((DAY_NAME (COMMA)?)?
  (NUM_OR_ORDINAL DASH MONTH_NAME (DASH)? YEAR)) |

 ((DAY_NAME (COMMA)?)?
  NUM_OR_ORDINAL DASH MONTH_NAME)
)
:date
-->
 :date.TempDate = {rule = "DateDash"}



Rule: 	DateName
Priority: 20
// Wed 10 July
// Wed 10 July, 2000
// Sun, 21 May 2000
// 10th of July, 2000
// 10 July
// 10th of July
// July, 2000

(
 (DAY_NAME NUM_OR_ORDINAL MONTH_NAME)|

 (DAY_NAME (COMMA)? 
  NUM_OR_ORDINAL MONTH_NAME ((COMMA)? YEAR)?) 
|

 ((DAY_NAME (COMMA)? )?
 NUM_OR_ORDINAL MONTH_NAME 
 ((COMMA)? YEAR)?)
|

 (NUM_OR_ORDINAL MONTH_NAME) 
| 
(MONTH_NAME (COMMA)? YEAR)
)
:date
-->
 :date.TempDate = {rule = "DateName"}


Rule: DateNameSpan1
// 5-20 Jan

(
 NUM_OR_ORDINAL
 {Token.string == "-"}
 (NUM_OR_ORDINAL MONTH_NAME ((COMMA)?  YEAR)?)
)
:date
-->
 :date.TempDate = {rule = "DateNameSpan1"}

Rule: DateNameSpan2
// Jan 5-20

(MONTH_NAME

 NUM_OR_ORDINAL 
 {Token.string == "-"}
 (NUM_OR_ORDINAL ((COMMA)?  YEAR)?)
)
:date
-->
 :date.TempDate = {rule = "DateNameSpan2"}

Rule: DateNameRev
// Wed. July 1st, 2000
// Wed, July 1, 2000
// Wed, July 1st, 2000
(         
 ((DAY_NAME (COMMA)? )?
  MONTH_NAME  
  ({Token.string == "the"})?
 NUM_OR_ORDINAL 
 ((COMMA)? YEAR)?) |
         
 (MONTH_NAME (COMMA)? YEAR)
)
:date
-->
 :date.TempDate = {rule = "DateNameRev"}


Rule:	DateNumDash
// 01-07-00
// Note: not 07-00
  
(
 (DAY_MONTH_NUM DASH DAY_MONTH_NUM DASH YEAR)
)
:date
-->
 :date.TempDate = {rule = "DateNumDash"}

Rule:	DateNumDashRev
// 00-07-01
// 2000-07

(
 (YEAR DASH DAY_MONTH_NUM DASH DAY_MONTH_NUM)|
 (FOUR_DIGIT DASH DAY_MONTH_NUM)
)
:date
-->
 :date.TempDate = {rule = "DateNumDashRev"}


Rule:	DateNumSlash
// 01/07/00
// Note: not 07/00

( 
DAY_MONTH_NUM SLASH DAY_MONTH_NUM SLASH YEAR
)
:date
-->
 :date.TempDate = {rule = "DateNumSlash"}


Rule: ModifierMonth
//early October

( DATE_PRE 
  {Lookup.minorType == month}
)
:date -->
 :date.TempDate = {rule = "ModifierMonth"}


Rule: YearAdBc

// 1900 AD
(
 (YEAR 
  AD_BC)
)
:year -->
 :year.YearTemp = {kind = "positive", rule = "YearAdBc"}

Rule: YearSpan1
// the early 90s
// the late 80s
(
 {Token.string == "the"}
 (DATE_PRE)?
 (YEAR)
 ({Token.string == "'"})?
 ({Token.string == "s"})
)
:date -->
 :date.TempDate = {rule = "YearSpan1"}

Rule: YearSpan2
// 1980/81

(
 (FOUR_DIGIT)
 ({Token.string == "/"}|
  {Token.string == "-"})
 (FOUR_DIGIT|TWO_DIGIT | ONE_DIGIT)
)
:date -->
 :date.TempDate = {rule = "YearSpan2"}

Rule: YearSpan3
Priority: 80

// from 1980 to 1981
// between 1980 and 1981
(
 (({Token.string == "from"}| {Token.string == "From"})
  (FOUR_DIGIT)
  {Token.string == "to"}
  (FOUR_DIGIT)
 ) |
  (({Token.string == "between"}|{Token.string == "Between"}) 
  (FOUR_DIGIT)
  {Token.string == "and"}
  (FOUR_DIGIT)
 )
)
:date -->
 :date.TempDate = {rule = "YearSpan3"}


Rule: YearContext1
Priority: 40
({Token.string == "in"}|
 {Token.string == "by"}
)
(YEAR)
:date -->
 :date.TempDate = {rule = "YearContext1"}


// Currently, temp1, temp2 and temp3 look good; temp4 is not to be counted
// but this may change according to the text
// only positives will be used in final grammar, not negatives

Rule: YearTemp1
Priority: 30
// (1987)

({Token.position == startpunct})
(FOUR_DIGIT)
:date
({Token.position == endpunct})
 -->
 :date.TempYear = {kind = "positive", rule = "TempYear1"}


Rule: TempYear2
Priority: 20
// 1987

(
 {Lookup.majorType == year}
)
:date -->
 :date.TempYear = {kind = "positive", rule = "TempYear2"}


Rule: TempYear3
Priority: 10
// 1922

(FOUR_DIGIT)
:date -->
 :date.TempYear = {kind = "positive", rule = "TempYear3"}


Rule: YearWords
// nineteen twenty three
// nineteen ten

(
 {Token.string == "nineteen"}
 ({Lookup.majorType == number}
 )?
 {Lookup.majorType == number}
)
 :date -->
 :date.TempYear = {kind = "positive", rule = "YearWords"}



Rule: TimeZone
// +0400
(
 (TIME_DIFF)
 (TIME_ZONE)?
)
:date
-->
 :date.TempZone = {rule = "TimeZone"}


/*
*  reldate.jape
*
* Copyright (c) 1998-2001, The University of Sheffield.
*
*  This file is part of GATE (see http://gate.ac.uk/), and is free
*  software, licenced under the GNU Library General Public License,
*  Version 2, June 1991 (in the distribution as file licence.html,
*  and also available at http://gate.ac.uk/gate/licence.html).
*
*  Diana Maynard, 10 Sep 2001
* 
*  $Id: grammars.tex,v 1.2 2001/11/29 18:07:27 diana Exp $
*/

// this file must follow date.jape
// handles relative date and time sequences

Phase:	Name
Input: Token Lookup TempDate
Options: control = appelt

Rule: GazDateWords
Priority: 10
// yesterday evening

(
 {Lookup.majorType == date_key}
 (
  {Lookup.majorType == time_unit}
 )?
)
:date -->
  :date.TempDate = {rule = "GazDateWords"}


Rule: TimeAgo
Priority:30
// 2 hours ago

(
 {Token.kind == number}
 ({Lookup.majorType == time_unit})
 {Token.string == "ago"}
)
:date -->
 :date.TempDate = {rule = "TimeAgo"}


Rule: DateAgo
Priority:30
// 2 weeks ago

(
 {Token.kind == number}
 ({Lookup.majorType == date_unit})
 {Token.string == "ago"}
)
:date -->
 :date.TempDate = {rule = "TimeAgo"}


Rule: ModifierDate
Priority: 30
// last year
// next 10 years

(
 {Lookup.majorType == time_modifier} 
 (
  ({Lookup.majorType == number}|
   {Token.kind == number}
  )
 )?
 {Lookup.majorType == date_unit}
)
:date -->
 :date.TempDate = {rule = "ModifierDate"}

Rule: EarlyDate
// early in 2002
// in early 2002

(
 ({Token.string == early}|
  {Token.string == late}
 )
 ({Token.string == "in"}
 )?
 ({TempDate}|
  (
   {Lookup.majorType == time_modifier} 
   {Lookup.majorType == date_unit}
  )
 )
)
:date
-->
 {
//removes TempDate annotation, gets the rule feature and adds a new TempDate annotation
gate.AnnotationSet date = (gate.AnnotationSet)bindings.get("date");
gate.Annotation dateAnn = (gate.Annotation)date.iterator().next();
gate.FeatureMap features = Factory.newFeatureMap();
features.put("rule1", dateAnn.getFeatures().get("rule"));
features.put("rule2", "EarlyDate");
annotations.add(date.firstNode(), date.lastNode(), "TempDate",
features);
annotations.removeAll(date);
}
 
Rule:FiscalDate
// first half of next year
// first nine months of the financial year 

(
 {Lookup.minorType == ordinal}
 (
  ({Lookup.majorType == number}|
   {Token.kind == number}
  )
 )?
 {Lookup.majorType == date_unit}
 
 ({Token.string == "of"}
 )?
 ((
   {Token.category == DT}|
   {Token.category == "PRP$"}
  )
 )?
 ({Lookup.majorType == time_modifier} 
 )?
 {Lookup.majorType == date_unit}
)
:date -->
  :date.TempDate = {rule = FiscalDate}



/*
*  number.jape
*
* Copyright (c) 1998-2001, The University of Sheffield.
*
*  This file is part of GATE (see http://gate.ac.uk/), and is free
*  software, licenced under the GNU Library General Public License,
*  Version 2, June 1991 (in the distribution as file licence.html,
*  and also available at http://gate.ac.uk/gate/licence.html).
*
*  Diana Maynard, 02 Aug 2001
* 
*  $Id: grammars.tex,v 1.2 2001/11/29 18:07:27 diana Exp $
*/


Phase:	Number
Input: Token Lookup
Options: control = appelt

///////////////////////////////////////////////////////////////
//Money Rules


Macro: MILLION_BILLION
({Token.string == "m"}|
{Token.string == "million"}|
{Token.string == "b"}|
{Token.string == "billion"}|
{Token.string == "bn"}|
{Token.string == "k"}|
{Token.string == "K"}
)

Macro: NUMBER_WORDS
// two hundred and thirty five
// twenty five

(
 (({Lookup.majorType == number} 
   ({Token.string == "-"})?
  )*
   {Lookup.majorType == number}
   {Token.string == "and"}
 )*
 ({Lookup.majorType == number} 
  ({Token.string == "-"})?
 )*
   {Lookup.majorType == number}
)


Macro: AMOUNT_NUMBER
// enables commas, decimal points and million/billion
// to be included in a number
 
(({Token.kind == number}
  (({Token.string == ","}|
    {Token.string == "."}
   )
   {Token.kind == number}
  )*
  |
  (NUMBER_WORDS)
 )
 (MILLION_BILLION)?
)


Rule:	MoneyCurrencyUnit
// 30 pounds
  (        
      (AMOUNT_NUMBER)
      ({Lookup.majorType == currency_unit})
  )
:number -->
  :number.Money = {kind = "number", rule = "MoneyCurrencyUnit"}


Rule:	MoneySymbolUnit

// $30 
// $30 US
// not $1$21
// $20US


(   
 ({Token.symbolkind == currency}|
  {Lookup.majorType == currency_unit})
 (AMOUNT_NUMBER)
 (
  {Lookup.majorType == currency_unit}
 )?
) 
:number 

 -->
  :number.Money = {kind = "number", rule = "MoneySymbolUnit"}

//////////////////////////////////////////////////////////////

// Percentage Rules

Macro: PERCENT
({Token.string == "%"} | 
 {Token.string == "percent"}|
 ({Token.string == "per"}
 {Token.string == "cent"})
)


Rule: PercentBasic
// +20%
// minus 10 percent
// two point four percent

(
 ({Token.string == "+"}|
  {Token.string == "-"}|
  {Token.string == "minus"}  
 )?
  ((AMOUNT_NUMBER|NUMBER_WORDS)
  {Token.string == "point"}
 )? 
 (AMOUNT_NUMBER|NUMBER_WORDS)
 (PERCENT)
)
:number -->
  :number.Percent = {rule = "PercentBasic"}

Rule: PercentSpan
// 20-30%
// two to four percent

(
 (AMOUNT_NUMBER|NUMBER_WORDS)
 ({Token.string == "-"} |
  {Token.string == "to"})
 (AMOUNT_NUMBER|NUMBER_WORDS)
 (PERCENT)
)
:number -->
  :number.Percent = {rule = "PercentSpan"}


/*
*  address.jape
*
* Copyright (c) 1998-2001, The University of Sheffield.
*
*  This file is part of GATE (see http://gate.ac.uk/), and is free
*  software, licenced under the GNU Library General Public License,
*  Version 2, June 1991 (in the distribution as file licence.html,
*  and also available at http://gate.ac.uk/gate/licence.html).
*
*  Diana Maynard, 02 Aug 2001
* 
*  $Id: grammars.tex,v 1.2 2001/11/29 18:07:27 diana Exp $
*/


Phase:	Address
Input:  Token Lookup
Options: control = appelt

/////////////////////////////////////////////////////////


///////////////////////////////////////////////////////////////////
//Phone Rules


Macro: PHONE_COUNTRYCODE
// +44

({Token.string == "+"}
 {Token.kind == number,Token.length == "2"}
)	

Macro: PHONE_AREACODE
// 081 (for the old style codes)
// 01234
// (0123)

(
 ({Token.kind == number,Token.length == "3"} |
  {Token.kind == number,Token.length == "4"} |
  {Token.kind == number,Token.length == "5"}) 
|
 ({Token.string == "("} 
  ({Token.kind == number,Token.length == "3"} |
   {Token.kind == number,Token.length == "4"} |
   {Token.kind == number,Token.length == "5"}) 
  ({Token.string == ")"})
 )
)

Macro: PHONE_REG
// 222 1481
// 781932

(
  ({Token.kind == number,Token.length == "3"}

  ({Token.kind == number,Token.length == "3"} |
   {Token.kind == number,Token.length == "4"}) 
  )
                                              |

  ({Token.kind == number,Token.length == "5"}|
   {Token.kind == number,Token.length == "6"})
)
 
Macro: PHONE_EXT
// x1234
// ext. 1234/5

(
 (({Token.string == "x"})		|
  ({Token.string == "x"}{Token.string == "."}) |
  ({Token.string == "ext"})	|
  ({Token.string == "ext"}{Token.string == "."})
 )  
 ({Token.kind == number, Token.length == "4"}|
  {Token.kind == number, Token.length == "5"}|
  {Token.kind == number, Token.length == "6"})
 ({Token.string == "/"}{Token.kind == number})?
)
	
Macro: PHONE_PREFIX
//Tel:

(
 {Lookup.majorType == phone_prefix}
 ({Lookup.majorType == phone_prefix})?
 ({Token.string == ":"})?
)
  
//////////////////////////////////////////////////
Rule:PhoneReg
Priority: 20
// regular types of number
// 01234 123 456
// (01234) 123456


(

 (PHONE_AREACODE)
 (PHONE_REG)
)
:phoneNumber -->
 :phoneNumber.Phone = {kind = "phoneNumber", rule = "PhoneReg"}

Rule: PhoneRegContext
Priority: 100
// tel: 0114 222 1929
(
 (PHONE_PREFIX)
)
(
 (PHONE_AREACODE)
 (PHONE_REG)
)
:phoneNumber -->
 :phoneNumber.Phone = {kind = "phoneNumber", rule = "PhoneRegContext"}


Rule:PhoneFull
Priority: 50
// +44 161 222 1234
// +44 (0)161 222 1234
(
 (PHONE_PREFIX)?
)
(
 ((PHONE_COUNTRYCODE) |
  ({Token.string == "("} 
   (PHONE_COUNTRYCODE)
   {Token.string == ")"})
 )

 ({Token.string == "("} 
   {Token.string == "0"} 
   {Token.string == ")"})?

 ({Token.kind == number,Token.length == "3"}|
  {Token.kind == number,Token.length == "4"})
      
 (PHONE_REG)
)
:phoneNumber -->
 :phoneNumber.Phone = {kind = "phoneNumber", rule = "PhoneFull"}



Rule:PhoneExt
Priority: 20

// extension number only
// ext. 1234
(
 (PHONE_EXT)
)
:phoneNumber --> 
 :phoneNumber.Phone = {kind = "phoneNumber", rule = "PhoneExt"}

Rule:PhoneRegExt
Priority: 40

// 01234 12345 ext. 1234
// 01234 12345 x1234/5

(

 (PHONE_AREACODE)?
 (PHONE_REG)
 (PHONE_EXT)
)
:phoneNumber -->
 :phoneNumber.Phone = {kind = "phoneNumber", rule = "PhoneRegExt"}


Rule:PhoneRegExtContext
Priority: 40
// tel: 12345 ext. 1234
(
 (PHONE_PREFIX)
)
(
 (PHONE_AREACODE)?
 (PHONE_REG)
 (PHONE_EXT)
)
:phoneNumber -->
 :phoneNumber.Phone = {kind = "phoneNumber", rule = "PhoneRegExtContext"}

Rule:PhoneNumberOnly
Priority: 20
// Phone 123456
// Fax: 123 456
// Tel. No. 123456
// only recognise numbers like this when preceded by clear context.

(PHONE_PREFIX)
(
 (PHONE_REG)
)
:phoneNumber -->
 :phoneNumber.Phone = {kind = "phoneNumber", rule = "PhoneNumberOnly"}


Rule: PhoneOtherContext
// sometimes they're unusual
// tel: 0124 1963

(
 (PHONE_PREFIX)
)
(
 {Token.kind == number}
 ({Token.kind == number})*
)
:phoneNumber 
-->
 :phoneNumber.Phone = {kind = "phoneNumber", rule = "PhoneOtherContext"}

///////////////////////////////////////////////////////////////
// Postcode Rules


// S1 4DP, S10 4DP, SO1 10DP, etc.
// UK postocdes only

Rule: Postcode1
Priority: 50
(
 ({Token.orth == allCaps,Token.length == "2"}|
  {Token.orth == upperInitial,Token.length == "1"}			
 )	
 ({Token.kind == number,Token.length == "1"}|
  {Token.kind == number,Token.length == "2"}
 )
 ({Token.kind == number,Token.length == "1"}|
  {Token.kind == number,Token.length == "2"}
 )
 ({Token.orth == allCaps,Token.length == "2"}|
  {Token.orth == upperInitial,Token.length == "1"}		
 )	
):postcode -->
:postcode.Postcode = {kind = "postcode", rule = Postcode1}

//////////////////////////////////////////////////////////

////////////////////////////////////////////////////////

// IP Address Rules

Rule: IPaddress1
(	{Token.kind == number}
	{Token.string == "."}
	{Token.kind == number}
	{Token.string == "."}
	{Token.kind == number}	
	{Token.string == "."}
	{Token.kind == number}
):ipAddress -->
:ipAddress.Ip = {kind = "ipAddress", rule = "IPaddress1"}

/////////////////////////////////////////////////////////////
// Street Rules


Rule: StreetName1
(
 ({Token.kind == number}
  ({Token.string == ","})? 
  )?
 {Token.orth == upperInitial}
 {Lookup.minorType == "street"}
)
:streetAddress -->
 :streetAddress.Street = {kind = "streetAddress", rule = StreetName1}


Rule: POBoxAddress
(
 (({Token.string == "P"}
   ({Token.string == "."})? 
   {Token.string == "O"}
   ({Token.string == "."})?
  ) |
  ({Token.string == "PO"})
 )
 {Token.string == "Box"}
 {Token.kind == number}
)
:address -->
 :address.Street = {kind = "poBox", rule = POBoxAddress}


/*
*  url.jape
*
* Copyright (c) 1998-2001, The University of Sheffield.
*
*  This file is part of GATE (see http://gate.ac.uk/), and is free
*  software, licenced under the GNU Library General Public License,
*  Version 2, June 1991 (in the distribution as file licence.html,
*  and also available at http://gate.ac.uk/gate/licence.html).
*
*  Diana Maynard, 02 Aug 2001
* 
*  $Id: grammars.tex,v 1.2 2001/11/29 18:07:27 diana Exp $
*/

Phase:	Url
Input:  Token Lookup SpaceToken
Options: control = appelt


// Email rules

Rule:Emailaddress1
Priority: 50
(
 (
  {Token.kind == word}|
  {Token.kind == number}|
  {Token.string == "_"}
 )
 ({Token.string == "."}
  ({Token.kind == word}|
   {Token.kind == number}|
   {Token.string == "_"}
  )
 )?
 {Token.string == "@"}		
 (
  {Token.kind == word}|
  {Token.kind == symbol}|
  {Token.kind == punctuation}|
  {Token.kind == number}
 )
 ({Token.string == "."})?
 (
  {Token.kind == word}|
  {Token.kind == symbol}|
  {Token.kind == punctuation}|
  {Token.kind == number}
 )?
 ({Token.string == "."})?
(
  {Token.kind == word}|
  {Token.kind == symbol}|
  {Token.kind == punctuation}|
  {Token.kind == number}
 )?
({Token.string == "."})?
 (
  {Token.string == "."}		
  (
   {Token.kind == word}|
   {Token.kind == number}	
  )
  ({Token.string == "."})?
  (
   {Token.kind == word}|
   {Token.kind == number}	
  )?
  ({Token.string == "."})?
  (
   {Token.kind == word}|
   {Token.kind == number}	
  )?
 )
)
:emailAddress -->
  :emailAddress.Email= {kind = "emailAddress", rule = "Emailaddress1"}




// Url Rules

// http://www.amazon.com
// ftp://amazon.com
// www.amazon.com

Rule: Url1
Priority: 50

(	((({Token.string == "http"}	|
	   {Token.string == "ftp"})
	   {Token.string == ":"}
	   {Token.string == "/"}
           {Token.string == "/"})	|
	 ({Token.string == "www"}
          {Token.string == "."})	
	)			
	
	({Token.orth == lowercase}	|
        {Token.orth == upperInitial}	|
        {Token.kind == number}		|
        {Token.kind == punctuation}	|
        {Token.kind == symbol}		|
	{Token.string == "."})+ 
	
	({Token.orth == lowercase}	|
        {Token.orth == upperInitial}	|
        {Token.kind == number}		|
        {Token.kind == punctuation}	|
        {Token.kind == symbol}		|
	{Token.string == "/"}		|
        {Token.string == "."})*
):urlAddress -->
:urlAddress.Url = {kind = "urlAddress", rule = "Url1"}

Rule: UrlContext
Priority: 20

(
 {Token.string == "at"}
 {Token.string == ":"}
)
(
 ({Token.orth == lowercase}	|
        {Token.orth == upperInitial}	|
        {Token.kind == number}		|
        {Token.kind == punctuation}	|
        {Token.kind == symbol}		|
	{Token.string == "."})+ 
	
        {Token.string == "."}
	
  	({Token.orth == lowercase}	|
        {Token.orth == upperInitial}	|
        {Token.kind == number}		|
        {Token.kind == punctuation}	|
        {Token.kind == symbol}		|
	{Token.string == "/"}		|
        {Token.string == "."})*
)
:urlAddress 
-->
 :urlAddress.Url = {kind = "urlAddress", rule = "UrlContext"}


/*
*  identifier.jape
*
* Copyright (c) 1998-2001, The University of Sheffield.
*
*  This file is part of GATE (see http://gate.ac.uk/), and is free
*  software, licenced under the GNU Library General Public License,
*  Version 2, June 1991 (in the distribution as file licence.html,
*  and also available at http://gate.ac.uk/gate/licence.html).
*
*  Diana Maynard, 10 Sep 2001
* 
*  $Id: grammars.tex,v 1.2 2001/11/29 18:07:27 diana Exp $
*/


Phase:	Identifier
Options: control = appelt



Macro: IDENT

({Lookup.majorType == ident_key, Lookup.minorType == pre}
 ({Token.string == "."}|{Token.string == ":"})?
)


Rule: GazNotIdentifier
Priority: 100
(
 {Lookup.majorType == spur_ident}
)
:ident -->
 :ident.Temp = {rule = "GazNotIdentifier"}  


Rule: Identifier1
(	(({Token.kind == number}	
	  {Token.orth == allCaps}) |
	 ({Token.orth == allCaps}
	  {Token.kind == number})
        )
	({Token.orth == allCaps}|
         {Token.orth == upperInitial, Token.length == "1"}| 
         {Token.string == "."}|
	 {Token.kind == number})*
):ident 
-->
 :ident.TempIdentifier = {rule = "Identifier1"}



Rule: NotContextIdentifier
Priority: 100
// prevents things like "reference list"

(IDENT)
(
{Token.orth == lowercase}
):ident 
(SPACE|{Token.kind == punctuation})
-->
:ident.Temp = {rule = "NotContextIdentifier"}


Rule: ContextIdentifier
(IDENT)
(	(({Token.kind == number}	
	 {Token.orth == allCaps}) |
	({Token.orth == allCaps}
	 {Token.kind == number})|
        ({Token.orth == upperInitial, Token.length == "1"}))
	({Token.orth == allCaps}|
         {Token.orth == upperInitial, Token.length == "1"}| 
         {Token.string == "."}|
	 {Token.kind == number})*
)
:ident 
-->
 :ident.TempIdentifier = {rule = "ContextIdentifier"}


/*
*  jobtitle.jape
*
* Copyright (c) 1998-2001, The University of Sheffield.
*
*  This file is part of GATE (see http://gate.ac.uk/), and is free
*  software, licenced under the GNU Library General Public License,
*  Version 2, June 1991 (in the distribution as file licence.html,
*  and also available at http://gate.ac.uk/gate/licence.html).
*
*  Diana Maynard, 10 Sep 2001
* 
*  $Id: grammars.tex,v 1.2 2001/11/29 18:07:27 diana Exp $
*/


Phase:	Jobtitle
Input: Lookup
Options: control = appelt

Rule: Jobtitle1
(
 {Lookup.majorType == jobtitle} 
 (
  {Lookup.majorType == jobtitle} 
 )?
)
:jobtitle
-->
 :jobtitle.JobTitle = {rule = "JobTitle1"}



/*
*  final.jape
*
* Copyright (c) 1998-2001, The University of Sheffield.
*
*  This file is part of GATE (see http://gate.ac.uk/), and is free
*  software, licenced under the GNU Library General Public License,
*  Version 2, June 1991 (in the distribution as file licence.html,
*  and also available at http://gate.ac.uk/gate/licence.html).
*
*  Diana Maynard, 10 Sep 2001
* 
*  $Id: grammars.tex,v 1.2 2001/11/29 18:07:27 diana Exp $
*/

//note: organization should be included as part of the address ??

Phase:	Final
Input: Token Lookup Jobtitle TempPerson TempLocation TempOrganization TempDate TempTime TempYear TempZone Street Postcode Email Url Phone Ip TempIdentifier TempSpecs
Options: control = appelt


///////////////////////////////////////////////////////////////

Rule: PersonFinal
Priority: 30
({JobTitle}
)?
(
 {TempPerson.kind == personName}
)
:person
--> 
{
 gate.FeatureMap features = Factory.newFeatureMap();
gate.AnnotationSet personSet = (gate.AnnotationSet)bindings.get("person");
gate.Annotation person1Ann = (gate.Annotation)personSet.iterator().next();

gate.AnnotationSet firstPerson = (gate.AnnotationSet)personSet.get("TempPerson");
if (firstPerson != null && firstPerson.size()>0)
{
  gate.Annotation personAnn = (gate.Annotation)firstPerson.iterator().next();
  features.put("gender", personAnn.getFeatures().get("gender"));
}
  features.put("rule1", person1Ann.getFeatures().get("rule"));
  features.put("rule", "PersonFinal");
annotations.add(personSet.firstNode(), personSet.lastNode(), "Person",
features);
annotations.removeAll(personSet);
}


Rule:	OrgCountryFinal
Priority: 50
// G M B Scotland
// Scottish Electricity Board

(
 ({TempOrganization})?

 (({Lookup.majorType == country_adj}|
   {Lookup.majorType == location}) 
  {TempOrganization}
 )|
 ({TempOrganization}
  ({Token.position == startpunct})?
  ({Lookup.majorType == country_adj}|
   {Lookup.majorType == location})
  ({Token.position == endpunct})?
 )
 ({TempOrganization})?
 ({Lookup.majorType == org_ending})?
)
:org
-->
{
//removes TempOrg annotation, gets the rule feature and adds a new Org annotation
gate.AnnotationSet org = (gate.AnnotationSet)bindings.get("org");
gate.Annotation orgAnn = (gate.Annotation)org.iterator().next();
gate.FeatureMap features = Factory.newFeatureMap();
features.put("orgType", orgAnn.getFeatures().get("orgType"));
features.put("rule1", orgAnn.getFeatures().get("rule"));
features.put("rule2", "OrgCountryFinal");
annotations.add(org.firstNode(), org.lastNode(), "Organization",
features);
annotations.removeAll(org);
}
 


//note - move this rule to after final

Rule: OrgFinal
Priority: 10
(
 {TempOrganization}
)
:org
--> 
{
//removes TempOrg annotation, gets the rule feature and adds a new Org annotation
gate.AnnotationSet org = (gate.AnnotationSet)bindings.get("org");
gate.Annotation orgAnn = (gate.Annotation)org.iterator().next();
gate.FeatureMap features = Factory.newFeatureMap();
features.put("orgType", orgAnn.getFeatures().get("orgType"));
features.put("rule1", orgAnn.getFeatures().get("rule"));
features.put("rule2", "OrgFinal");
annotations.add(org.firstNode(), org.lastNode(), "Organization",
features);
annotations.removeAll(org);
}


Rule: PersonLocFinal
Priority: 100
// George Airport
// later we might change this to any facility, rather than just airports

(
 {TempPerson}
 ({Token.string == "airport"} |
  {Token.string == "Airport"})
)
:loc
-->
 {
//removes TempLoc annotation, gets the rule feature and adds a new Loc annotation
gate.AnnotationSet loc = (gate.AnnotationSet)bindings.get("loc");
gate.Annotation locAnn = (gate.Annotation)loc.iterator().next();
gate.FeatureMap features = Factory.newFeatureMap();
features.put("rule1", locAnn.getFeatures().get("rule"));
features.put("rule2", "PersonLocFinal");
annotations.add(loc.firstNode(), loc.lastNode(), "Location",
features);
annotations.removeAll(loc);
}


 
Rule: LocFinal
Priority: 10
(
 {TempLocation}
)
:loc
--> 
 {
//removes TempLoc annotation, gets the rule feature and adds a new Loc annotation
gate.AnnotationSet loc = (gate.AnnotationSet)bindings.get("loc");
gate.Annotation locAnn = (gate.Annotation)loc.iterator().next();
gate.FeatureMap features = Factory.newFeatureMap();
features.put("locType",locAnn.getFeatures().get("locType"));
features.put("rule1", locAnn.getFeatures().get("rule"));
features.put("rule2", "LocFinal");
annotations.add(loc.firstNode(), loc.lastNode(), "Location",
features);
annotations.removeAll(loc);
}


//////////////////////////////////////////////////////////////
// Rules from Timex


Rule: DateTimeFinal
Priority: 20
// Friday 10 January 2000 2pm

(
 {TempDate}
 (
  ({Token.string == ","})?
  {TempDate})?
 ({Token.string == ":"})?
 {TempTime}
 ({TempYear})?
 ({TempZone})?
)
:date
-->
 {
//removes TempDate annotation, gets the rule feature and adds a new Date annotation
gate.AnnotationSet date = (gate.AnnotationSet)bindings.get("date");
gate.Annotation dateAnn = (gate.Annotation)date.iterator().next();
gate.FeatureMap features = Factory.newFeatureMap();
//features.put("rule1", dateAnn.getFeatures().get("rule"));
features.put("rule2", "DateTimeFinal");
features.put("kind", "dateTime");
annotations.add(date.firstNode(), date.lastNode(), "Date",
features);
annotations.removeAll(date);
}


Rule: SeasonYearFinal
Priority: 15
(
 ({Token.string == "spring"} |
  {Token.string == "Spring"})
 {TempYear.kind == positive}
)
:date
-->
 {
//removes TempDate annotation, gets the rule feature and adds a new Date annotation
gate.AnnotationSet date = (gate.AnnotationSet)bindings.get("date");
gate.Annotation dateAnn = (gate.Annotation)date.iterator().next();
gate.FeatureMap features = Factory.newFeatureMap();
features.put("rule1", dateAnn.getFeatures().get("rule"));
features.put("rule2", "SeasonYearFinal");
features.put("kind", "date");
annotations.add(date.firstNode(), date.lastNode(), "Date",
features);
annotations.removeAll(date);
}


Rule: DateYearFinal
Priority: 10
(
 {TempDate}
 (
 ({Token.string == ","})?
 {TempDate})?
 {TempYear}
 ({TempDate})?
)
:date
-->
 {
//removes TempDate annotation, gets the rule feature and adds a new Date annotation
gate.AnnotationSet date = (gate.AnnotationSet)bindings.get("date");
gate.Annotation dateAnn = (gate.Annotation)date.iterator().next();
gate.FeatureMap features = Factory.newFeatureMap();
features.put("rule1", dateAnn.getFeatures().get("rule"));
features.put("rule2", "DateYearFinal");
features.put("kind", "date");
annotations.add(date.firstNode(), date.lastNode(), "Date",
features);
annotations.removeAll(date);
}


Rule: TimeDateFinal
Priority: 10
// 2pm 10 January 2000
// 2pm 10 January 2000 +0400
(
 {TempTime}
 ({Token.string == ":"})?
 {TempDate} 
 ({TempYear})?
 ({TempZone})?
)
:date
-->
 {
//removes TempDate annotation, gets the rule feature and adds a new Date annotation
gate.AnnotationSet date = (gate.AnnotationSet)bindings.get("date");
gate.Annotation dateAnn = (gate.Annotation)date.iterator().next();
gate.FeatureMap features = Factory.newFeatureMap();
//features.put("rule1", dateAnn.getFeatures().get("rule"));
features.put("rule2", "TimeDateFinal");
features.put("kind", "dateTime");
annotations.add(date.firstNode(), date.lastNode(), "Date",
features);
annotations.removeAll(date);
}


Rule: TimeYearFinal
Priority: 10
// 21:00:00 2000 +0400

(
 {TempTime}
 ({Token.string == ":"})? 
 ({TempYear})
 ({TempZone})?
)
:date
-->
  {
//removes TempDate annotation, gets the rule feature and adds a new Date annotation
gate.AnnotationSet date = (gate.AnnotationSet)bindings.get("date");
gate.Annotation dateAnn = (gate.Annotation)date.iterator().next();
gate.FeatureMap features = Factory.newFeatureMap();
//features.put("rule1", dateAnn.getFeatures().get("rule"));
features.put("rule2", "TimeYearFinal");
features.put("kind", "dateTime");
annotations.add(date.firstNode(), date.lastNode(), "Date",
features);
annotations.removeAll(date);
}


//Date Only Rules


Rule: DateOnlyFinal
Priority: 10
(
 {TempDate}
)
:date 
-->
 {
//removes TempDate annotation, gets the rule feature and adds a new Date annotation
gate.AnnotationSet date = (gate.AnnotationSet)bindings.get("date");
gate.Annotation dateAnn = (gate.Annotation)date.iterator().next();
gate.FeatureMap features = Factory.newFeatureMap();
features.put("rule1", dateAnn.getFeatures().get("rule"));
features.put("rule2", "DateOnlyFinal");
features.put("kind", "date");
annotations.add(date.firstNode(), date.lastNode(), "Date",
features);
annotations.removeAll(date);
}

//fix this later
Rule: TimeContextFinal
Priority: 10
// Wednesday [mdash ] 8-15

(
 ({TempTime}|{TempDate}):date
 {Token.string == "["}
 {Token.string == "mdash"}
 {Token.string == "]"}
)
( {TempTime.kind == temp}
):time
-->
  {
//removes TempDate annotation, gets the rule feature and adds a new Date annotation
gate.AnnotationSet date = (gate.AnnotationSet)bindings.get("date");
gate.Annotation dateAnn = (gate.Annotation)date.iterator().next();
gate.FeatureMap features = Factory.newFeatureMap();
features.put("rule1", dateAnn.getFeatures().get("rule"));
features.put("rule", "TimeContextFinal");
features.put("kind", "date");
annotations.add(date.firstNode(), date.lastNode(), "Date",
features);
annotations.removeAll(date);
//removes TempTime annotation, gets the rule feature and adds a new Date annotation
gate.AnnotationSet time = (gate.AnnotationSet)bindings.get("time");
gate.Annotation timeAnn = (gate.Annotation)time.iterator().next();
gate.FeatureMap features2 = Factory.newFeatureMap();
features2.put("rule1", timeAnn.getFeatures().get("rule"));
features2.put("rule", "TimeContextFinal");
features2.put("kind", "time");
annotations.add(time.firstNode(), date.lastNode(), "Date",
features2);
annotations.removeAll(time);
}


Rule: TimeWordsContextFinal
Priority: 50

//seven to nine o'clock
(
 {TempTime.kind == timeWords}
 {Token.string == "to"}
 {TempTime.kind == positive}
)
:date
-->
 {
//removes TempTime annotation, gets the rule feature and adds a new Date annotation
gate.AnnotationSet date = (gate.AnnotationSet)bindings.get("date");
gate.Annotation dateAnn = (gate.Annotation)date.iterator().next();
gate.FeatureMap features = Factory.newFeatureMap();
features.put("rule1", dateAnn.getFeatures().get("rule"));
features.put("rule2", "TimeWordsContextFinal");
features.put("kind", "time");
annotations.add(date.firstNode(), date.lastNode(), "Date",
features);
annotations.removeAll(date);
}


Rule: YearOnlyFinal
Priority: 10
(
 {TempYear.kind == positive}
)
:date
--> 
{
//removes TempDate annotation, gets the rule feature and adds a new Date annotation
gate.AnnotationSet date = (gate.AnnotationSet)bindings.get("date");
gate.Annotation dateAnn = (gate.Annotation)date.iterator().next();
gate.FeatureMap features = Factory.newFeatureMap();
features.put("rule1", dateAnn.getFeatures().get("rule"));
features.put("rule2", "YearOnlyFinal");
features.put("kind", "date");
annotations.add(date.firstNode(), date.lastNode(), "Date",
features);
annotations.removeAll(date);
}



Rule: TimeOnlyFinal
Priority: 10
(
 {TempTime.kind == positive}
)
:date
-->
{
//removes TempDate annotation, gets the rule feature and adds a new Date annotation
gate.AnnotationSet date = (gate.AnnotationSet)bindings.get("date");
gate.Annotation dateAnn = (gate.Annotation)date.iterator().next();
gate.FeatureMap features = Factory.newFeatureMap();
features.put("rule1", dateAnn.getFeatures().get("rule"));
features.put("rule2", "TimeOnlyFinal");
features.put("kind", "time");
annotations.add(date.firstNode(), date.lastNode(), "Date",
features);
annotations.removeAll(date);
}


////////////////////////////////////////////////////////////
Rule: AddressFull
Priority: 100
(
 ({Street}
  {Token.string == ","})?
 ({TempLocation} 
 ({Token.string == ","})?
 )+
 ({Postcode})
 ({Token.string == ","})?
 ({TempLocation})*
)
:address
-->
{
//removes TempAddress annotation, gets the rule feature and adds a new Address annotation
gate.AnnotationSet address = (gate.AnnotationSet)bindings.get("address");
gate.Annotation addressAnn = (gate.Annotation)address.iterator().next();
gate.FeatureMap features = Factory.newFeatureMap();
features.put("rule1", addressAnn.getFeatures().get("rule"));
features.put("rule2", "AddressFull");
features.put("kind", "complete");
annotations.add(address.firstNode(), address.lastNode(), "Address",
features);
annotations.removeAll(address);
}


Rule: EmailFinal
Priority: 50
(
{Email}
)
:address
-->
{
//removes Email annotation, gets the rule feature and adds a new Address annotation
gate.AnnotationSet address = (gate.AnnotationSet)bindings.get("address");
gate.Annotation addressAnn = (gate.Annotation)address.iterator().next();
gate.FeatureMap features = Factory.newFeatureMap();
features.put("rule1", addressAnn.getFeatures().get("rule"));
features.put("rule2", "EmailFinal");
features.put("kind", "email");
annotations.add(address.firstNode(), address.lastNode(), "Address",
features);
annotations.removeAll(address);
}


Rule: PhoneFinal
Priority: 50
(
{Phone}
)
:address
-->
{
//removes TempAddress annotation, gets the rule feature and adds a new Address annotation
gate.AnnotationSet address = (gate.AnnotationSet)bindings.get("address");
gate.Annotation addressAnn = (gate.Annotation)address.iterator().next();
gate.FeatureMap features = Factory.newFeatureMap();
features.put("rule1", addressAnn.getFeatures().get("rule"));
features.put("rule2", "PhoneFinal");
features.put("kind", "phone");
annotations.add(address.firstNode(), address.lastNode(), "Address",
features);
annotations.removeAll(address);
}


Rule: PostcodeFinal
Priority: 50
(
{Postcode}
)
:address
-->
{
//removes TempAddress annotation, gets the rule feature and adds a new Address annotation
gate.AnnotationSet address = (gate.AnnotationSet)bindings.get("address");
gate.Annotation addressAnn = (gate.Annotation)address.iterator().next();
gate.FeatureMap features = Factory.newFeatureMap();
features.put("rule1", addressAnn.getFeatures().get("rule"));
features.put("rule2", "PostcodeFinal");
features.put("kind", "postcode");
annotations.add(address.firstNode(), address.lastNode(), "Address",
features);
annotations.removeAll(address);
}


Rule: IpFinal
Priority: 50
(
{Ip}
)
:address
-->
{
//removes TempAddress annotation, gets the rule feature and adds a new Address annotation
gate.AnnotationSet address = (gate.AnnotationSet)bindings.get("address");
gate.Annotation addressAnn = (gate.Annotation)address.iterator().next();
gate.FeatureMap features = Factory.newFeatureMap();
features.put("rule1", addressAnn.getFeatures().get("rule"));
features.put("rule2", "IpFinal");
features.put("kind", "ip");
annotations.add(address.firstNode(), address.lastNode(), "Address",
features);
annotations.removeAll(address);
}


Rule: UrlFinal
Priority: 50
(
{Url}
)
:address
-->
{
//removes TempAddress annotation, gets the rule feature and adds a new Address annotation
gate.AnnotationSet address = (gate.AnnotationSet)bindings.get("address");
gate.Annotation addressAnn = (gate.Annotation)address.iterator().next();
gate.FeatureMap features = Factory.newFeatureMap();
features.put("rule1", addressAnn.getFeatures().get("rule"));
features.put("rule2", "UrlFinal");
features.put("kind", "url");
annotations.add(address.firstNode(), address.lastNode(), "Address",
features);
annotations.removeAll(address);
}


Rule: StreetFinal
//make streets locations
Priority: 50
(
{Street}
)
:address
-->
{
//removes TempAddress annotation, gets the rule feature and adds a new Address annotation
gate.AnnotationSet address = (gate.AnnotationSet)bindings.get("address");
gate.Annotation addressAnn = (gate.Annotation)address.iterator().next();
gate.FeatureMap features = Factory.newFeatureMap();
features.put("rule1", addressAnn.getFeatures().get("rule"));
features.put("rule2", "StreetFinal");
annotations.add(address.firstNode(), address.lastNode(), "Location",
features);
annotations.removeAll(address);
}

////////////////////////////////////////////////////////////


Rule: IdentifierFinal
Priority: 10

(
 {TempIdentifier}
)
:ident
-->
{
//removes TempIdent annotation, gets the rule feature and adds a new Identifier annotation
gate.AnnotationSet ident = (gate.AnnotationSet)bindings.get("ident");
gate.Annotation identAnn = (gate.Annotation)ident.iterator().next();
gate.FeatureMap features = Factory.newFeatureMap();
features.put("rule1", identAnn.getFeatures().get("rule"));
features.put("rule2", "IdentifierFinal");
annotations.add(ident.firstNode(), ident.lastNode(), "Identifier",
features);
annotations.removeAll(ident);
}



// this gets used when specs rule for emails is fired (in eml-final.jape) 

Rule: SpecsFinal
Priority: 1000

(
 {TempSpecs}
):spec
-->
{
//removes TempSpecs annotation
gate.AnnotationSet spec = (gate.AnnotationSet)bindings.get("spec");
//gate.FeatureMap features = Factory.newFeatureMap();
annotations.removeAll(spec);
}

//////////////////////////////////////////////////////

Rule: UnknownPerson
Priority: 5
( 
 {Token.category == NNP}
 ((SPACE|{Token.string == "-"})
  {Token.category == NNP})?
 (SPACE {Token.category == NNP})?
 (SPACE {Token.category == NNP})?
):unknown
 (SPACE)
(
 {TempPerson}
):person
-->
:unknown.Unknown = {kind = "PN", rule = UnknownTempPerson},
{
//removes TempPerson annotation, gets the rule feature and adds a new Person annotation
gate.AnnotationSet person = (gate.AnnotationSet)bindings.get("person");
gate.Annotation personAnn = (gate.Annotation)person.iterator().next();
gate.FeatureMap features = Factory.newFeatureMap();
features.put("gender", personAnn.getFeatures().get("gender"));
features.put("rule1", personAnn.getFeatures().get("rule"));
features.put("rule2", "UnknownPerson");
annotations.add(person.firstNode(), person.lastNode(), "Person",
features);
annotations.removeAll(person);
}

/*
*  unknown.jape
*
* Copyright (c) 1998-2001, The University of Sheffield.
*
*  This file is part of GATE (see http://gate.ac.uk/), and is free
*  software, licenced under the GNU Library General Public License,
*  Version 2, June 1991 (in the distribution as file licence.html,
*  and also available at http://gate.ac.uk/gate/licence.html).
*
*  Diana Maynard, 10 Sep 2001
* 
*  $Id: grammars.tex,v 1.2 2001/11/29 18:07:27 diana Exp $
*/

Phase:	Unknown
Input: Location Person Date Organization Address Money Percent Token Jobtitle Lookup
Options: control = appelt


Rule: Known
Priority: 100
(
 {Location}| 
 {Person}|
 {Date}|
 {Organization}|
 {Address}|
 {Money} |
 {Percent}|
 {Token.string == "Dear"}|
 {JobTitle}|
 {Lookup}
):known
-->
{}
 

Rule:Unknown
Priority: 50
( 
 {Token.category == NNP}
) 
:unknown
-->
 :unknown.Unknown = {kind = "PN", rule = Unknown}



/*
*  name_context.jape
*
* Copyright (c) 1998-2001, The University of Sheffield.
*
*  This file is part of GATE (see http://gate.ac.uk/), and is free
*  software, licenced under the GNU Library General Public License,
*  Version 2, June 1991 (in the distribution as file licence.html,
*  and also available at http://gate.ac.uk/gate/licence.html).
*
*  Diana Maynard, 10 Sep 2001
* 
*  $Id: grammars.tex,v 1.2 2001/11/29 18:07:27 diana Exp $
*/

Phase:	NameContext
Input: Lookup Unknown Person Token Organization
Options: control = appelt

Rule: Jobtitle1
Priority: 50
(
 {Lookup.majorType == jobtitle}
)
(
 {Unknown}
 ({Unknown})?
)
:person
-->
{
//removes old "Unknown" annotation and adds a "Person" one
gate.AnnotationSet person = (gate.AnnotationSet) bindings.get("person");
gate.FeatureMap features = Factory.newFeatureMap();
features.put("rule", "JobTitle1");
annotations.add(person.firstNode(), person.lastNode(), "Person",
features);
annotations.removeAll(person);
}


Rule:PersonTitle1
Priority: 40
(
 {Person.rule1 == PersonTitle}
 {Unknown}
):person
-->
{
//removes old "Person" and "Unknown" annotations and adds a new "Person" one
gate.AnnotationSet person = (gate.AnnotationSet) bindings.get("person");
gate.FeatureMap features = Factory.newFeatureMap();
features.put("rule", "PersonTitle1");
annotations.add(person.firstNode(), person.lastNode(), "Person",
features);
annotations.removeAll(person);
}



Rule: PersonContext1
Priority: 10
//note: this should really move to grammar following co-ref

(
 {Token.orth == upperInitial}
):person

(
 {Token.string == "from"}
 ({Organization} | {Location})
)
-->
  :person.Person = {rule = "PersonContext1"}


Rule: NotPersonContext1
Priority: 20
//if the unknown thing is already an org or person, don't change it


(
 ({Organization}|{Person})
):label

(
 {Token.string == "from"}
 ({Organization} | {Location})
)
-->
  {}

/*  org_context.jape
*
* Copyright (c) 1998-2001, The University of Sheffield.
*
*  This file is part of GATE (see http://gate.ac.uk/), and is free
*  software, licenced under the GNU Library General Public License,
*  Version 2, June 1991 (in the distribution as file licence.html,
*  and also available at http://gate.ac.uk/gate/licence.html).
*
*  Diana Maynard, 10 Sep 2001
* 
*  $Id: grammars.tex,v 1.2 2001/11/29 18:07:27 diana Exp $
*/


Phase:	Org_Context
Input: Token Lookup Organization Unknown Location
Options: control = appelt



Rule:OrgContext1
Priority: 1
// company X
// company called X

(
 {Token.string == "company"}
 (({Token.string == "called"}|
   {Token.string == "dubbed"}|
   {Token.string == "named"}
  )
 )?
)
(
 {Unknown.kind == PN}
)
:org
-->
{
gate.AnnotationSet org = (gate.AnnotationSet) bindings.get("org");
gate.FeatureMap features = Factory.newFeatureMap();
features.put("rule ", "OrgContext1");
annotations.add(org.firstNode(), org.lastNode(), "Organization",
features);
annotations.removeAll(org);
}

Rule: OrgContext2
Priority: 5

// Telstar laboratory
// Medici offices
(
 {Unknown.kind == PN}
): org
(
 ({Token.string == "offices"} |
 {Token.string == "Offices"} |
 {Token.string == "laboratory"} | 
 {Token.string == "Laboratory"} |
 {Token.string == "laboratories"} |
 {Token.string == "Laboratories"})
)
-->
{
gate.AnnotationSet org = (gate.AnnotationSet) bindings.get("org");
gate.FeatureMap features = Factory.newFeatureMap();
features.put("rule ", "OrgContext2");
annotations.add(org.firstNode(), org.lastNode(), "Organization",
features);
annotations.removeAll(org);
}

Rule:OrgContext3
Priority: 5
// X shares

(
 {Unknown.kind == PN}
):org
( 
 {Token.string == "shares"}
)
-->
{
gate.AnnotationSet org = (gate.AnnotationSet) bindings.get("org");
gate.FeatureMap features = Factory.newFeatureMap();
features.put("rule ", "OrgContext3");
annotations.add(org.firstNode(), org.lastNode(), "Organization",
features);
annotations.removeAll(org);
}

Rule:OrgContext4
Priority: 10
// shares in X

( 
 {Token.string == "shares"}
 {Token.string == "in"}
)
(
 {Unknown.kind == PN}
):org
-->
{
gate.AnnotationSet org = (gate.AnnotationSet) bindings.get("org");
gate.FeatureMap features = Factory.newFeatureMap();
features.put("rule ", "OrgContext4");
annotations.add(org.firstNode(), org.lastNode(), "Organization",
features);
annotations.removeAll(org);
}


Rule: OrgContext5
Priority: 10
// officials at X

(
 ({Token.string == "officials"}|
  {Token.string == "Officials"})
 {Token.string == "at"}
)
(
 {Unknown.kind == PN}
):org
-->
 {
gate.AnnotationSet org = (gate.AnnotationSet) bindings.get("org");
gate.FeatureMap features = Factory.newFeatureMap();
features.put("rule ", "OrgContext5");
annotations.add(org.firstNode(), org.lastNode(), "Organization",
features);
annotations.removeAll(org);
}


Rule:JoinOrg
Priority: 50
// Smith joined Energis
// later we should use the morph PR to prevent having to list morphological variants

(
 ({Token.string == "joined"}|
  {Token.string == "joining"}|
  {Token.string == "joins"}|
  {Token.string == "join"}
 )
 SPACE
)
( 
 {Unknown.kind ==PN}
)
:org
-->
 {
gate.AnnotationSet org = (gate.AnnotationSet) bindings.get("org");
gate.FeatureMap features = Factory.newFeatureMap();
features.put("rule ", "JoinOrg");
annotations.add(org.firstNode(), org.lastNode(), "Organization",
features);
annotations.removeAll(org);
}


Rule:OrgPerson
Priority: 20
// Nokia Vice-President William Plummer

(
 {Unknown.kind == PN}
):org 
(
 ({Token.string == "'"}
  ({Token.string == "s"})?
 )?
 SPACE
 {Person.rule1 == PersonTitle}
)
--> 
 {
//get the matched annotation(s)
gate.AnnotationSet org = (gate.AnnotationSet) bindings.get("org");

//create the new annotation
gate.FeatureMap features = Factory.newFeatureMap();
features.put("rule ", "OrgPerson");
annotations.add(org.firstNode(), org.lastNode(), "Organization",
features);

//delete the old annotation(s)
annotations.removeAll(org);
}


Rule: OrgConjOrg1 
Priority: 10

(
{Unknown.kind == PN}
):org
(
(SPACE)
{Token.category == CC}
(SPACE)
{Organization}
)
-->
{
gate.AnnotationSet org = (gate.AnnotationSet) bindings.get("org");
gate.FeatureMap features = Factory.newFeatureMap();
features.put("rule ", "OrgConjOrg1");
annotations.add(org.firstNode(), org.lastNode(), "Organization",
features);
annotations.removeAll(org);
}


Rule: OrgConjOrg2
Priority: 10

(
 {Organization}
 (SPACE)
 {Token.category == CC}
 (SPACE)
)
(
 {Unknown.kind == PN}
):org
-->
 {
gate.AnnotationSet org = (gate.AnnotationSet) bindings.get("org");
gate.FeatureMap features = Factory.newFeatureMap();
features.put("rule ", "OrgConjOrg2");
annotations.add(org.firstNode(), org.lastNode(), "Organization",
features);
annotations.removeAll(org);
}



Rule: OrgJobtitle
Priority: 30
(
 {Unknown.kind == PN}
):org
( 
 SPACE
 {Lookup.majorType == jobtitle}
)
-->
  {
gate.AnnotationSet org = (gate.AnnotationSet) bindings.get("org");
gate.FeatureMap features = Factory.newFeatureMap();
features.put("rule ", "OrgJobTitle");
annotations.add(org.firstNode(), org.lastNode(), "Organization",
features);
annotations.removeAll(org);
}



Rule:LocOrg
Priority: 20
// guess that Unknown preceded by Loc is an Org

(
 {Location}
 {Unknown.kind == PN}
):org
-->
{
//removes Unknown annotation, adds a new Org annotation
gate.AnnotationSet org = (gate.AnnotationSet) bindings.get("org");
gate.FeatureMap features = Factory.newFeatureMap();
features.put("rule", "LocOrg");
annotations.add(org.firstNode(), org.lastNode(), "Organization",
features);
annotations.removeAll(org);
}
/*
*  loc_context.jape
*
* Copyright (c) 1998-2001, The University of Sheffield.
*
*  This file is part of GATE (see http://gate.ac.uk/), and is free
*  software, licenced under the GNU Library General Public License,
*  Version 2, June 1991 (in the distribution as file licence.html,
*  and also available at http://gate.ac.uk/gate/licence.html).
*
*  Diana Maynard, 02 Aug 2001
* 
*  $Id: grammars.tex,v 1.2 2001/11/29 18:07:27 diana Exp $
*/

Phase:	Loc_Context
Input: Unknown Token Location
Options: control = appelt


Rule: LocConjLoc1 
Priority: 10

(
{Unknown.kind == PN}
):loc
(
{Token.category == CC}
({Token.category == DT}
)?
{Location}
)
-->
{
gate.AnnotationSet loc = (gate.AnnotationSet) bindings.get("loc");
gate.FeatureMap features = Factory.newFeatureMap();
features.put("rule ", "LocConjLoc1");
annotations.add(loc.firstNode(), loc.lastNode(), "Location",
features);
annotations.removeAll(loc);
}


Rule: LocConjLoc2
Priority: 10

(
 {Location}
 {Token.category == CC}
 ({Token.category == DT}
 )?
)
(
 {Unknown.kind == PN}
):loc
-->
 {
gate.AnnotationSet loc = (gate.AnnotationSet) bindings.get("loc");
gate.FeatureMap features = Factory.newFeatureMap();
features.put("rule ", "LocConjLoc2");
annotations.add(loc.firstNode(), loc.lastNode(), "Location",
features);
annotations.removeAll(loc);
}


Rule: UnknownLocRegion
Priority: 50
(
 ({Token.string == "at"}|
  {Token.string == "in"}
 )
)
( 
 {Unknown}
):loc
(
 {Token.string == ","}
 {Location.kind == region}
)
-->
 :loc.Location = {rule = "UnknownLocRegion"}


/*
*  clean.jape
*
* Copyright (c) 1998-2001, The University of Sheffield.
*
*  This file is part of GATE (see http://gate.ac.uk/), and is free
*  software, licenced under the GNU Library General Public License,
*  Version 2, June 1991 (in the distribution as file licence.html,
*  and also available at http://gate.ac.uk/gate/licence.html).
*
*  Diana Maynard, 10 Sep 2001
* 
*  $Id: grammars.tex,v 1.2 2001/11/29 18:07:27 diana Exp $
*/

Phase:	Clean
Input: TempPerson TempLocation TempOrganization TempDate TempTime TempYear TempZone Street Postcode Email Url Phone Ip TempIdentifier TempSpecs
Options: control = appelt

Rule:CleanTempAnnotations
(
 {TempPerson}|
 {TempLocation}|
 {TempOrganization}|
 {TempDate}|
 {TempTime}|
 {TempYear}|
 {TempZone}|
 {Street}|
 {Postcode}|
 {Email}|
 {Url}|
 {Phone}|
 {Ip}|
 {TempIdentifier}|
 {TempSpecs}
):temp
-->
{
 gate.AnnotationSet temp = (gate.AnnotationSet)bindings.get("temp");
 annotations.removeAll(temp);
}

\end{verbatim}







