
%%%%%%%%%%%%%%%%%%%%%%%%%%%%%%%%%%%%%%%%%%%%%%%%%%%%%%%%%%%%%%%%%%%%%%%%%%%%%
\chapt[chap:gettingstarted]{Installing and Running GATE}
\markboth{Installing and Running GATE}{Installing and Running GATE}
%%%%%%%%%%%%%%%%%%%%%%%%%%%%%%%%%%%%%%%%%%%%%%%%%%%%%%%%%%%%%%%%%%%%%%%%%%%%%


%%%%%%%%%%%%%%%%%%%%%%%%%%%%%%%%%%%%%%%%%%%%%%%%%%%%%%%%%%%%%%%%%%%%%%%%%%%%%
\sect[sec:gettingstarted:download]{Downloading GATE}
%%%%%%%%%%%%%%%%%%%%%%%%%%%%%%%%%%%%%%%%%%%%%%%%%%%%%%%%%%%%%%%%%%%%%%%%%%%%%

To download GATE point your web browser at
\htlinkplain{http://gate.ac.uk/download/}.

%You should next read the section \ref{sec:howto:install} to install and run
%GATE Developer.

%%%%%%%%%%%%%%%%%%%%%%%%%%%%%%%%%%%%%%%%%%%%%%%%%%%%%%%%%%%%%%%%%%%%%%%%%%%%%
\sect[sec:gettingstarted:install]{Installing and Running GATE}
%%%%%%%%%%%%%%%%%%%%%%%%%%%%%%%%%%%%%%%%%%%%%%%%%%%%%%%%%%%%%%%%%%%%%%%%%%%%%

\mbox{ }

GATE will run anywhere that supports Java 7 or later, including Solaris, Linux,
Mac OS X and Windows platforms. We don't run tests on other platforms, but have
had reports of successful installs elsewhere.

%%%%%%%%%%%%%%%%%%%%%%%%%%%%%%%%%%%%%%%%%%%%%%%%%%%%%%%%%%%%%%%%%%%%%%%%%%%%%
\subsect[sec:gettingstarted:easy]{The Easy Way}
%%%%%%%%%%%%%%%%%%%%%%%%%%%%%%%%%%%%%%%%%%%%%%%%%%%%%%%%%%%%%%%%%%%%%%%%%%%%%

The easy way to install is to use one of the platform-specific installers
(created using the excellent \htlink{http://izpack.org/}{IzPack}).
\htlink{http://gate.ac.uk/download/}{Download a `platform-specific
installer'} and follow the instructions it gives
you. Once the installation is complete, you can start GATE Developer using
\verb|gate.exe| (Windows) or \verb|GATE.app| (Mac) in the top-level
installation directory, on Linux and other platforms use \verb|gate.sh| 
in the \texttt{bin} directory (see section~\ref{sec:gettingstarted:runonlinux}).

%%%%%%%%%%%%%%%%%%%%%%%%%%%%%%%%%%%%%%%%%%%%%%%%%%%%%%%%%%%%%%%%%%%%%%%%%%%%%
\subsect[sec:gettingstarted:hard]{The Hard Way (1)}
%%%%%%%%%%%%%%%%%%%%%%%%%%%%%%%%%%%%%%%%%%%%%%%%%%%%%%%%%%%%%%%%%%%%%%%%%%%%%

Download the Java-only release package or the binary build snapshot, and follow 
the instructions below.

{\bf Prerequisites:}
\begin{itemize}
\item
A conforming Java 2 environment,
  \begin{itemize}
  \item version 1.4.2 or above for GATE 3.1
  \item version 5.0 for GATE 4.0 beta 1 or later.
  \item version 6.0 for GATE 6.1 or later.
  \item version 7.0 for GATE 8.0 or later.
  \end{itemize}
available free from
\htlink{http://www.oracle.com/technetwork/java/javase/overview/index.html}{Oracle}
or from your UNIX supplier. (We test on various Sun
JDKs on Solaris, Linux and Windows XP.)
\item Binaries from the GATE distribution you downloaded: {\tt gate.jar} (which
can be found in the directory called {\tt bin}). You will also need the {\tt
lib} directory, containing various libraries that GATE depends on.
\item a suitable \htlink{http://ant.apache.org/}{Apache ANT} installation
(version 1.8.1 or newer). You will need to add an environment variable named
\verb!ANT_HOME! pointing to your ANT installation, and add \verb!ANT_HOME/bin!
to your {\tt PATH}.

\item An open mind and a sense of humour.
\end{itemize}

Using the binary distribution:
\begin{itemize}
\item
Unpack the distribution, creating a directory containing jar files and
scripts.
\item To run GATE Developer: 
  \begin{itemize}
  \item on Windows, start a Command Prompt window, change to the directory where
    you unpacked the GATE distribution and run `\verb!bin\gate.bat!';
  \item on Windows using the GUI, double-click the `\verb!gate.exe!' file;
  \item on UNIX/Linux or Mac open a terminal window and run
    `\verb!bin/gate.sh!'.
  \end{itemize}
\item
To embed GATE as a library (GATE Embedded), put {\tt gate.jar} and all
the libraries in the lib directory in your {\tt CLASSPATH}.
\end{itemize}

The Ant scripts that start GATE Developer ({\tt ant.bat} or {\tt ant})
require you to set the {\tt JAVA\_HOME} environment variable to point to the
top level directory of your JAVA installation.
%
The value of {\tt GATE\_CONFIG} is passed to the system by the scripts using
either a {\tt -i} command-line option, or the Java property {\tt
gate.site.config}.


%%%%%%%%%%%%%%%%%%%%%%%%%%%%%%%%%%%%%%%%%%%%%%%%%%%%%%%%%%%%%%%%%%%%%%%%%%%%%
\subsect[sec:gettingstarted:svn]{The Hard Way (2): Subversion}
%%%%%%%%%%%%%%%%%%%%%%%%%%%%%%%%%%%%%%%%%%%%%%%%%%%%%%%%%%%%%%%%%%%%%%%%%%%%%

The GATE code is maintained in a
\htlink{http://subversion.tigris.org}{Subversion} repository.  You can use
a Subversion client to check out the source code -- the most up-to-date version
of GATE is the trunk:
\\
{\tt svn checkout http://svn.code.sf.net/p/gate/code/gate/trunk gate}

Once you have checked out the code you can build GATE using Ant (see
Section~\ref{sec:gettingstarted:build})

You can browse the complete Subversion repository online at\\
\htlinkplain{https://sourceforge.net/p/gate/code/HEAD/tree/}.

%%%%%%%%%%%%%%%%%%%%%%%%%%%%%%%%%%%%%%%%%%%%%%%%%%%%%%%%%%%%%%%%%%%%%%%%%%%%%
\subsect[sec:gettingstarted:runonlinux]{Running GATE Developer on Unix/Linux}
%%%%%%%%%%%%%%%%%%%%%%%%%%%%%%%%%%%%%%%%%%%%%%%%%%%%%%%%%%%%%%%%%%%%%%%%%%%%%
%
The script \verb|gate.sh| in the directory \verb|bin| of your installation
can be used to start GATE Developer. You can run this script by entering
its full path in a terminal or by adding the \verb|bin| directory to your 
binary path. In addition you can also add a symbolic link to this script 
in any directory that already is in your binary path.

If \verb|gate.sh| is invoked without parameters, GATE Developer will use the 
files \verb|~/.gate.xml| and \verb|~/.gate.session| to store session and 
configuration data. Alternately you can run \verb|gate.sh| with the 
following parameters:
\begin{description}
  \item[-h] show usage information
  \item[-ld] create or use the files \texttt{.gate.session} and 
     \texttt{.gate.xml} in the current directory as the session and
     configuration files. If option \texttt{-dc }\emph{DIR} occurs before this option,
     the file \texttt{.gate.session} is created from \emph{DIR}\texttt{/default.session}
     if it does not already exist and the file \texttt{.gate.xml} is created 
     from \emph{DIR}\texttt{/default.xml} if it does not already exist.
  \item[-ln \emph{NAME}] create or use \emph{NAME}\texttt{.session} and
    \emph{NAME}\texttt{.xml} in the current directory as the session and 
    configuration files. If option \texttt{-dc DIR} occurs before this option,
     the file \emph{NAME}\texttt{.session} is created from \emph{DIR}\texttt{/default.session}
     if it does not already exist and the file \emph{DIR}\texttt{.xml} is created 
     from \emph{DIR}\texttt{/default.xml} if it does not already exist.
  \item[-ll] if the current directory contains a file named \texttt{log4j.properties} 
    then use it instead of the default (\verb|GATE_HOME/bin/log4j.properties|) to
    configure logging. Alternately, you can specify any log4j configuration file 
    by setting the \texttt{log4j.configuration} property explicitly (see below).
  \item[-rh \emph{LOCATION}] set the resources home directory to the \emph{LOCATION} provided. 
    If a resources home location is provided, the URLs in a saved application are
    saved relative to this location instead of relative to the application state file
    (see section~\ref{sec:developer:savestate}).
    This is equivalent to setting the property \texttt{gate.user.resourceshome} to
    this location.
  \item[-d \emph{URL}] loads the CREOLE plugin at the given URL during the
  start-up process.
  \item[-i \emph{FILE}] uses the specified file as the site configuration.
  \item[-dc \emph{DIR}] copy \texttt{default.xml} and/or \texttt{default.session} 
    from the directory \emph{DIR} when creating a new config or session file. This option works
    only together with either the \texttt{-ln}, \texttt{-ll} or \texttt{-tmp} option 
    and must occur before \texttt{-ln}, \texttt{-ll} or \texttt{-tmp}. 
    An existing config or session file 
    is used, but if it does not exist, the file from the 
    given directory is copied to create the file instead of using an empty/default file.
  \item[-tmp] creates temporary configuration and session files in the current
    directory, optionally copying \texttt{default.xml} and \texttt{default.session}
    from the directory specified with a \texttt{-dc DIR} option that occurs
    before it. After GATE exits, those session and config files are removed.
  \item[\emph{all other parameters}] are passed on to the \texttt{java} command. 
     This can be used to e.g. set properties using the \texttt{java} option
     \texttt{-D}. For example to set the maximum amount of heap memory to be
     used when running GATE to 6000M, you can add
     \texttt{-Xmx6000m} as a parameter.
     In order to change the default encoding used by GATE to \texttt{UTF-8} add 
     \texttt{-Dfile.encoding=utf-8} as a parameter. To specify 
     a log4j configuration file add something like \\
     \texttt{-Dlog4j.configuration=file:///home/myuser/log4jconfig.properties}.
\end{description}
Running GATE Developer with either the \texttt{-ld} or the \texttt{-ln} 
option from different directories is useful to keep several projects 
separate and can be used to run multiple instances of GATE Developer (or
even different versions of GATE Developer) in succession or even simultanously 
without the configuration files getting mixed up between them.

%%%%%%%%%%%%%%%%%%%%%%%%%%%%%%%%%%%%%%%%%%%%%%%%%%%%%%%%%%%%%%%%%%%%%%%%%%%%%
\sect[sec:gettingstarted:sysprop]{Using System Properties with GATE}
%%%%%%%%%%%%%%%%%%%%%%%%%%%%%%%%%%%%%%%%%%%%%%%%%%%%%%%%%%%%%%%%%%%%%%%%%%%%%

During initialisation, GATE reads several Java
system properties in order to decide where to find its configuration
files.

Here is a list of the properties used, their default values and their meanings:
\begin{description}
\item[gate.home] sets the location of the GATE install directory. This should
point to the top level directory of your GATE installation. This is the only
property that is required. If this is not set, the system will display an error
message and them it will attempt to guess the correct value.
\item[gate.plugins.home] points to the location of the directory containing 
installed plugins (a.k.a. CREOLE directories). If this is not set then the
default value of {\tt \{gate.home\}/plugins} is used.
\item[gate.site.config] points to the location of the configuration file
containing the site-wide options. If not set this will default to {\tt
\{gate.home\}/gate.xml}. The site configuration file must exist!
\item[gate.user.config] points to the file containing the user's options. If
not specified, or if the specified file does not exist at startup time, the
default value of gate.xml (.gate.xml on Unix platforms) in the user's home
directory is used.
\item[gate.user.session] points to the file containing the user's saved
session.  If not specified, the default value of gate.session
(.gate.session on Unix) in the user's home directory is used.  When
starting up GATE Developer, the session is reloaded from this file if
it exists, and when exiting GATE Developer the session is saved to
this file (unless the user has disabled `save session on exit' in
the configuration dialog).  The session is not used when using GATE Embedded.
\item[gate.user.filechooser.defaultdir] sets the default directory to
be shown in the file chooser of GATE Developer  
to the specified directory instead of the 
user's operating-system specific default directory.
\item[load.plugin.path] is a path-like structure, i.e. a list of URLs
separated by `;'.
All directories listed here will be loaded as CREOLE plugins during
initialisation. This has similar functionality with the the {\tt -d} command line 
option.
\item[gate.builtin.creole.dir] is a URL pointing to the location of GATE's
built-in CREOLE directory.  This is the location of the {\tt
creole.xml} file that defines the fundamental GATE resource types,
such as documents, document format handlers, controllers and the basic
visual resources that make up GATE.  The default points to a
location inside {\tt gate.jar} and should not generally need to be
overridden.
\end{description}

When using GATE Embedded, you can set the values for these properties before
you call {\tt Gate.init()}.  Alternatively, you can set the values
programmatically using the static methods {\tt setGateHome()},
{\tt setPluginsHome()}, {\tt setSiteConfigFile()}, etc.  before calling
{\tt Gate.init()}.  See the Javadoc documentation for details.
If you want to set these values from the command line you can use the following
syntax for setting {\tt gate.home} for example:

{\tt java -Dgate.home=/my/new/gate/home/directory -cp... gate.Main}

When running GATE Developer, you can set the properties by creating a file
{\tt build.properties} in the top level GATE directory.  In this file, any
system properties which are prefixed with `{\tt run.}' will be passed to
GATE.  For example, to set an alternative user config file, put the following
line in {\tt build.properties}\footnote{In this specific case, the alternative
config file must already exist when GATE starts up, so you should copy your
standard {\tt gate.xml} file to the new location.}:

{\tt run.gate.user.config=\$\{user.home\}/alternative-gate.xml}

This facility is not limited to the GATE-specific properties listed above, for
example the following line changes the default temporary directory for GATE
(note the use of forward slashes, even on Windows platforms):

{\tt run.java.io.tmpdir=d:/bigtmp}

When running GATE Developer from the command line via \verb|ant| or via the
{\tt gate.sh} script you can set properties using \verb!-D!.  Note that the
``run'' prefix is required when using ant:

\texttt{ant run -Drun.gate.user.config=/my/path/to/user/config.file}

but not when using {\tt gate.sh}:

\texttt{./bin/gate.sh -Dgate.user.config=/my/path/to/user/config.file} 

The GATE Developer launcher also supports the system property
{\tt gate.class.path} to specify additional classpath entries that should be
added to the classloader that is used to load GATE classes.  This is expected
to be in the normal ``classpath'' format, i.e. a list of directory or JAR file
paths separated by semicolons on Windows and colons on other platforms.  The
standard Java 6 shorthand of {\tt /path/to/directory/*}\footnote{Remember to
protect the * from expansion by your shell if necessary.} to include all
{\tt .jar} files from a given directory is also supported.  As an alternative
to this system property, the environment variable \verb!GATE_CLASSPATH! can be
used, but the environment variable is only read if the system property is
{\em not} set.

\texttt{./bin/gate.sh -Dgate.class.path=/shared/lib/myclasses.jar}

%%%%%%%%%%%%%%%%%%%%%%%%%%%%%%%%%%%%%%%%%%%%%%%%%%%%%%%%%%%%%%%%%%%%%%%%%%%%%
\sect[sec:gettingstarted:launchconfig]{Changing GATE's launch configuration}
%%%%%%%%%%%%%%%%%%%%%%%%%%%%%%%%%%%%%%%%%%%%%%%%%%%%%%%%%%%%%%%%%%%%%%%%%%%%%
%
With effect from build 4723 (13 November 2013), all the JVM and GATE launch
options can be set in the \texttt{gate.l4j.ini} file on all platforms, as well
as by using options to the \texttt{gate.sh} command.  

The \texttt{gate.l4j.ini} file supplied by default with GATE simply sets two
standard JVM memory options:
%
\begin{verbatim}
-Xmx1G
-Xms200m
\end{verbatim}
%
\texttt{-Xmx} specifies the maximum heap size in megabytes (\texttt{m}) or
gigabytes (\texttt{g}), and -Xms specifies the initial size. Other parameters of
interest are \texttt{-XX:MaxPermSize=128m} for the "permanent generation", which
you may wish to specify if you are getting OutOfMemoryErrors that say "PermGen
space".



Note that the format consists of one option per line.  All the properties listed
in Section~\ref{sec:gettingstarted:sysprop} can be configured here by prefixing
them with \texttt{-D}, e.g., \texttt{-Dgate.user.config=path/to/other-gate.xml}.


Proxy configuration (see Section~\ref{sec:misc-creole:proxy}) can now be set in
this file by adding these lines and editing them as needed for your
configuration.
%
\begin{verbatim}
-Drun.java.net.useSystemProxies=true  
-Dhttp.proxyHost=proxy.example.com  
-Dhttp.proxyPort=8080  
-Dhttp.nonProxyHosts=*.example.com 
\end{verbatim}
%
%%%%%%%%%%%%%%%%%%%%%%%%%%%%%%%%%%%%%%%%%%%%%%%%%%%%%%%%%%%%%%%%%%%%%%%%%%%%%
\sect[sec:gettingstarted:gateconfig]{Configuring GATE}
%%%%%%%%%%%%%%%%%%%%%%%%%%%%%%%%%%%%%%%%%%%%%%%%%%%%%%%%%%%%%%%%%%%%%%%%%%%%%

When GATE Developer is started, or when {\tt Gate.init()} is called from GATE
Embedded, GATE loads various sorts of configuration data stored as XML in files
generally called something like {\tt gate.xml} or {\tt .gate.xml}. This data
holds information such as:
\begin{itemize}
  
\item whether to save settings on exit;

\item whether to save session on exit;

\item what fonts GATE Developer should use;

\item plugins to load at start;

\item colours of the annotations;

\item locations of files for the file chooser;

\item and a lot of other GUI related options;

%\item where the local Oracle database lives.

\end{itemize}
%
This type of data is stored at two levels (in order from general to
specific):
%
\begin{itemize}
\item
  the site-wide level, which by default is located the {\tt gate.xml} file in
  top level directory of the GATE installation (i.e. the {\tt GATE home}.
  This location can be overridden by the Java system property {\tt
  gate.site.config};
\item
  the user level, which lives in the user's HOME directory on
  UNIX or their profile directory on Windows (note that parts of this file
  are overwritten when saving user settings).
  The default location for this file can be overridden by the Java system
  property {\tt gate.user.config}.
\end{itemize}
%
Where configuration data appears on several different levels, the
more specific ones overwrite the more general. This means that you can set
defaults for all GATE users on your system, for example, and allow individual
users to override those defaults without interfering with others.

Configuration data can be set from the GATE Developer GUI via the
`Options' menu then `Configuration'. The user can change the
appearance of the GUI in the `Appearance' tab, which includes the
options of font and the `look and feel'. The `Advanced' tab
enables the user to include annotation features when saving the
document and preserving its format, to save the selected Options
automatically on exit, and to save the session automatically on exit.
The `Input Methods' submenu from the `Options' menu enables the
user to change the default language for input. These options are all
stored in the user's .gate.xml file.

When using GATE Embedded, you can also set the site config location
using\linebreak {\tt Gate.setSiteConfigFile(File)} prior to calling {\tt
Gate.init()}.




%%%%%%%%%%%%%%%%%%%%%%%%%%%%%%%%%%%%%%%%%%%%%%%%%%%%%%%%%%%%%%%%%%%%%%%%%%%%%
\sect[sec:gettingstarted:build]{Building GATE}
%%%%%%%%%%%%%%%%%%%%%%%%%%%%%%%%%%%%%%%%%%%%%%%%%%%%%%%%%%%%%%%%%%%%%%%%%%%%%

\mbox{ }

Note that you don't need to build GATE unless you're doing development
on the system itself.

{\bf Prerequisites:}
\begin{itemize}
\item
A conforming Java environment as above.
\item
A copy of the GATE sources and the build scripts -- either the SRC distribution 
package from the nightly snapshots or a copy of the code
obtained through Subversion (see Section~\ref{sec:gettingstarted:svn}).
\item 
A working installation of \htlink{http://ant.apache.org/}{Apache ANT} version
1.8.1 or newer. You will need to add an environment variable named
\verb!ANT_HOME! pointing to your ANT installation, and add \verb!ANT_HOME/bin!
to your {\tt PATH}. It is advisable that you also set your \verb!JAVA_HOME!
environment variable to point to the top-level directory of your Java
installation.
\item
An appreciation of natural beauty.
\end{itemize}

{\bf To build gate}, cd to gate and: 
\begin{enumerate}
\item
Type: {\tt \newline
ant
}
\item
$[$optional$]$ To test the system:
{\tt \newline
ant test\newline
}
%(Note that DB tests may fail unless you can connect to Sheffield's Oracle
%server.)
\item
$[$optional$]$ To make the Javadoc documentation:
{\tt \newline
ant doc\newline
}
\item
You can also run GATE Developer using Ant, by typing:\newline
{\tt ant run}
\item
To see a full list of options type: {\tt ant help}\newline
\end{enumerate}

(The details of the build process are all specified by the build.xml file 
in the gate directory.)

You can also use a development environment like Eclipse (the required .project
file and other metadata are included with the sources), but note that it's still
advisable to use ant to generate documentation, the jar file and so on. Also
note that the run configurations have the location of a {\tt gate.xml} site
configuration file hard-coded into them, so you may need to change these for
your site.

%%%%%%%%%%%%%%%%%%%%%%%%%%%%%%%%%%%%%%%%%%%%%%%%%%%%%%%%%%%%%%%%%%%%%%%%%%%%%
\subsect[sec:gettingstarted:maven]{Using GATE with Maven/Ivy}
%%%%%%%%%%%%%%%%%%%%%%%%%%%%%%%%%%%%%%%%%%%%%%%%%%%%%%%%%%%%%%%%%%%%%%%%%%%%%

This section is based on contributions by Marin Nozhchev (Ontotext) and Benson
Margulies (Basis Technology Corp).

Stable releases of GATE (since 5.2.1) are available in the standard central
Maven repository, with group ID ``uk.ac.gate'' and artifact ID ``gate-core''.
To use GATE in a Maven-based project you can simply add a dependency:
\begin{small}
\begin{verbatim}
<dependency>
  <groupId>uk.ac.gate</groupId>
  <artifactId>gate-core</artifactId>
  <version>6.0</version>
</dependency>
\end{verbatim}
\end{small}

Similarly, with a project that uses Ivy for dependency management:
\begin{small}
\begin{verbatim}
<dependency org="uk.ac.gate" name="gate-core" rev="6.0"/>
\end{verbatim}
\end{small}

In addition you will require the matching versions of any GATE plugins you wish
to use in your application -- these are not managed by Maven or Ivy, but can be
obtained from the standard GATE release download or downloaded using the GATE
Developer plugin manager as appropriate.

Nightly snapshot builds of gate-core are available from our own Maven
repository at \verb|http://repo.gate.ac.uk/content/groups/public|.

%%%%%%%%%%%%%%%%%%%%%%%%%%%%%%%%%%%%%%%%%%%%%%%%%%%%%%%%%%%%%%%%%%%%%%%%%%%%%
\sect[sec:gettingstarted:uninstalling]{Uninstalling GATE}
%%%%%%%%%%%%%%%%%%%%%%%%%%%%%%%%%%%%%%%%%%%%%%%%%%%%%%%%%%%%%%%%%%%%%%%%%%%%%

If you have used the installer, run:

\begin{small}
\begin{verbatim}
java -jar uninstaller.jar
\end{verbatim}
\end{small}

or just delete the whole of the installation directory (the one containing
bin, lib, Uninstaller, etc.). The installer doesn't install anything outside
this directory, but for completeness you might also want to delete the
settings files GATE creates in your home directory (.gate.xml and
.gate.session).

%%%%%%%%%%%%%%%%%%%%%%%%%%%%%%%%%%%%%%%%%%%%%%%%%%%%%%%%%%%%%%%%%%%%%%%%%%%%%
\sect[sec:gettingstarted:troubleshooting]{Troubleshooting}
%%%%%%%%%%%%%%%%%%%%%%%%%%%%%%%%%%%%%%%%%%%%%%%%%%%%%%%%%%%%%%%%%%%%%%%%%%%%%

See the \htlink{http://gate.ac.uk/wiki/gate-user-faq.html}{FAQ on the GATE
Wiki} for frequent questions about running and using GATE.

%%%%%%%%%%%%%%%%%%%%%%%%%%%%%%%%%%%%%%%%%%%%%%%%%%%%%%%%%%%%%%%%%%%%%%%%%%%%%
