%%%%%%%%%%%%%%%%%%%%%%%%%%%%%%%%%%%%%%%%%%%%%%%%%%%%%%%%%%%%%%%%%%%%%%%%%%%%%
%
% design.tex
%
% hamish, 25/8/1
%
% $Id: design.tex,v 1.7 2005/02/10 17:08:59 ian Exp $
%
%%%%%%%%%%%%%%%%%%%%%%%%%%%%%%%%%%%%%%%%%%%%%%%%%%%%%%%%%%%%%%%%%%%%%%%%%%%%%


%%%%%%%%%%%%%%%%%%%%%%%%%%%%%%%%%%%%%%%%%%%%%%%%%%%%%%%%%%%%%%%%%%%%%%%%%%%%%
\chapt[chap:design]{Design Notes}
\markboth{Design Notes}{Design Notes}
%%%%%%%%%%%%%%%%%%%%%%%%%%%%%%%%%%%%%%%%%%%%%%%%%%%%%%%%%%%%%%%%%%%%%%%%%%%%%

%%%% qqqqqqqqqqqqqqqqqqqqqqqqq %%%%
\begin{quote}
Why has the pleasure of slowness disappeared? Ah, where have they gone,
the amblers of yesteryear? Where have they gone, those
loafing heroes of folk song, those vagabonds who roam from one mill to
another and bed down under the stars? Have they vanished
along with footpaths, with grasslands and clearings, with nature? There
is a Czech proverb that describes their easy indolence by a
metaphor: `they are gazing at God's windows.' A person gazing at God's
windows is not bored; he is happy. In our world, indolence
has turned into having nothing to do, which is a completely different
thing: a person with nothing to do is frustrated, bored, is
constantly searching for an activity he lacks. 

{\it Slowness}, Milan Kundera, 1995  (pp. 4-5).
\end{quote}
%%%% qqqqqqqqqqqqqqqqqqqqqqqqq %%%%


GATE is a backplane into which specialised Java Beans plug.
These beans are loose-coupled with respect to each other - they communicate
entirely by means of the GATE framework.
Inter-component communication is handled by model components -
LanguageResources, and events.

Components are defined by conformance to various interfaces (e.g.
LanguageResource), ensuring
separation of interface and implementation.

The reason for adding to the normal bean initialisation mech is that
LRs, PRs and VRs all have characteristic parameterisation
phases; the GATE resources/components model makes explicit these phases.


%%%%%%%%%%%%%%%%%%%%%%%%%%%%%%%%%%%%%%%%%%%%%%%%%%%%%%%%%%%%%%%%%%%%%%%%%%%%%
\sect[sec:design:patterns]{Patterns}
%%%%%%%%%%%%%%%%%%%%%%%%%%%%%%%%%%%%%%%%%%%%%%%%%%%%%%%%%%%%%%%%%%%%%%%%%%%%%

GATE is structured around a number of what we might call principles, or
patterns, or alternatively, clever ideas stolen from better minds than
mine. These patterns are:
\begin{itemize}
\item 
modelling most things as extensible sets of components (cf.\ Section~\ref{sec:design:components});
\item 
separating components into
model, view, or controller (cf.\ Section~\ref{sec:design:mvc}) types;
\item 
hiding implementation behind interfaces (cf.\ Section~\ref{sec:design:interfaces}).
\end{itemize}

Four interfaces in the top-level package describe the GATE view of
components:
Resource, ProcessingResource, LanguageResource and VisualResource.


%%%%%%%%%%%%%%%%%%%%%%%%%%%%%%%%%%%%%%%%%%%%%%%%%%%%%%%%%%%%%%%%%%%%%%%%%%%%%
\subsect[sec:design:components]{Components}
%%%%%%%%%%%%%%%%%%%%%%%%%%%%%%%%%%%%%%%%%%%%%%%%%%%%%%%%%%%%%%%%%%%%%%%%%%%%%

%%%%%%%%%%%%%%%%%%%%%%%%%%%%%%%%%%%%%%%%%%%%%%%%%%%%%%%%%%%%%%%%%%%%%%%%%%%%%
\subsubsect{Architectural Principle}
%%%%%%%%%%%%%%%%%%%%%%%%%%%%%%%%%%%%%%%%%%%%%%%%%%%%%%%%%%%%%%%%%%%%%%%%%%%%%


Wherever users of the architecture may wish to extend the set of a
particular type of entity, those types should be expressed as components.

Another way to express this is to say that the architecture is based on
{\em agents}. I've avoided this in the past because of an association
between this term and the idea of bits of code moving around between
machines of their own volition. I take this to be somewhat pointless, and
probably the result of an anthropomorphic obsession with mobility as a
correlate of intelligence. If we drop this connotation, however, we can
say that GATE is an agent-based architecture. If we want to, that is.
%%%%%%%%%%%%%%%%%%%%%%%%%%%%%%%%%%%%%%%%%%%%%%%%%%%%%%%%%%%%%%%%%%%%%%%%%


%%%%%%%%%%%%%%%%%%%%%%%%%%%%%%%%%%%%%%%%%%%%%%%%%%%%%%%%%%%%%%%%%%%%%%%%%%%%%
\subsubsect{Framework Expression}
%%%%%%%%%%%%%%%%%%%%%%%%%%%%%%%%%%%%%%%%%%%%%%%%%%%%%%%%%%%%%%%%%%%%%%%%%%%%%

Many of the classes in the framework are components, by which we mean
classes that conform to an interface with certain standard properties. In
our case these properties are based on the Java Beans component
architecture, with the addition of component metadata, automated loading
and standardised storage, threading and distribution.

All components inherit from Resource, via one of the three sub-interfaces
LanguageResource (LR), VisualResource (VR) or ProcessingResource (PR)
VisualResources (VRs) are straightforward -- they represent visualisation and
editing components that participate in GUIs -- but the distinction between
language and processing resources merits further discussion.

Like other software, LE programs consist of data and algorithms. The
current orthodoxy in software development is to model both data and
algorithms together, as {\it objects}\footnote{Older
development methods like Jackson Structured Design \cite{Jac75}
or Structured Analysis \cite{You89} kept them
largely separate.}. Systems that adopt the new approach are
referred to as Object-Oriented (OO), and there are good reasons to believe
that OO software is easier to build and maintain than other varieties
\cite{Boo94,You96}.

In the domain of human language processing R\&D, however, the
terminology is a little more complex.
Language data, in various forms, is of such
significance in the field that it is frequently worked on independently
of the algorithms that process it. For example: a treebank\footnote{A
corpus of texts annotated with syntactic analyses.}
can be developed independently of the
parsers that may later be trained from it; a thesaurus can be developed
independently of the query expansion or sense tagging mechanisms that may later
come to use it. This type of data has come to have its
own term, {\it Language Resources} (LRs) \cite{Lrec98}, covering many
data sources, from lexicons to corpora.

In recognition of this distinction, we will adopt the following
terminology:
%
\begin{description}
%
\item[Language Resource (LR):] refers to data-only resources such as
lexicons, corpora, thesauri or ontologies.
Some LRs come with software (e.g. Wordnet has both a user
query interface and C and Prolog APIs), but where this is only a means of
accessing the underlying data we will still define such resources as LRs.
%
\item[Processing Resource (PR):] refers to resources whose character is
principally programmatic or algorithmic, such as lemmatisers, generators,
translators, parsers or speech recognisers. For example, a part-of-speech
tagger is best characterised by reference to the process it performs on
text. PRs typically {\it include} LRs, e.g. a tagger often has a lexicon; a
word sense disambiguator uses a dictionary or thesaurus.
%
\end{description}
%
Additional terminology worthy of note in this context:
{\it language data}
refers to LRs which are at their core examples of language in practice, or
`performance data', e.g. corpora of texts
or speech recordings (possibly including
added descriptive information as markup);
{\it data about language} refers to LRs which are purely descriptive, such as a
grammar or lexicon.

PRs can be viewed as algorithms that map between different types of LR,
and which typically use LRs in the mapping process. An MT engine, for
example, maps a monolingual corpus into a multilingual aligned corpus
using lexicons, grammars, etc.\footnote{This point is due to Wim Peters.}

Further support for the PR/LR terminology may be gleaned from the argument in
favour of declarative data structures for grammars, knowledge bases, etc. This
argument was current in the late 1980s and early 1990s \cite{Gaz89},
partly as a response to what has been seen as the overly
procedural nature of previous techniques such as augmented transition networks.
Declarative structures represent a separation between data about language and
the algorithms that use the data to perform language processing tasks; a similar
separation to that used in GATE.

Adopting the PR/LR distinction is a matter of conforming to established domain
practice and terminology. It does not imply that we cannot model the domain (or
build software to support it) in an Object-Oriented manner; indeed the models
in GATE are themselves Object-Oriented.

%%%%%%%%%%%%%%%%%%%%%%%%%%%%%%%%%%%%%%%%%%%%%%%%%%%%%%%%%%%%%%%%%%%%%%%%%
\subsect[sec:design:mvc]{Model, view, controller}
%%%%%%%%%%%%%%%%%%%%%%%%%%%%%%%%%%%%%%%%%%%%%%%%%%%%%%%%%%%%%%%%%%%%%%%%%

According to Buschmann et al
(Pattern-Oriented Software Architecture, 1996),
the Model-View-Controller (MVC)
pattern 
\begin{quotation} 
...divides an interactive application into three components. The model
contains the core functionality and data. Views display information to the
user. Controllers handle user input. Views and controllers together comprise
the user interface. A change-propagation mechanism ensures consistency between
the user interface and the model. \mbox{$[$}p.125\mbox{$]$}\end{quotation}
A variant of MVC, the Document-View pattern, 
\begin{quotation} 
...relaxes the separation of view and controller... The View component of
Document-View combines the responsibilities of controller and view in MVC, and
implements the user interface of the system.\end{quotation}
A benefit of both arrangements is that 
\begin{quotation} 
...loose coupling of the document and view components enables multiple
simultaneous synchronized but different views of the same document.\end{quotation}

Geary (Graphic Java 2, 3rd Edtn., 1999) gives a slightly different view:
\begin{quotation} 
MVC separates applications into three types of objects:
\begin{itemize}
\item 
{\bf Models:} Maintain data and provide data accessor methods
\item 
{\bf Views:} Paint a visual representation of some or all of a model's data
\item 
{\bf Controllers:} Handle events
... By encapsulating what other architectures intertwine, MVC applications are
much more flexible and reusable than their traditional counterparts.
\end{itemize}
\mbox{$[$}pp. 71, 75\mbox{$]$}\end{quotation}
Swing, the Java user interface framework, uses
\begin{quotation} 
a specialised version of the classic MVC meant to support pluggable look and
feel instead of applications in general.
\mbox{$[$}p. 75\mbox{$]$}\end{quotation}

GATE may be regarded as an MVC architecture in two ways:
\begin{itemize}
\item 
directly, because we use the Swing toolkit for the GUIs;
\item 
by analogy, where LRs are models, VRs are views and PRs are controllers.
Of these, the latter sits least
easily with the MVC scheme, as PRs may indeed be controllers but may also
not be.
\end{itemize}


%%%%%%%%%%%%%%%%%%%%%%%%%%%%%%%%%%%%%%%%%%%%%%%%%%%%%%%%%%%%%%%%%%%%%%%%%%%%%
\subsect[sec:design:interfaces]{Interfaces}
%%%%%%%%%%%%%%%%%%%%%%%%%%%%%%%%%%%%%%%%%%%%%%%%%%%%%%%%%%%%%%%%%%%%%%%%%%%%%

%%%%%%%%%%%%%%%%%%%%%%%%%%%%%%%%%%%%%%%%%%%%%%%%%%%%%%%%%%%%%%%%%%%%%%%%%%%%%
\subsubsect{Architectural Principle}
%%%%%%%%%%%%%%%%%%%%%%%%%%%%%%%%%%%%%%%%%%%%%%%%%%%%%%%%%%%%%%%%%%%%%%%%%%%%%

The implementation of types should generally be hidden from the clients of
the architecture.


%%%%%%%%%%%%%%%%%%%%%%%%%%%%%%%%%%%%%%%%%%%%%%%%%%%%%%%%%%%%%%%%%%%%%%%%%%%%%
\subsubsect{Framework Expression}
%%%%%%%%%%%%%%%%%%%%%%%%%%%%%%%%%%%%%%%%%%%%%%%%%%%%%%%%%%%%%%%%%%%%%%%%%%%%%

With a few exceptions (such as for utility classes),
clients of the framework work with the {\tt gate.*} package.
This package is mostly composed of interface definitions.
Instantiations of these interfaces are obtained via the {\tt Factory}
class.

The subsidiary packages of GATE provide the implementations of the
{\tt gate.*} interfaces that are accessed via the factory. They
themselves avoid directly constructing classes from other packages
(with a few exceptions, such as JAPE's need for unattached annotation
sets). Instead they use the factory.


%%%%%%%%%%%%%%%%%%%%%%%%%%%%%%%%%%%%%%%%%%%%%%%%%%%%%%%%%%%%%%%%%%%%%%%%%%%%%
\sect[sec:design:exceptions]{Exception Handling}
%%%%%%%%%%%%%%%%%%%%%%%%%%%%%%%%%%%%%%%%%%%%%%%%%%%%%%%%%%%%%%%%%%%%%%%%%%%%%

When and how to use exceptions? Borrowing from Bill Venners, here
are some {\bf guidelines} (with examples):

\begin{enumerate}
\item 
Exceptions exist to refer problem conditions up the call stack to a level at
which they may be dealt with.
{\tt{}"{}}If your method encounters an abnormal condition {\em that it can't handle},
it should throw an exception.{\tt{}"{}} If the method can handle the problem rationally,
it should catch the exception and deal with it.\newline

{\bf Example:}\newline
If the creation of a resource such as a document requires a URL as a parameter,
the method that does the creation needs to construct the URL and read from it. If
there is an exception during this process, the GATE method should abort by throwing
its own exception. The exception will be dealt with higher up the food chain, e.g. 
by asking the user to input another URL, or by aborting a batch script.

\item 
All GATE exceptions should inherit from gate.util.GateException (a descendant of
java.lang.Exception, hence a checked exception) or
gate.util.GateRuntimeException (a descendant of java.lang.RuntimeException,
hence an unchecked exception). This rule means that clients of GATE code can
catch all sorts of exceptions thrown by the system with only two catch
statements. (This rule may be broken by methods that are not public, so long as
their callers catch the non-GATE exceptions and deal with them or convert them
to GateException/GateRuntimeException.)
Almost {\bf all} exceptions thrown by GATE should be checked exceptions: the
point of an exception is that clients of your code get to know about it, so use
a checked exception to make the compiler force them to deal with it. Except:\newline

{\bf Example:}\newline
With reference to the previous example, a problem using the URL will be
signalled by something like an UnknownHostException or an IOException. These
should be caught and re-thrown as descendants of GateException.

\item 
In a situation where an exceptional condition is an indication of a bug in the
GATE library, or in the implementation of some other library, then it is
permissible to throw an unchecked exception.\newline

{\bf Example:}\newline
If a method is creating annotations on a document, and before creating the
annotations it checks that their start and end points are valid ranges in
relation to the content of the document (i.e. they fall within the offset space
of the document, and the end is after the start), then if the method receives an
InvalidOffsetException from the AnnotationSet.add call, something is seriously
wrong. In such cases it may be best to throw a GateRuntimeException.

\item 
Where you are inheriting from a non-GATE class and therefore have the exception
signatures fixed for you, you may add a new exception deriving from a non-GATE
class.\newline

{\bf Example:}\newline
The SAX XML parser API uses SaxException. Implementing a SAX parser for a
document type involves overriding methods that throw this exception. Where you
want to have a subtype for some problem which is specific to GATE processing,
you could use GateSaxException which extends SaxException.

\item 
Test code is different: in the JUnit test cases it is fine just to declare that
each method throws Exception and leave it at that. The JUnit test runner will
pick up the exceptions and report them to you. Test methods should, however,
try and ensure that the exceptions thrown are meaningful. For example, avoid
null pointer exceptions in the test code itself, e.g. by using assertNonNull.\newline

{\bf Example:}
\begin{lstlisting}  
  public void testComments() throws Exception { 
    ResourceData docRd = (ResourceData) reg.get("gate.Document"); 
    assertNotNull(
      "testComments: couldn't find document res data", docRd
    ); 
    String comment = docRd.getComment(); 
    assert( 
      "testComments: incorrect or missing COMMENT on document", 
      comment != null && comment.equals("GATE document") 
    ); 
  } // testComments() 
\end{lstlisting}

See also the testing notes.

\item 
{\tt{}"{}}Throw a different exception type for each abnormal condition.{\tt{}"{}} You can go too
far on this one - a hundred exception types per package would certainly be too
much - but in general you should create a new exception type for each different
sort of problem you encounter.\newline

{\bf Example:}\newline
The gate.creole package has a ResourceInstantiationException - this deals with
all problems to do with creating resources. We could have had
{\tt{}"{}}ResourceUrlProblem{\tt{}"{}} and {\tt{}"{}}ResourceParameterProblem{\tt{}"{}} but that would probably have
ended up with too many. On the other hand, just throwing everything as
GateException is too coarse (Hamish take note!).

\item 
Put exceptions in the package that they're thrown from (unless they're used in
many packages, in which case they can go in gate.util). This makes it easier to
find them in the documentation and prevents name clashes.\newline

{\bf Example:}\newline
gate.jape.ParserException is correctly placed; if it was in gate.util it might
clash with, for example, gate.xml.ParserException if there was such.
\end{enumerate}
